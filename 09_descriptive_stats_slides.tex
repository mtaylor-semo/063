%!TEX TS-program = lualatex
%!TEX encoding = UTF-8 Unicode

\documentclass[t]{beamer}

%%%% HANDOUTS For online Uncomment the following four lines for handout
%\documentclass[t,handout]{beamer}  %Use this for handouts.
%\usepackage{handoutWithNotes}
%\pgfpagesuselayout{3 on 1 with notes}[letterpaper,border shrink=5mm]


%%% Including only some slides for students.
%%% Uncomment the following line. For the slides,
%%% use the labels shown below the command.
%\includeonlylecture{student}

%% For students, use \lecture{student}{student}
%% For mine, use \lecture{instructor}{instructor}


%\usepackage{pgf,pgfpages}
%\pgfpagesuselayout{4 on 1}[letterpaper,border shrink=5mm]

% FONTS
\usepackage{fontspec}
\def\mainfont{Linux Biolinum O}
\setmainfont[Ligatures={Common,TeX}, Contextuals={NoAlternate}, BoldFont={* Bold}, ItalicFont={* Italic}, Numbers={OldStyle, Proportional}]{\mainfont}
\setsansfont[Scale=MatchLowercase, Numbers=OldStyle]{Linux Biolinum O} 
\usepackage{microtype}

\usepackage{unicode-math}
\setmathfont[Scale=MatchLowercase]{TeX Gyre Termes Math}

\usepackage{graphicx}
	\graphicspath{{/Users/goby/pictures/teach/163/lecture/}
	{/Users/goby/pictures/teach/163/lab/}} % set of paths to search for images

\usepackage{amssymb, amsmath}
\usepackage{mathtools}
	\everymath{\displaystyle}

%\usepackage{units}
\usepackage{booktabs}
%\usepackage{textcomp}

\usepackage{tikz}
%	\tikzstyle{every picture}+=[remember picture,overlay]

\mode<presentation>
{
  \usetheme{Lecture}
  \setbeamercovered{invisible}
%  \setbeamertemplate{items}[square]
}

\usefonttheme[onlymath]{serif}
\newcommand*\meanY{\overline{Y\kern1.67pt}\kern-1.67pt}
\newcommand*\meansubY{\overline{Y}}

\begin{document}

\lecture{student}{student}

\begin{frame}[t]{Our goals for this lab are to learn}

	\hangpara the difference between null and research hypotheses,
		
	\hangpara the difference between explanatory and response variables,
	
	\hangpara how to calculate the mean, standard deviation, and standard error for a sample,
	
	\hangpara how to use a $t$-test to test your hypotheses.
	
\end{frame}
%
\begin{frame}{\highlight{Carefully and thoroughly} study the lab handout. On an exam, you might be asked to}

\hangpara write null and research hypotheses,

\hangpara identify explanatory and response variables,

\hangpara calculate one or more descriptive statistics, and

\hangpara interpret a $t$-test to decide whether to accept or reject a null hypothesis.

\hangpara We might even ask questions about the lecture material!

\end{frame}
%
\begin{frame}{\highlight{Hypothesis testing} uses data sampled from one or more populations to make inferences about those populations}

\hangpara  \highlight{Null hypothesis:} There is \emph{no} difference between between populations or samples. 	Any differences are probably due to natural variation and other random effects. 
	
\hangpara  \highlight{Research hypothesis:} There \emph{is} a difference between populations or samples. Or, changing a variable will affect the population(s). Any differences are probably not random
	and have a biological explanation. 
	
\end{frame}
%
\begin{frame}{Scientists usually manipulate and measure variables.}

\hangpara  \highlight{Explanatory variable:} a manipulated or controlled variable in an experiment
  or study whose presence or degree determines a change in the response
  variable.  
	
\hangpara  \highlight{Response hypothesis:} an observed variable in an experiment or
  study that changes in response to the presence or degree of one
  or more explanatory variables.  
	
\end{frame}
%
\begin{frame}[t]{Take 15 minutes to read the experiment on pages 2–3 and answers questions 1–6.}

\end{frame}
%
\begin{frame}[t]{\highlight{Descriptive statistics} are numbers that summarize data and estimate population parameters.}

\hangpara Common descriptive statistics include the:

\hangpara \hspace{1em} \highlight{mean,}

\hangpara \hspace{1em} \highlight{standard deviation,} and

\hangpara \hspace{1em} \highlight{standard error.}

\end{frame}
%
\begin{frame}{\highlight{Hypothesis testing} asks whether the data sampled would be expected under a specific null hypothesis.}

\end{frame}
%
{
%\usebackgroundtemplate{\includegraphics[width=\paperwidth]{09_normal_curve}}
\begin{frame}[t]{The arithmetic \highlight{mean} is a measure of central tendency in the data.}

	\vspace*{-\baselineskip}
	
	{\centering 
	\includegraphics[width=\textwidth]{09_normal_distribution} \par
	}
	
	\hangpara The U.S. adult population height: $\meanY = 66.5 \pm 4.43$ inches.
	
\end{frame}
}
%
\begin{frame}[t]{The formula to calculate the mean $\left(\meanY\kern1.67pt\right)$ of a sample is}

{\Large
\[ \meanY = \frac{\sum\limits^n_{i=1} Y_i}{n} \]
}

\pause\hangpara e.g.,

\[ \meanY = \frac{66.0 + 68.7 + 64.2 + 59.6}{4} = 64.6\,\mathrm{inches.} \]

\end{frame}
%
\begin{frame}[t]{The \highlight{standard deviation} is a measure of variability \emph{within} the data.}

	\vspace*{-0.5\baselineskip}
	
	{\centering 
	\includegraphics[width=\textwidth]{09_standard_deviation} \par
	}
	
	\hangpara U.S. adult height: $\meanY = 66.5 \pm 2$ inches (left) or $ \pm\ 8$ inches (right). 
	
\end{frame}
%
\begin{frame}[t]{The formula to calculate the standard deviation $(s)$ of a sample is}

{\Large
\[ s = \sqrt{\frac{\sum\left(Y_i - \meanY\kern1.67pt\right)^2}{n-1}} \]
}

\hangpara The standard deviation tells how far on average any
\emph{one} measurement is from the mean of \emph{all} measurements in 
the data. 

\end{frame}
%
\begin{frame}[t]{The \highlight{standard error} is a measure of variability of the estimated mean.}

	\vspace*{-\baselineskip}
	
	{\centering 
	\includegraphics[width=\textwidth]{09_standard_error} \par
	}
	
	\hangpara Repeated samples yield different means due to random sampling. 
	
\end{frame}
%
\begin{frame}[t]{The formula to calculate the standard error of the mean for a sample is}

{\Large
\[ \mathrm{SE}_{\meansubY} = \frac{s}{\sqrt{n}} \]
}

\hangpara Larger sample sizes increase the precision of the estimated mean.

\end{frame}
%
%
\begin{frame}[t]{\highlight{Hypothesis testing} asks whether the sampled data would be expected under a specific null hypothesis.}

	\vspace{-\baselineskip}
	
	{\centering
	\includegraphics[height=0.85\textheight]{09_adult_height}\par
	}

\end{frame}
%
\begin{frame}[t]{The null hypothesis is that male and female height were sampled from the same statistical population.}

	\vspace{-\baselineskip}

	{\centering
	\includegraphics[height=0.85\textheight]{09_adult_height_male_female}\par
	}

\end{frame}
%
\begin{frame}{The $t$-test determines if the means of two independent samples  were taken from the same statistical population.}

	\vspace{-\baselineskip}
	
	{\centering
	\includegraphics[width=\textwidth]{09_t_distrib}\par
	}

\end{frame}
%
\begin{frame}[t]{The formula for the $t$-test is}

\begin{equation*}
t = \dfrac{\meanY_1 - \meanY_2}{\sqrt{\dfrac{\left(n_1 - 1\right)s_1^2 + \left(n_2 - 1\right)s_2^2}{n_1 + n_2 -2} \times \left[\dfrac{1}{n_1} + \dfrac{1}{n_2}\right]}}
\end{equation*}

\end{frame}
%


\end{document}
