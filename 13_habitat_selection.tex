%!TEX TS-program = lualatex
%!TEX encoding = UTF-8 Unicode

\documentclass[12pt, hidelinks]{exam}

\printanswers

\usepackage{graphicx}
	\graphicspath{{/Users/goby/Pictures/teach/163/lab/}
	{img/}} % set of paths to search for images

\usepackage{geometry}
\geometry{letterpaper, left=1.5in, bottom=1in}                   
%\geometry{landscape}                % Activate for for rotated page geometry
\usepackage[parfill]{parskip}    % Activate to begin paragraphs with an empty line rather than an indent
\usepackage{amssymb, amsmath}
\usepackage{mathtools}
	\everymath{\displaystyle}

\usepackage{fontspec}
\setmainfont[Ligatures={TeX}, BoldFont={* Bold}, ItalicFont={* Italic}, BoldItalicFont={* BoldItalic}, Numbers={Proportional, OldStyle}]{Linux Libertine O}
\setsansfont[Scale=MatchLowercase,Ligatures=TeX, Numbers={Proportional,OldStyle}]{Linux Biolinum O}
\setmonofont[Scale=MatchLowercase]{Linux Libertine Mono O}
\newfontfamily{\liningnum}[Numbers=Lining]{Linux Libertine O}
\usepackage{microtype}

\usepackage[table]{xcolor}

\usepackage{unicode-math}
\setmathfont[Scale=MatchLowercase]{Tex Gyre Pagella Math}


% To define fonts for particular uses within a document. For example, 
% This sets the Libertine font to use tabular number format for tables.
 %\newfontfamily{\tablenumbers}[Numbers={Monospaced}]{Linux Libertine O}
% \newfontfamily{\libertinedisplay}{Linux Libertine Display O}

\usepackage{booktabs}
\usepackage{multicol}

\usepackage{caption}
\captionsetup{format=plain, justification=raggedright, singlelinecheck=off,labelsep=period,skip=3pt} % Removes colon following figure / table number.

%\usepackage{caption}
%\captionsetup{font=small} 
%\captionsetup{singlelinecheck=false}
%\captionsetup[figure]{labelsep=period, format=plain}

\usepackage{longtable}
%\usepackage{siunitx}
\usepackage{array}
\newcolumntype{L}[1]{>{\raggedright\let\newline\\\arraybackslash\hspace{0pt}}p{#1}}
\newcolumntype{C}[1]{>{\centering\let\newline\\\arraybackslash\hspace{0pt}}p{#1}}
\newcolumntype{R}[1]{>{\raggedleft\let\newline\\\arraybackslash\hspace{0pt}}p{#1}}

\usepackage{enumitem}
\setlist{leftmargin=*}
\setlist[1]{labelindent=\parindent}
\setlist[enumerate]{label=\textsc{\alph*}.}
\setlist[itemize]{label=\color{gray}\textbullet}

\usepackage{hyperref}
%\usepackage{placeins} %PRovides \FloatBarrier to flush all floats before a certain point.
\usepackage{hanging}

\usepackage[sc]{titlesec}

%% Commands for Exam class
\renewcommand{\solutiontitle}{\noindent}
\unframedsolutions
\SolutionEmphasis{\bfseries}

\renewcommand{\questionshook}{%
	\setlength{\leftmargin}{-\leftskip}%
}

%Change \half command from 1/2 to .5
\renewcommand*\half{.5}

\pagestyle{headandfoot}
\firstpageheader{\textsc{bi}\,063 Evolution and Ecology}{}{\ifprintanswers\textbf{KEY}\else Name: \enspace \makebox[2.5in]{\hrulefill}\fi}
\runningheader{}{}{\footnotesize{pg. \thepage}}
\footer{}{}{}
\runningheadrule

\newcommand*\AnswerBox[2]{%
    \parbox[t][#1]{0.92\textwidth}{%
    \begin{solution}#2\end{solution}}
    \vspace{\stretch{1}}
}

\newenvironment{AnswerPage}[1]
    {\begin{minipage}[t][#1]{0.92\textwidth}%
    \begin{solution}}
    {\end{solution}\end{minipage}
    \vspace{\stretch{1}}}

\newlength{\basespace}
\setlength{\basespace}{5\baselineskip}


\newcommand\chisq{$\chi^2$}
\newcommand*\meanY{\overline{Y}\kern0.67pt}

\newcommand*\AnswerBlank[1]{%
	\ifprintanswers%
		\textbf{#1}
	\else%
		\rule{0.75in}{0.4pt}\kern0.67pt.\fi%
	}

%\newcommand*\AnswerBlank{\rule{0.75in}{0.4pt}\kern0.67pt.}
\newcommand*\xcell[1]{cell~\liningnum{#1}}

%
%\makeatletter
%\def\SetTotalwidth{\advance\linewidth by \@totalleftmargin
%\@totalleftmargin=0pt}
%\makeatother



\begin{document}

\subsection*{Testing for habitat selection by pill bugs.}

Organisms choose habitats that provide the best \emph{available} conditions for survival and reproduction.  While the best available habitat does not guarantee increased fitness, that organism has a better chance for maximum fitness in the proper habitat.  Habitats should provide shelter from inclement weather, protection from predators, suitable amounts of food and water, and other resources required by the organism. 

For this experiment, you will test whether \textit{Armadillidium} pill bugs (roly-polies, doodle bugs) have a preference for the amount of moisture in their habitat. They require moisture to keep their gills wet for respiration; therefore, pill bugs can be found under rotting logs and in leaf litter where moisture is always present. The pill bugs will be able to choose from among three moisture levels: dry, moderate, and high.

\begin{questions}

\question \label{ques:hypothesis}
Write a hypothesis that states which moisture level (or levels) the pill bugs will choose?

\AnswerBox{0.35\basespace}{Most will probably choose one of the moister levels.}


\question
Turn your hypothesis is a prediction. Remember that your prediction should predict the results of the experiment that would support your hypothesis. 

\AnswerBox{0.35\basespace}{%
	If given a choice of three moisture levels, most pill bugs will end up in the chamber with \rule{0.75in}{0.4pt} amount of moisture.
}

You will create a small habitat for sow bugs.  Your habitat will contain three interconnected chambers, each with a different amount of moisture.  One chamber will have zero moisture, another chamber will contain 7 ml of water, and the third chamber will contain 15 ml of water.


\subsubsection*{Methods}

\begin{enumerate}
	
	\item Work in teams of 2–3 students. Decide on a catchy but easy to remember name for your group.
	
	\item Carefully remove the lid from each habitat chamber. 
	
	\item Use the small glass beaker to add about 40 ml of sand to each container. It is better to be slightly under the 40 ml line than above. Put your finger over the hole leading to the next chamber and carefully pour the sand into the chamber.
	
	\item Add 7 ml of water to one chamber. Add 15 ml of water to another chamber. Leave one chamber without any added water.
	
	\item Add dried, crumbled leaf litter to just below the rim of each chamber to provide shelter for the pill bugs.
	
	\item Add the same number of pill bugs to each chamber. Your instructor will tell you how many to use.
	
	\item Assemble the three chambers into a single stack. Arrange the chambers so that the chamber with the most moisture is on the bottom and the dry chamber is on top. 
	
	\item Place a piece of masking tape on the top of the upper chamber. Write your team name and section number on the tape.
	
	\item Place the chambers in the box indicated by your instructor.
	
\end{enumerate}

\subsubsection*{Collect your data}


{\scshape \textbf{Important:}} At least one of your team must return to the lab in within two days to count the number of individuals in each chamber. Your instructor will tell you when the lab is open and the deadline to record your data. 

\begin{enumerate}
	\item Remove your chambers from the box and count the number of individuals in each chamber. \emph{Count dead individuals as being present in the chamber you find them in.}
	
	\item Write your team name and results on the sheet provided.
	
	\item Put live individuals in the tank or aquarium provided. Discard dead individuals in the trash.
	
	\item Discard the leaf litter into the trash can in the lab. Pour the sand back into the sand box. Use a paper towel to wipe out the containers. Remove the tape from the lid.
	
	\item Reassemble your empty and cleaned chambers and place them on the back counter. 
	
	\item \textbf{If your team does not report its results on the official sheet, you will receive at most 50\% of the total possible points for this exercise. Be sure at least one of you returns to lab, count pill bugs, and reports the results.}
		
\end{enumerate}

\subsubsection*{Final results and analysis}

The results from the team experiments will be pooled together to make the final data set that you will use for analysis. A spreadsheet with all of the data will be available to you on the course website after all results have been reported. You will use the data to perform a \chisq{} (chi-squared) test to determine whether the pill bugs were randomly distributed or chose among the offered habitats.

\subsubsection*{The $\chi^2$ test.}

The \chisq{} test is used to determine whether an observed pattern differs from an expected pattern. For example, if pill bugs are not choosing their habitats, then roughly equal numbers of individuals in should be present in each habitat type. 

The formula to calculate \chisq{} is

\[ \chi^2 = \sum_{i=1}^n \dfrac{(O_i-E_i)^2}{E_i}, \]

where $O_i$ is the observed value, $E_i$ is the expected value, $i$ is each observation, and $n$ is the total number of observations. This statistic  tests the \textbf{null hypothesis} that observed values are not different from expected values.

This example shows you how to calculate \chisq{}. Assume that you had $n=4$ habitats and 40 individuals. You wanted to test the research hypothesis that the individuals were choosing among habitats. The null hypothesis was that the individuals were randomly distributed among the habitats. If the distribution of individuals was random, then observed values would not differ significantly from expected values. The number of individuals observed in each habitat was:

{\liningnum
\begin{longtable}{@{}rR{0.8in}@{}}
	\toprule
	Habitat ($i$) &	Number Observed \tabularnewline
	\midrule
	1	& 7 \tabularnewline
	2	& 14  \tabularnewline
	3	& 8	 \tabularnewline
	4	& 11 \tabularnewline
	\bottomrule
\end{longtable}
}

\begin{enumerate}

\item Calculate the \emph{expected number of individuals} in each habitat. For a random distribution, each habitat should have about the same number of individuals, so divide the total number of individuals by the number of habitats:  

\[ 40/4 = 10. \]

\item Subtract the expected number of individuals from the observed number, and then square that value.

{\liningnum
\begin{longtable}{@{}rR{0.8in}R{0.8in}R{0.8in}R{0.8in}@{}}
	\toprule
	Habitat ($i$) &	Observed & Expected & $\left(O_i-E_i \right)$ & $\left(O_i-E_i \right)^2$\tabularnewline
	\midrule
	1	& 7  & 10 & -3 & 9 \tabularnewline
	2	& 14 & 10 & 4  & 16 \tabularnewline
	3	& 8	 & 10 & -2 & 4 \tabularnewline
	4	& 11 & 10 & 1  & 1 \tabularnewline
	\bottomrule
\end{longtable}
}

\item Divide each result by the expected values.

{\liningnum
\begin{longtable}{@{}rR{0.8in}R{0.8in}R{0.8in}R{0.8in}R{0.8in}@{}}
	\toprule
	Habitat ($i$) &	Observed & Expected & $\left(O_i-E_i \right)$ & $\left(O_i-E_i \right)^2$ & $\frac{\left(O_i-E_i \right)^2}{E}$\tabularnewline
	\midrule
	1	& 7  & 10 & -3 & 9  & 0.9 \tabularnewline
	2	& 14 & 10 & 4  & 16 & 1.6 \tabularnewline
	3	& 8	 & 10 & -2 & 4  & 0.4 \tabularnewline
	4	& 11 & 10 & 1  & 1  & 0.1 \tabularnewline
	\bottomrule
\end{longtable}
}

\item Add together the final results to obtain \chisq{}.

{\liningnum
\begin{longtable}{@{}rR{0.8in}R{0.8in}R{0.8in}R{0.8in}R{0.8in}@{}}
	\toprule
	Habitat ($i$) &	Observed & Expected & $\left(O_i-E_i \right)$ & $\left(O_i-E_i \right)^2$ & $\frac{\left(O_i-E_i \right)^2}{E}$\tabularnewline
	\midrule
	1	& 7  & 10 & -3 & 9  & 0.9 \tabularnewline
	2	& 14 & 10 & 4  & 16 & 1.6 \tabularnewline
	3	& 8	 & 10 & -2 & 4  & 0.4 \tabularnewline
	4	& 11 & 10 & 1  & 1  & 0.1 \tabularnewline
	\midrule
	    &    &    &    & $\chi^2=$ & 3.0 \tabularnewline
	\bottomrule
\end{longtable}
}

	\item Calculate \textbf{degrees of freedom} as the number of habitats minus 1,
	
	\[ \mathrm{df} = 4 - 1 = 3.\]

\end{enumerate}

\setcounter{table}{0}
You compare your calculated \chisq{} to a table of critical \chisq{} values, using $\alpha=0.05$ and your degrees of freedom (df). Critical values of \chisq{} are shown in Table~\ref{tab:chi_table} on page~\pageref{tab:chi_table}.


%\newsavebox\ltmcbox

%\setbox\ltmcbox\vbox{
%\makeatletter\col@number\@ne
{\setlength{\LTcapwidth}{2.2in}\liningnum
\begin{longtable}{@{}rrrrr@{}}
	\caption{Critical values of $\chi^2$.\label{tab:chi_table}} \tabularnewline
\toprule
 & \multicolumn{4}{c}{$\alpha$} \tabularnewline
 \cmidrule(l){2-5}
df & 0.1 & \textbf{0.05} & 0.01 & 0.001 \tabularnewline
\midrule
 1 &  2.71 &  3.84 &  6.63 & 10.83 \tabularnewline
 2 &  4.61 &  5.99 &  9.21 & 13.82 \tabularnewline
\textbf{3} &  6.25 &  \textbf{7.81} & 11.34 & 16.27 \tabularnewline
 4 &  7.78 &  9.49 & 13.28 & 18.47 \tabularnewline
 5 &  9.24 & 11.07 & 15.09 & 20.52 \tabularnewline
 6 & 10.64 & 12.59 & 16.81 & 22.46 \tabularnewline
 7 & 12.02 & 14.07 & 18.48 & 24.32 \tabularnewline
 8 & 13.36 & 15.51 & 20.09 & 26.12 \tabularnewline
 9 & 14.68 & 16.92 & 21.67 & 27.88 \tabularnewline
10 & 15.99 & 18.31 & 23.21 & 29.59 \tabularnewline
\bottomrule
\end{longtable}}
%\unskip
%\unpenalty
%\unpenalty}
%\unvbox\ltmcbox


If your calculated \chisq{} is \emph{less} than the critical value in the table, then you \emph{accept the null hypothesis.} In this example, for $\alpha=0.05$ and 3 degrees of freedom, $\chi^2 = 3.0$ is less than the critical value of 7.81 (bolded in the table) so you would have accepted the null hypothesis that the observed values are not significantly different from the expected values. The distribution of the 40 individuals among the 4 habitats was probably random.
	
	\question
	If the calculated value of \chisq{} is larger than the critical value in the table, would you accept or reject the \emph{research} hypothesis?
	
	\AnswerBox{0.5\basespace}{Accept the research hypothesis.}
	
	\question
	Assume that you calculated $\chi^2 = 17.03$, that you want to use $\alpha = 0.01,$ and that your study has 6 degrees of freedom. Would you accept or reject the null hypothesis? Why?

	\AnswerBox{2\basespace}{Reject the null hypothesis. The calculated value is greater than the tabled value of 16.81.}

	\question[Checkout]
	Assume you performed a similar habitat selection study with 120 individuals and 5 habitats. You observed the following number of individuals in each habitat: 23, 32, 19, 13, and 33. Calculate degrees of freedom and \chisq{}, and look up the critical value of \chisq{}.\bigskip
	
	Degrees of Freedom: \AnswerBlank{4} \bigskip 
	
	Calculated \chisq{}: \AnswerBlank{10.58} \bigskip
	
	Critical \chisq{}: \AnswerBlank{9.49}
	
	\question[Checkout]
	Do you accept or reject the null hypothesis?  Tell why.
	
	\AnswerBox{\basespace}{%
		Reject the null hypothesis. The calculated value (10.58) is larger than the critical value (9.49) so the null hypothesis is rejected.
	}
		
	Discuss your answers with your instructor to receive credit for lab today.
	
\end{questions}
	

\subsection*{Formal lab report: habitat selection by pill bugs}

You will write a formal lab report. Each student is responsible for writing his or her own lab report based on
the experiment you ran in your group, and on the data collected as a
class. Your report must be typed. Your lab instructor will tell you the due date and whether to submit it in lab or online. Your report must contain the following sections. Your instructor will tell you the due date and time.
 
A lab report is written in narrative form and in past tense—you are
explaining to the reader your hypothesis, how you tested it, and the
results from the experiment. Each lab report section should be written
in paragraph form, \emph{no bulleted lists or orphan sentences!} Keep in
mind that a paragraph has at least three sentences, each paragraph should
discuss one major point or idea, and all sentences in the paragraph
should provide support to the first sentence (your topic sentence).

Your report must include the information described below, divided into the
sections as listed. Include the section headings in your lab report in the order listed. You must 
include all information listed in each section to earn full credit. 
Use this handout to guide you as you write.

 
\subsubsection*{Introduction (1 paragraph; 5 points)} 

This section should begin with general observations made regarding
pill bugs and their habitat. Next, state your hypothesis (you developed this as a team in question~\ref{ques:hypothesis}).
From your hypothesis, make clear predictions in if/then form. Review your handouts from
the first week on the scientific method.


\subsubsection*{Methods (2 paragraphs, 5 points)} 

This section should be written in narrative form and in past tense. Do
not make a list of materials. You will mention the materials you used
throughout this section as relevant. \emph{This is not like writing a
	recipe, you are not providing instructions for your reader}. Rather, you
are explaining to the reader how you set up the experiment and collected
your data.

As you write, consider which aspects are important to report to allow
the reader to understand how you set up your experiment. For example, it
is not necessary to tell that you labeled your chambers with tape and your team name. Likewise, you do not need to tell that you used a graduated cylinder to measure water. In the methods, you might
write something like, ``We added 15 ml of water to one chamber, 7 ml of water to another chamber, and one chamber had no added water.'' (\emph{Do not use that example sentence verbatim because that would be plagiarism.})

The second paragraph should explain how you
collected your data. This section must be written in \emph{past tense} because you have
already completed the experiment and data collection. No credit will be
awarded for methods sections written in list or bullet form. As above, you do not need to tell how you disposed of the pill bugs or that your instructor provided you with the data. But, you do need to indicate that your results were pooled with the results of other experiments to obtain the final data set.

\subsubsection*{Results (1–2 paragraphs, 10 points)}

You must calculate the means and standard deviations for the number of individuals in each habitat. Refer to the lab on the mean and standard deviation if you do not remember how to calculate them in Excel. You must make a column chart that shows the mean number of individuals in each habitat as columns and the standard deviation as error bars. If necessary, refer to the exercise in the Graphing Skills lab that showed insect richness based on tree species (Lodgepole Pine or Douglas Fir). You must write a suitable legend (caption) for the figure. 

When you report your results, you must properly refer to the figure in your narrative. For example, ``The dry habitat had the highest mean number of individuals but the wet habitat had the greatest variability in number of individuals (Figure 1).'' Use the results section only used  present the data collected in a brief and
descriptive format. You explain and interpret your results in the discussion section.   

You must also report the total number of individuals in each habitat. \emph{Use the total number of individuals in each habitat to calculate} \chisq{}. Your report must include your calculated \chisq{} value, the critical \chisq{} value, and whether you accept or reject the null hypothesis. Do not give all of these values in a single sentence. Do not show all of the steps you used to calculate \chisq{}. 


\subsubsection*{Discussion (2 paragraphs, 10 points)} 

In this section, you summarize and interpret your results. Use what
was discussed in class as well as this handout to guide
your discussion. But, \emph{do not} directly copy from any material provided.
You must address the following questions in narrative form, \emph{in your own words,} with two
paragraphs. Was your hypothesis supported by your results? Explain why or why not. 
The results should agree with your prediction for your hypothesis to be supported. 
Finally, place your experimental results in the broader context of ecology or natural
selection. Explain why your results might be important to the
scientific community.

\subsubsection*{The \chisq{} distribution}

The critical values in Table~\ref{tab:chi_table} are based on the \chisq{} distribution. Values of \chisq{} greater than the critical value (Figure~\ref{fig:chi_square}, shaded area) indicate that observed values are significantly different from expected values. Calculated values less than the critical value (Figure~\ref{fig:chi_square}, unshaded area) indicate that differences are probably random and not significant.

\hfil\begin{minipage}{0.8\textwidth}
	\includegraphics[width=\columnwidth]{13_chi_distrib}
	\captionof{figure}{\chisq{} distribution for $\alpha = 0.05$ and $3$ degrees of freedom.\label{fig:chi_square}}
\end{minipage}\hfill
 

\end{document}  