%!TEX TS-program = lualatex
%!TEX encoding = UTF-8 Unicode

\documentclass[12pt]{exam}
\usepackage{graphicx}
	\graphicspath{{/Users/goby/Pictures/teach/163/lab/}
	{img/}} % set of paths to search for images

\usepackage{geometry}
\geometry{letterpaper, left=1.5in, bottom=1in}                   
%\geometry{landscape}                % Activate for for rotated page geometry
\usepackage[parfill]{parskip}    % Activate to begin paragraphs with an empty line rather than an indent
\usepackage{amssymb, amsmath}
\usepackage{mathtools}
	\everymath{\displaystyle}

\usepackage{fontspec}
\setmainfont[Ligatures={TeX}, BoldFont={* Bold}, ItalicFont={* Italic}, BoldItalicFont={* BoldItalic}, Numbers={Proportional}]{Linux Libertine O}
\setsansfont[Scale=MatchLowercase,Ligatures=TeX]{Linux Biolinum O}
%\setmonofont[Scale=MatchLowercase]{Inconsolatazi4}
\usepackage{microtype}

\usepackage{unicode-math}
\setmathfont[Scale=MatchLowercase]{Asana Math}
%\setmathfont[Scale=MatchLowercase]{XITS Math}

% To define fonts for particular uses within a document. For example, 
% This sets the Libertine font to use tabular number format for tables.
\newfontfamily{\tablenumbers}[Numbers={Monospaced}]{Linux Libertine O}
\newfontfamily{\libertinedisplay}{Linux Libertine Display O}

\usepackage{booktabs}
\usepackage{multicol}

%\usepackage{tabularx}
\usepackage{longtable}
\usepackage[justification=raggedright, labelsep=period]{caption}
\captionsetup{singlelinecheck=off}
\captionsetup{skip=0.2em}

%\usepackage{siunitx}
\usepackage{array}
\newcolumntype{L}[1]{>{\raggedright\let\newline\\\arraybackslash\hspace{0pt}}p{#1}}
\newcolumntype{C}[1]{>{\centering\let\newline\\\arraybackslash\hspace{0pt}}p{#1}}
\newcolumntype{R}[1]{>{\raggedleft\let\newline\\\arraybackslash\hspace{0pt}}p{#1}}

\usepackage{enumitem}
\setlist{leftmargin=*}
\setlist[1]{labelindent=\parindent}
\setlist[enumerate]{label=\textsc{\alph*}.}

%\usepackage{hyperref}
%\usepackage{placeins} %PRovides \FloatBarrier to flush all floats before a certain point.
%\usepackage{hanging}

\usepackage[sc]{titlesec}


\renewcommand{\solutiontitle}{\noindent}
\unframedsolutions
\SolutionEmphasis{\bfseries}

\renewcommand{\questionshook}{%
	\setlength{\leftmargin}{-\leftskip}%
}
%Change \half command from 1/2 to .5
%\renewcommand*\half{.5}


\makeatletter
\def\SetTotalwidth{\advance\linewidth by \@totalleftmargin
\@totalleftmargin=0pt}
\makeatother


\pagestyle{headandfoot}
\firstpageheader{BI 063: Evolution and Ecology}{}{\ifprintanswers\textbf{KEY}\else Name: \enspace \makebox[2.5in]{\hrulefill}\fi}
\runningheader{}{}{\footnotesize{pg. \thepage}}
\footer{}{}{}
\runningheadrule

\newcommand*\AnswerBox[2]{%
    \parbox[t][#1]{0.92\textwidth}{%
    \begin{solution}#2\end{solution}}
    \vspace*{\stretch{1}}
}

\newenvironment{AnswerPage}[1]
    {\begin{minipage}[t][#1]{0.92\textwidth}%
    \begin{solution}}
    {\end{solution}\end{minipage}
    \vspace*{\stretch{1}}}

\newlength{\basespace}
\setlength{\basespace}{5\baselineskip}

%\printanswers

\begin{document}

\subsection*{The scientific method: mutations}

It has long been known that changes in the genes of organisms can occur.
Such changes are commonly called “mutations.” In the 1940s, it was not
yet known how mutations occur. Part of the answer to this question was
discovered by Salvador Luria and Max
Delbrück. They grew bacteria grown in covered dishes containing nourishment in which bacteria 
generally thrive. These are called bacterial cultures.

Research that took place before Luria and Delbrück's experiment 
showed that some types of viruses (bacteriophages, or “phages”) 
could attack and kill some types of bacteria. Luria and Delbrück discovered that in some bacterial cultures a few of
the bacteria survive attacks by phages. Moreover, descendants of the
surviving bacteria tend also to survive phage attacks. This shows that
the genes of some of the bacteria had undergone mutations that made them
resistant to the phage, and that these resistant bacteria passed their
mutant genes onto their offspring. The question remained as to whether the mutations that made the bacteria
resistant were caused by the attacking virus. Their experiment was designed to
answer this question.

Luria and Delbrück considered what would happen if a number 
of bacterial cultures, each with a similar small number of bacteria,
were allowed to grow for 24 hours, then all were infected with the
same quantity of the phage, and then were allowed to grow another 24 hours. If
the phages were producing the mutations, they argued, then all the
bacterial cultures should end up with roughly the same number of
resistant bacteria (Figure~1, top panel). Any other result would suggest that the viruses are
not causing the mutations (Figure~1, bottom panel). In the upper panel, each plate
has about the same number of resistant colonies. In the lower panel, plates have widely different
numbers of resistant colonies.

\medskip

{\centering\noindent\includegraphics[width=0.8\textwidth]{01_delbruck_luria}\par
}

\noindent{\footnotesize Figure 1. Graphic representation of possible outcomes from the 
Luria and Delbrück experiment. Each cluster represents a single bacterial plate 
descending from a single bacterium (at the top). Bacterial colonies were
exposed to the phage after 24 hours, and then grown for another 24 hours (bottom). 
Figure modified from image by Madeleine Price Ball, 
Wikimedia Commons. }

\bigskip

Here is an outline of Luria and Delbrück's experiment.

\begin{enumerate}

\item Make 20 bacterial cultures by adding bacteria to covered dishes with nutrients. 

\item Grow the cultures for 24 hours.

\item Infect all 20 bacterial cultures with the phage.

\item Grow the cultures for another 24 hours to allow the phage to infect the bacteria.

\item Record the number of resistant colonies at the end of the experiment. Their results are given in Table~1.

\end{enumerate}

%Table~1 shows the number of resistant colonies from 20 bacterial cultures at the end of the experiment.
%, and later found that the actual number of
%resistant bacteria differed widely from one bacterial culture to the
%next.

\medskip

\begin{longtable}[l]{@{}C{1.15in}C{1.15in}C{0.4in}C{1.15in}C{1.15in}@{}}
\caption[]{Number of bacterial colonies resistant to the phage. The numbers were recorded at the end of the experiment after infection by the phage for 24 hours. The last two columns (cultures 11–20) continue from the first two columns (cultures 1–10).}\tabularnewline
\toprule
Culture Number	&	Resistant Colonies &		&	Culture Number	&	Resistant Colonies \tabularnewline
\midrule
1	&	1	&	& 	11	&	107 \tabularnewline
2	&	0	&	&	12	&	0	\tabularnewline
3	&	3	&	&	13	&	0	\tabularnewline
4	&	0	&	&	14	&	0	\tabularnewline
5	&	0	&	&	15	&	1	\tabularnewline
6	&	5	&	&	16	&	0	\tabularnewline
7	&	0	&	&	17	&	0	\tabularnewline
8	&	5	&	&	18	&	64	\tabularnewline
9	&	0	&	&	19	&	0	\tabularnewline
10	&	6	&	&	20	&	35	\tabularnewline
\bottomrule
\end{longtable}

\bigskip

Apply the scientific method to the above report. Determine what the various
components are and list them on the next page. These will be graded for
correctness and the clarity of your explanations. \textbf{Write answers below in
 your own words. Do not just copy sentences from above.}

\begin{questions}

\question
What is the real world observation?

\AnswerBox{3\baselineskip}{%
Some bacteria developed resistance to viruses. The resistance
was inherited by their offspring.}

\question
What is the hypothesis?

\AnswerBox{2\baselineskip}{%
The viruses that infected the bacteria caused the mutations.}

\newpage

\question
What is the prediction made by the hypothesis?

\AnswerBox{4\baselineskip}{%
If the viruses bacteria caused the mutations, then each 
	culture should have about the same number of resistant colonies.}


\question
What are the results of of their experiment, based on Table~1?

\AnswerBox{4\baselineskip}{%
Table 1 shows that the number of resistant bacterial colonies varied widely among cultures (0--107).}

\question
Do the results and predictions agree? Explain why or why not.

\AnswerBox{4\baselineskip}{%
The prediction was that the number of resistant colonies should
be about they same but the number varied. The results do not agree 
with this prediction. }

\question
What can you conclude about the hypothesis?

\AnswerBox{4\baselineskip}{%
The hypothesis (viruses as the cause) was falsified.
}

\question[Checkout]
A competing hypothesis proposed that mutations occur at random. If so, then mutations could have occurred in the bacterial cultures at any time during Luria and Delbrück's experiment. Are the results in Table~1 consistent with the competing hypothesis? Explain. Use numbers from Table~1 to support your explanation.

\AnswerBox{5\baselineskip}{%
The results are consistent. Mutations that occur early in the experiment should leave more resistant colonies (e.g., 107 colonies). Mutations that occur late in the experiment should leave fewer resistant colonies (e.g., 5 colonies).}

\end{questions}

\end{document}  