

\subsubsection{C. Third capture---Detecting change in population
size}\label{c.-third-capturedetecting-change-in-population-size}

Ask the instructor to adjust your population (s/he) will add or remove
some individuals from your population. Mix spiders \emph{really} well
after the instructor has adjusted the numbers.

Capture your spiders as before, recording the total number of spiders in
the third capture (\textbf{n\textsubscript{3}}) and the number of
spiders marked in the second capture and then recaptured in the third
sample (\textbf{M\textsubscript{23}}; all spiders with marks used in the
second capture). Mark the spiders captured this time with a \emph{third
color on the same leg}. Yes, so many colors makes for beautiful spiders.

2) Calculate the population size (\textbf{N\textsubscript{2}}) of
spiders \emph{after the third capture} using the following equation:

N\textsubscript{2} =
(M\textsubscript{2}n\textsubscript{3})/M\textsubscript{23}

Was this population size (N\textsubscript{2}) similar to your original
estimate (N\textsubscript{1})? If not, how does your estimate differ?


3) Calculate your estimated rate of population change (ΔP) to determine
how many spiders were added or subtracted from your population by the
instructor. Use the following equation:

The value you calculated will be a positive or negative number and will
be in decimal form. To determine how many spiders were added/removed
(added will be +, removed will be -)

\# spiders = Original \# of spiders in the box *

Once you have calculated Δ\emph{P}, provide your answer to your
instructor to determine if you were able to accurately detect the change
in your population.

How many spiders were added/removed? \_\_\_\_\_

Was your estimate regarding the change in population correct? Explain.

\textbf{E. \emph{Examining the effect of sampling time on population
estimates}}

In the first part of this lab, the spider sampler had only 45 seconds to
capture (sample) spiders from the population. To examine the effect of
sampling time on your estimate of population size, repeat parts A and B
(just through calculating n\textsubscript{2} and M\textsubscript{12}).
\emph{However, this time allow the spider sampler 60 seconds to capture
spiders}. Mark the spiders on a DIFFERENT LEG so you do not get mixed up
with marks from previous captures. Record your new M\textsubscript{1},
n\textsubscript{2}, and M\textsubscript{12} as you go.

4) Calculate the population size (\textbf{N\textsubscript{1}}) of
spiders after the second capture using the following equation:

\begin{quote}
Was your estimate of the population size closer to the actual population
size in this sampling effort, given more time for sampling?
\end{quote}

What does this tell you about the effect of sampling time on estimating
population size using mark-recapture? If your estimate this time was
similar to your original estimate, describe what you imagine the effect
of having more time to sample would be on your estimate of population
size in a field setting, in which organisms are more likely to be spread
out across a larger space.

5) If migration occurred in a natural population being studied, how
would this influence the reliability of your estimate of population size
determined using the mark-and-recapture technique?

6) You decide to test out a new type of marking technique on a lizard
population you are interested in characterizing. While other researchers
have used nail polish to mark individuals, you decided to use acrylic
paint (ex. Crayola washable kids paint) as that is what you have on
hand. You are able to mark 20 individuals. Given the area you sampled,
you are confident you have captured a large proportion of the population
(as in competition will limit the \# of individuals in any one location
because they set up territories of a known size). While you intended to
go out to recapture after a few days, the weather has prevented you from
doing so from massive rains and flooding. So, it ultimately takes a week
after releasing your paint-marked individuals for you to return to the
same location to see how many you can recapture. This time you catch 25
lizards, but none of them are marked.

What are the two possible scenarios that could have led to this result?

Based on your study design, which scenario is most likely to explain the
absence of marked individuals in the study area upon your return? (Which
assumption of the model was likely violated?) Explain your reasoning.
