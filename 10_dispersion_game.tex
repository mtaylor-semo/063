%!TEX TS-program = lualatex
%!TEX encoding = UTF-8 Unicode

\documentclass[12pt, hidelinks]{exam}
\usepackage{graphicx}
	\graphicspath{{/Users/goby/Pictures/teach/163/lab/}
	{img/}} % set of paths to search for images

\usepackage{geometry}
\geometry{letterpaper, left=1.5in, bottom=1in}                   
%\geometry{landscape}                % Activate for for rotated page geometry
\usepackage[parfill]{parskip}    % Activate to begin paragraphs with an empty line rather than an indent
\usepackage{amssymb, amsmath}
\usepackage{mathtools}
	\everymath{\displaystyle}

\usepackage{fontspec}
\setmainfont[Ligatures={TeX}, BoldFont={* Bold}, ItalicFont={* Italic}, BoldItalicFont={* BoldItalic}, Numbers={Proportional, OldStyle}]{Linux Libertine O}
\setsansfont[Scale=MatchLowercase,Ligatures=TeX, Numbers={Proportional,OldStyle}]{Linux Biolinum O}
\setmonofont[Scale=MatchLowercase]{Linux Libertine Mono O}
\newfontfamily{\liningnum}[Numbers=Lining]{Linux Libertine O}
\usepackage{microtype}

\usepackage{unicode-math}
\setmathfont[Scale=MatchLowercase]{Tex Gyre Pagella Math}


% To define fonts for particular uses within a document. For example, 
% This sets the Libertine font to use tabular number format for tables.
 %\newfontfamily{\tablenumbers}[Numbers={Monospaced}]{Linux Libertine O}
% \newfontfamily{\libertinedisplay}{Linux Libertine Display O}

\usepackage{booktabs}
\usepackage{multicol}

\usepackage{caption}
\captionsetup{font=small} 
\captionsetup{singlelinecheck=false}
\captionsetup[figure]{labelsep=period, format=plain}

\usepackage{longtable}
%\usepackage{siunitx}
\usepackage{array}
\newcolumntype{L}[1]{>{\raggedright\let\newline\\\arraybackslash\hspace{0pt}}p{#1}}
\newcolumntype{C}[1]{>{\centering\let\newline\\\arraybackslash\hspace{0pt}}p{#1}}
\newcolumntype{R}[1]{>{\raggedleft\let\newline\\\arraybackslash\hspace{0pt}}p{#1}}

\usepackage{enumitem}
\setlist{leftmargin=*}
\setlist[1]{labelindent=\parindent}
\setlist[enumerate]{label=\textsc{\alph*}.}
\setlist[itemize]{label=\color{gray}\textbullet}

\usepackage{hyperref}
%\usepackage{placeins} %PRovides \FloatBarrier to flush all floats before a certain point.
\usepackage{hanging}

\usepackage[sc]{titlesec}

%% Commands for Exam class
\renewcommand{\solutiontitle}{\noindent}
\unframedsolutions
\SolutionEmphasis{\bfseries}

\renewcommand{\questionshook}{%
	\setlength{\leftmargin}{-\leftskip}%
}

%Change \half command from 1/2 to .5
\renewcommand*\half{.5}

\pagestyle{headandfoot}
\firstpageheader{\textsc{bi}\,063 Evolution and Ecology}{}{\ifprintanswers\textbf{KEY}\else Name: \enspace \makebox[2.5in]{\hrulefill}\fi}
\runningheader{}{}{\footnotesize{pg. \thepage}}
\footer{}{}{}
\runningheadrule

\newcommand*\AnswerBox[2]{%
    \parbox[t][#1]{0.92\textwidth}{%
    \begin{solution}#2\end{solution}}
    \vspace{\stretch{1}}
}

\newenvironment{AnswerPage}[1]
    {\begin{minipage}[t][#1]{0.92\textwidth}%
    \begin{solution}}
    {\end{solution}\end{minipage}
    \vspace{\stretch{1}}}

\newlength{\basespace}
\setlength{\basespace}{5\baselineskip}


\newcommand\chisq{$\chi^2$}
\newcommand*\meanY{\overline{Y}\kern0.67pt}
\newcommand*\AnswerBlank{\rule{0.75in}{0.4pt}\kern0.67pt.}
\newcommand*\xcell[1]{cell~\liningnum{#1}}

%
%\makeatletter
%\def\SetTotalwidth{\advance\linewidth by \@totalleftmargin
%\@totalleftmargin=0pt}
%\makeatother


%\printanswers


\begin{document}

\subsection*{Testing for dispersion of individuals in populations\footnote{Reproduced and modified from Dr. S. Borrett, UNC Wilmington Ecology Laboratory Manual.}}

\subsubsection*{Dispersion}

Ecologists who study populations and communities of
organisms are often interested in the spatial distribution of
individuals because the patterns can provide information
about the biology of a species and the factors limiting that
organism. Individuals in a population may have one of three general
types of spatial dispersions: clumped, random, or uniform (Figure~\ref{fig:dispersion_patterns}).
For example, a clumped distribution may indicate that a species is
responding to fine gradations in the environment or that it has a form
of reproduction that keeps juveniles near adults. Conversely, a uniform
distribution may indicate territoriality or some other aggressive
interaction among individuals. For this exercise, you will use a
combination of quadrat sampling and data analysis to characterize the
dispersion for two populations of plants found in a nearby field.

\begin{figure}[h!]
	\begin{center}
	\includegraphics[width=\textwidth]{10_dispersion_patterns}
	\caption{Three ways in which individuals of a population can be
distributed in space.}\label{fig:dispersion_patterns}
	\end{center}
\end{figure}

Ecologists have developed several approaches to count sedentary (non-moving)
 organisms. The most common method is plot sampling. This
method is straightforward in that it involves counting the number of
organisms of interest in a defined area (often, but not necessarily, a
quadrat, which is a defined sampling area, such as a square meter). 
Plot sampling is a versatile technique that can provide
information on densities, associations, dispersion patterns, and
indirect evidence on a variety of community or population processes. We
will use a quadrat sampling technique in this laboratory.

The primary approach to determining dispersion patterns is to compare
\emph{observed} patterns with what would be \emph{expected} if the
dispersion were random (using a statistical test). The \emph{null
hypothesis} for this investigation is that the individuals of the
population are randomly dispersed (not clumped or evenly spaced). Why?
If there is a difference between the observed pattern and the expected
pattern, then you must further examine the data to determine whether
individuals are found together (clumped) or are spaced apart (uniform).

Dispersion can be measured by several methods. You will use
a chi-squared test (\kern1.1667pt\chisq{}) to compare the sum of squares
(a measure of variability) to the mean to determine dispersion patterns
for this exercise. According to the \chisq{} test, if a species is
uniformly distributed, variability should be low (similar numbers in all
quadrats). If species are clumped, variability should be high (some
quadrats with many individuals, others with few or none). Random
distributions would have intermediate variability because some individuals
will be close together and other individuals will be farther apart.

\subsubsection*{Analysis of dispersion patterns}

In this exercise, you will determine the dispersion pattern of two
species. The game board represents the field in which you are interested
in determining the dispersion pattern of the blue and the green species.
The field has been divided into quadrats; you will randomly sample quadrats
and record the record the number of blue and green dots in each quadrat.
The yellow/orange and brown species are not of interest to your study at
this time. As part of this exercise, you will also examine how sample
size can affect your ability to detect patterns in nature.

\begin{enumerate}

\item To determine the dispersion pattern, you need to calculate the sample
mean $\left( \meanY\right) $, the sum of squares $(SS)$, a \chisq{} statistic, and the
degrees of freedom (d.f.). You will use a \chisq{} statistic
because this is most appropriate for discrete count data like these.
 You will then characterize the dispersion pattern using the
\chisq{}, the degrees of freedom (d.f.), and Figure~\ref{fig:chi_df}. The formula 
to calculate the mean $\left( \meanY \right)$ is


\[\meanY = \dfrac{\sum\limits_{i=1}^n Y_i}{n}\]

where $Y_i$ is the number of plants in the $i\mathrm{th}$ quadrat, and $n$ is the total number of quadrats.


Use the mean to calculate the sum of squares ($SS$) which is a measure of the variability in the sample set.  

\[SS =  \sum\limits_{i=1}^n \left( Y_i - \meanY \right) ^2 \]


You calculated the sum of squares in the Descriptive Statistics lab as one of the steps to needed to calculate standard deviation. This time, you will use the sum of squares to calculate the test statistic, \chisq{}. 

\[ \chi^2 = \dfrac{SS}{\meanY} = \dfrac{\sum\limits_{i=1}^n \left( Y_i - \meanY \right) ^2}{\meanY}\]

Degrees of freedom is calculated as d.f. $= n-1.$  You can determine the dispersion pattern for the candidate species using Figure~\ref{fig:chi_df}.  

\end{enumerate}

\begin{figure}[h!]
	\begin{center}
	\captionsetup{width=0.5\textwidth}
	\includegraphics[width=0.5\textwidth]{10_chi_df}
	\caption{Relationship between degrees of freedom and the \chisq{}
statistic to determine the dispersion pattern of a
population.}\label{fig:chi_df}
	\end{center}
\end{figure}

\begin{enumerate}[resume]

\item Now, set up a file in Excel to record your data.

In \xcell{A1}, type “Blue”. In \xcell{D1}, type “Green”. You will record the number of sampled individuals (color dots) in these columns for sampling effort~\#1. At the bottom of the Excel worksheet, click the $+$ next to Sheet 1 to add an additional sheet. If you see Sheet 2, just click on it instead. In Sheet 2,
type “Blue” in \xcell{A1} and “Green” in \xcell{D1}. You will record your data
in this sheet for sampling effort~\#2.

\end{enumerate}

\subsubsection*{Sampling effort \#1: non-random sampling}

You arrive at your field site (the game board) and must decide
how to best collect your data. The field has previously been divided
into 130 quadrats. Out of convenience, based on the proximity to the
parking lot and the fact that it is \emph{really} hot and you do not want to
walk very far, you decide to sample \textbf{all quadrats in rows 6–9}
for a total of 40 quadrats sampled. 

\begin{enumerate}

\item Collect your data. Starting in quadrat {\liningnum A6} and sampling every quadrat
through {\liningnum J9}, record the number of blue and green dots in each quadrat
in the Excel file you started. Be sure to record “0” for each
color that is missing in that quadrat.


\item Calculate the mean $\left( \meanY \right)$ number of blue individuals. Click on \xcell{A42}, and then type “\texttt{=average(A2:A41)}.”

 An alternate method to do this is to type “\texttt{=average(}”, select cells {\liningnum A2 to A41}, type~“\texttt{)}”, and then press the Enter key. Use whichever method you prefer.

\item Calculate the mean  $\left( \meanY \right)$ number of green individuals. Click on \xcell{D42}, and then type
“\texttt{=average(D2:D41)}.”

\item Calculate the sum of squares for the blue and the green species. You will do this in
columns B (blue species) and E (green species). This is calculated as the number of dots in each quadrat $\left( Y_i \right)$ minus the mean $\left( \meanY \right)$,
squared. Click in \xcell{B2}. Type “\texttt{=(A2$-$A\$42)\textasciicircum2}” and press the Enter key. 
Be sure to include the \$ symbol in \texttt{A\$42}, which tells Excel to use “absolute addressing.” When you fill in the rows (next step) with this formula, the first cell reference (\texttt{A2}) will automatically change for each row (\texttt{A3,\,A4,\,\dots,\,A41}) but the second cell will be always \texttt{A42}, with your calculated mean. 

\item Highlight \xcell{B2}, place the cursor over the bottom right corner of
the cell over the small, black square. Your cursor should become a “$+$”
sign. Click and drag all the way down the column to \xcell{B41} to fill
in all quadrats.  To get the $SS$, you need to take the sum of these
values. In \xcell{B42}, type “\texttt{=SUM(B2:B41)}” and press the Enter key. This value is your Sum of Squares $(SS)$.

Repeat this for the green species in column E. Be sure to reference \xcell{D42} with the \$ sign for the mean number of green individuals.

\item Calculate your \chisq{} for the blue species by dividing your $SS$ by the mean. Click on \xcell{C42}. Type “\texttt{=B42/A42)}.” Repeat this for the green species in \xcell{F42}.

\bigskip

\chisq{} (blue): \AnswerBlank{} \qquad \chisq{} (green): \AnswerBlank{}

\item Determine your degrees of freedom (d.f. $= n-1$, where $n =$ \# of sampled
quadrats). \textsc{Note:} d.f. should be the same for both species.

\item Determine the dispersion pattern for each species using Figure~\ref{fig:chi_df} (on page~\pageref{fig:chi_df}).

\bigskip

Blue: \ifprintanswers\textbf{Random}\else\rule{1.5in}{0.4pt}\fi \qquad Green: \ifprintanswers\textbf{Clumped}\else\rule{1.5in}{0.4pt}\fi

\end{enumerate}

\newpage

\begin{questions}

\question
Looking at your entire study site (the game board), do you think 
the non-random sampling method you used allowed you to accurately 
determine the dispersion patterns of the
blue and green dots? In other words, did your sampling method allow you
to collect the data needed to answer your research question? Explain why
or why not.

\vspace{6\baselineskip}

\subsubsection*{Sampling effort \#2: random sampling}

This time, the quadrats you sample will be determined randomly (like real
scientists, rather
than by laziness relative to the parking lot) by drawing pieces of blue
paper with quadrat coordinates on them (ex. {\liningnum A5}). You will select one
piece of paper at a time, place it on the board in the appropriate
quadrat, and record the number of blue and green dots in that quadrat in
Sheet 2 in your Excel file. Do not return the blue paper to the pile.
Repeat this process until you have sampled \textbf{90 quadrats}.
\emph{Be sure to record “{\liningnum 0}” if one or neither color is present in that
	quadrat}



\begin{enumerate}
\item Calculate the mean number of blue individuals. Click on cell \xcell{A42}, and then type
“\texttt{=average(A2:A91)}.”


\item Calculate the mean number of green individuals. Click on \xcell{D92}, and then type
“\texttt{=average(D2:D91)}.”


\item Calculate the sum of squares for the blue and the green species. You will do this in columns B (blue species) and E (green species). This is calculated as the number of dots in each quadrat $\left( Y_i \right)$ minus the mean $\left( \meanY \right)$,
squared. Click in \xcell{B2}. Type “\texttt{=(A2$-$A\$92)\textasciicircum2}” and press the Enter key. 
Be sure to include the \$ symbol in \texttt{A\$92}, which tells Excel to use “absolute addressing.” When you fill in the rows (next step) with this formula, the first cell reference (\texttt{A2}) will automatically change for each row (\texttt{A3,\,A4,\,\dots,\,A41}) but the second cell will be always \texttt{A92}, with your calculated mean. 

\item Highlight \xcell{B2}, place the cursor over the bottom right corner of
the cell over the small, black square. Your cursor should become a “$+$”
sign. Click and drag all the way down the column to \xcell{B91} to fill
in all quadrats.  To get the $SS$, you need to take the sum of these
values. In \xcell{B92}, type “\texttt{=SUM(B2:B91)}” and press the Enter key. This value is your Sum of Squares $(SS)$.

Repeat this for the green species in column \texttt{E}. Be sure to reference \xcell{D92} with the \$ sign for the mean number of green individuals.

\item Calculate your \chisq{} for the blue species by dividing your $SS$ by the mean. Click on \xcell{C92}. Type “\texttt{=B92/A92)}.” Repeat this for the green species in \xcell{F92}.

\bigskip

\chisq{} (blue): \AnswerBlank{} \qquad \chisq{} (green): \AnswerBlank{}

\bigskip

\item Determine your degrees of freedom (df $= n-1,$ where $n =$ \# of sampled
quadrats). Note: the df should be the same for both species.

\item Determine the dispersion pattern for each species using Figure 1 (on
page~\pageref{fig:chi_df}).

\bigskip

Blue: \ifprintanswers\textbf{Random}\else\rule{1.5in}{0.4pt}\fi \qquad Green: \ifprintanswers\textbf{Clumped}\else\rule{1.5in}{0.4pt}\fi

%\bigskip

\end{enumerate}

\subsubsection*{Conclusion (check out)}

\question
Did you get the same dispersion pattern for sampling effort \#1 and
sampling effort \#2? If not, why do you think there was a difference?

\AnswerBox{3\baselineskip}{%
	Answer Here
}

\question
What was the difference in your sampling method in effort \#1 compared to effort \#2?

\AnswerBox{3\baselineskip}{%
	Answer Here
}

\question
Explain the importance of non-random sampling and collecting a large
sample size when conducting ecological field work.

\AnswerBox{3\baselineskip}{%
	Answer Here
}

\end{questions}

\end{document}  