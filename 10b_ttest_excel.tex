%!TEX TS-program = lualatex
%!TEX encoding = UTF-8 Unicode

\documentclass[12pt]{exam}
\usepackage{graphicx}
	\graphicspath{{/Users/goby/Pictures/teach/163/lab/}
	{img/}} % set of paths to search for images

\usepackage{geometry}
\geometry{letterpaper, left=1.5in, bottom=1in}                   
%\geometry{landscape}                % Activate for for rotated page geometry
\usepackage[parfill]{parskip}    % Activate to begin paragraphs with an empty line rather than an indent
\usepackage{amssymb, amsmath}
\usepackage{mathtools}
	\everymath{\displaystyle}

\usepackage[table]{xcolor}

\usepackage{fontspec}
\setmainfont[Ligatures={TeX}, BoldFont={* Bold}, ItalicFont={* Italic}, BoldItalicFont={* BoldItalic}, Numbers={OldStyle}]{Linux Libertine O}
\setsansfont[Scale=MatchLowercase,Ligatures=TeX]{Linux Biolinum O}
\setmonofont[Scale=MatchLowercase]{Inconsolatazi4}
\newfontfamily{\liningnum}[Numbers={Lining}]{Linux Libertine O}
\usepackage{microtype}

\usepackage{unicode-math}
\setmathfont[Scale=MatchLowercase]{TeX Gyre Termes Math}

\usepackage{amsbsy}
%\usepackage{bm}

% To define fonts for particular uses within a document. For example, 
% This sets the Libertine font to use tabular number format for tables.
 %\newfontfamily{\tablenumbers}[Numbers={Monospaced}]{Linux Libertine O}
% \newfontfamily{\libertinedisplay}{Linux Libertine Display O}

\usepackage{multicol}
%\usepackage[normalem]{ulem}

\usepackage{longtable}
\usepackage{caption}
	\captionsetup{format=plain, justification=raggedright, singlelinecheck=off,labelsep=period,skip=3pt} % Removes colon following figure / table number.
%\usepackage{siunitx}
\usepackage{booktabs}
\usepackage{array}
\newcolumntype{L}[1]{>{\raggedright\let\newline\\\arraybackslash\hspace{0pt}}m{#1}}
\newcolumntype{C}[1]{>{\centering\let\newline\\\arraybackslash\hspace{0pt}}m{#1}}
\newcolumntype{R}[1]{>{\raggedleft\let\newline\\\arraybackslash\hspace{0pt}}m{#1}}

\usepackage{enumitem}
\setlist{leftmargin=*}
\setlist[1]{labelindent=\parindent}
\setlist[enumerate]{label=\textsc{\alph*}.}
\setlist[itemize]{label=\color{gray}\textbullet}
%\usepackage{hyperref}
%\usepackage{placeins} %PRovides \FloatBarrier to flush all floats before a certain point.
%\usepackage{hanging}

\usepackage[sc]{titlesec}

%% Commands for Exam class
\renewcommand{\solutiontitle}{\noindent}
\unframedsolutions
\SolutionEmphasis{\bfseries}

\renewcommand{\questionshook}{%
	\setlength{\leftmargin}{-\leftskip}%
}

%Change \half command from 1/2 to .5
\renewcommand*\half{.5}

\pagestyle{headandfoot}
\firstpageheader{\textsc{bi}\,063 Evolution and Ecology}{}{\ifprintanswers\textbf{KEY}\else Name: \enspace \makebox[2.5in]{\hrulefill}\fi}
\runningheader{}{}{\footnotesize{pg. \thepage}}
\footer{}{}{}
\runningheadrule

\newcommand*\AnswerBox[2]{%
    \parbox[t][#1]{0.92\textwidth}{%
    \begin{solution}#2\end{solution}}
%    \vspace*{\stretch{1}}
}

\newenvironment{AnswerPage}[1]
    {\begin{minipage}[t][#1]{0.92\textwidth}%
    \begin{solution}}
    {\end{solution}\end{minipage}
    \vspace*{\stretch{1}}}

\newlength{\basespace}
\setlength{\basespace}{5\baselineskip}

%% To hide and show points
\newcommand{\hidepoints}{%
	\pointsinmargin\pointformat{}
}

\newcommand{\showpoints}{%
	\nopointsinmargin\pointformat{(\thepoints)}
}

\newcommand{\bumppoints}[1]{%
	\addtocounter{numpoints}{#1}
}

\newcommand*\meanY{\overline{Y\kern1.67pt}\kern-1.67pt}
\newcommand*\meansubY{\overline{Y}}
%\newcommand*\meanY{\overline{Y}}
\newcommand*\ttest{\emph{t}-test}
\newcommand*\Popa{Population~\textsc{a}}
\newcommand*\Popb{Population~\textsc{b}}
\newcommand*\popa{population~\textsc{a}} %lower case
\newcommand*\popb{population~\textsc{b}} %lower case
\newcommand*\Corbicula{\textit{Corbicula}}
\newcommand*\AnswerBlank{\rule{0.75in}{0.4pt}\kern0.67pt.}

\newcommand*\xcell[1]{cell~\liningnum{#1}}
%
%\makeatletter
%\def\SetTotalwidth{\advance\linewidth by \@totalleftmargin
%\@totalleftmargin=0pt}
%\makeatother


%\printanswers


\begin{document}

\subsection*{The $t$-test in Excel}

The $t$-test is relatively simple to calculate by hand but would quickly grow tedious if you had a large dataset to analyze. Fortunately, you can analyze a dataset with a $t$-test easily in Excel. Excel provides several functions for the $t$-test. You will use two of those functions, and a few other functions to calculate degrees of freedom and to cross check your earlier results.

\emph{Perform these steps exactly as described. Failure to follow these instructions could result in failure to earn your points for this lab.} 

\begin{enumerate}
	\item Type "\Popa" into \xcell{B1}. Type "\Popb" into \xcell{C1}.

	\item Enter your measurements for shell width \emph{from the earlier part of this exercise} for \Popa{} into cells {\liningnum B2 through B16}. Enter one measurement per cell.
	
	\item Enter your measurements for shell width for \Popb{} into cells {\liningnum C2 through C16}.
	
	\item Enter "Deg. Freedom" in \xcell{A18}.

	\item Enter "Variance" in \xcell{A19}.
	
	\item Enter "Std. Dev." in \xcell{A20}.

	\item Enter "$p$-value" in \xcell{A21}.
	
	\item Enter "$t$-value" in \xcell{A22}.
	 
\end{enumerate}

\subsubsection*{Degrees of freedom}

One of the Excel functions you need to perform the $t$-test requires the degrees of freedom. Recall from the previous exercise that degrees of freedom is $n-1$ \emph{for each population.} The \texttt{COUNT()} function counts the number of cells selected. You are using the \texttt{COUNT()} function to count the number of individuals sampled. You must subtract 1 degree of freedom for each population.


\begin{enumerate}[resume]

	\item \label{df_calc} Click on \xcell{B18}. Enter \texttt{=COUNT(B2:B16)-1} and press Enter. Click on \xcell{C18}. Enter \texttt{=COUNT(C2:C16)-1} and press Enter. If the result is not 14 for each entry, check that you entered the formula correctly, or ask your instructor for help.
	
\end{enumerate}

\subsection*{Variance and standard deviation}

In the earlier part of this exercise, you calculated the variance and the standard deviation. The $t$-test uses the variance and the \texttt{T.TEST()} function in Excel includes the variance automatically. Here, you will use the \texttt{VAR()} and \texttt{STDEV()} functions in Excel as a check of your math. The values you get from these functions should match the values you calculated earlier.  If they do not, then you should double check your math before proceeding.

\begin{enumerate}[resume]

	\item Click on \xcell{B19}. Type \texttt{=VAR(B2:B16)} and press Enter. Click on \xcell{C19}. Type \texttt{=VAR(C2:C16)} and press Enter. 

	\item Click on \xcell{B20}. Type \texttt{=STDEV.S(B2:B16)} and press Enter. Click on \xcell{C20}. Type \texttt{=STDEV.S(C2:C16)} and press Enter. 
	
		If you get a \texttt{\#NAME} error, use \texttt{STDEV()} instead, without the \texttt{.S}. You will get the same result. The \texttt{\#NAME} error may occur if you are using an older version of Excel. 
		
		\textbf{Important:} Check that the standard deviation $\left(s\right)$ here is the same value you calculated be hand. If they disagree, then you will have incorrectly calculated the $t$ value by hand. Double check your math and be sure you have the same result before proceding.
	
\end{enumerate}

\subsubsection*{The $p$-value}

Excel provides two functions to perform a $t$-test. The first function you will use is \texttt{T.TEST(),} which returns the \emph{p}-value (probability) that you would have obtained your data assuming the null hypothesis is true. That is, the function returns the probability that the two samples are from the same population (have the same mean).  The format for the Excel function is \texttt{T.TEST(array1, array2, tails, type)}. \texttt{Array1} is your first sample (\Popa, cells {\liningnum B2:B16} in this case) and \texttt{array2} is your second sample (\Popb, cells {\liningnum C2:C16}). \texttt{Tails} can be either 1 or 2, which refers to whether you are doing a one-sided or two-sided test. You are doing a two-sided test because your research hypothesis is that the two populations have different shell widths. One population could be larger \emph{or} smaller than the other, so $t$ could be positive or negative (the two tails). If you made a specific prediction about shell width, such as \popb{} had a smaller shell width than \popa, then you would use a one-tailed test. Here, use a two-tailed test.  \texttt{Type} can be 1, 2, or 3. We performed a type~2 test, which specifies the two samples have equal variance. The other types are type~1 (paired samples) and type~3 (the two samples have unequal variance).

\begin{enumerate}[resume]
	\item To calculate the probability using the \texttt{T.TEST()} function, click in \xcell{B21} and enter\\ \texttt{=T.TEST(B2:B16,C2:C16,2,2)}. 
	
	If you get a \texttt{\#NAME} error, use \texttt{TTEST()} instead, without the period. You will get the same result. The \texttt{\#NAME} error may occur if you are using an older version of Excel. 

\end{enumerate}

\subsubsection*{The $t$-value}

The second function you need is \texttt{T.INV.2T(),} which gives you the actual $t$ value. This function needs the degrees of freedom and the $p$-value returned by \texttt{T.TEST()}, which is why you did those steps first. The format for the second function is \texttt{T.INV.2T(probability, deg\_freedom)}. \texttt{Probability} is the exact $p$-value returned by the \texttt{T.TEST()} function. Degrees of freedom is the value you calculated in step~\textsc{\ref{df_calc}} This returns the $t$ value for a 2-tailed test (which is the \texttt{.2T} part of the function name).

\begin{enumerate}[resume]
	\item Click on \xcell{B21} and enter \texttt{=T.INV.2T(B20,(B18+C18-2))}. You could enter 28 for degrees of freedom but do not. What if you added more data? You would have to manually edit all of your formulas. By properly using cells references, your calculations are automatically adjusted. 
	
	If you get a \texttt{\#NAME} error, use \texttt{TINV()} instead, without the period or \texttt{2T}. You will get the same result. The \texttt{\#NAME} error may occur if you are using an older version of Excel. 
\end{enumerate}
	
Because Excel gives you the exact probability, you do not have to compare the calculated $t$ value to the a table. To properly report the results of your $t$-test, you must use both formulas. 

\begin{questions}

\question
Check to see that you got the same results in Excel that you got in the previous exercise. If the results disagree, carefully check your analyses in both exercises to find and correct the error. Ask your instructor for help, if necessary.

\question[Checkout]
Show the results of your Excel $t$-test and your hand-calculated $t$-tests to your instructor. Failure to show your results to your instructor will result in no points for the lab. Do not waste your hard work!

\end{questions}


\end{document}  