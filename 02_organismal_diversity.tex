%!TEX TS-program = lualatex
%!TEX encoding = UTF-8 Unicode

\documentclass[11pt, addpoints]{exam}
\usepackage{graphicx}
	\graphicspath{{/Users/goby/Pictures/teach/151/}
	{img/}} % set of paths to search for images

\usepackage{geometry}
\geometry{letterpaper, bottom=1in}                   
%\geometry{landscape}                % Activate for for rotated page geometry
%\usepackage[parfill]{parskip}    % Activate to begin paragraphs with an empty line rather than an indent
\usepackage{amssymb, amsmath}
\usepackage{mathtools}
	\everymath{\displaystyle}

\usepackage{fontspec}
\setmainfont[Ligatures={TeX}, BoldFont={* Bold}, ItalicFont={* Italic}, BoldItalicFont={* BoldItalic}, Numbers={Proportional}]{Linux Libertine O}
\setsansfont[Scale=MatchLowercase,Ligatures=TeX]{Linux Biolinum O}
%\setmonofont[Scale=MatchLowercase]{Inconsolatazi4}
\usepackage{microtype}

\usepackage{unicode-math}
\setmathfont[Scale=MatchLowercase]{Asana Math}
%\setmathfont[Scale=MatchLowercase]{XITS Math}

% To define fonts for particular uses within a document. For example, 
% This sets the Libertine font to use tabular number format for tables.
\newfontfamily{\tablenumbers}[Numbers={Monospaced}]{Linux Libertine O}
\newfontfamily{\libertinedisplay}{Linux Libertine Display O}

\usepackage{booktabs}
\usepackage{multicol}
\usepackage[normalem]{ulem}

%\usepackage{tabularx}
\usepackage{longtable}
%\usepackage{siunitx}
\usepackage{array}
\newcolumntype{L}[1]{>{\raggedright\let\newline\\\arraybackslash\hspace{0pt}}p{#1}}
\newcolumntype{C}[1]{>{\centering\let\newline\\\arraybackslash\hspace{0pt}}p{#1}}
\newcolumntype{R}[1]{>{\raggedleft\let\newline\\\arraybackslash\hspace{0pt}}p{#1}}

\usepackage{enumitem}
\usepackage{hyperref}
%\usepackage{placeins} %PRovides \FloatBarrier to flush all floats before a certain point.
\usepackage{hanging}

\renewcommand{\solutiontitle}{\noindent}
\unframedsolutions
\SolutionEmphasis{\bfseries}

\pagestyle{headandfoot}
\firstpageheader{BI 151: Biological Reasoning}{}{\ifprintanswers\textbf{KEY}\else Name: \enspace \makebox[2.5in]{\hrulefill}\fi}
\runningheader{}{}{\footnotesize{pg. \thepage}}
\footer{}{}{}
\runningheadrule

\printanswers

\begin{document}

\subsection*{Organismal Diversity (\numpoints\ points)}

Organisms were once (prior to 1866) classified into two kingdoms -
plants and animals. The organization of some biology curricula still
reflects this — for instance, the Missouri State Legislature's
required courses for teacher education in biology specify botany and
zoology, but not microbiology or study of fungi, etc. Until fairly
recently, the generally accepted classification system put all organisms
into five kingdoms. Over the past several years, though, taxonomists
have become increasingly aware that some organisms classified as
bacteria (most of them living in extreme environments like hot springs
and salt flats) were chemically so different from other bacteria —
and all other organisms — that they should have their own distinct
group. More recently various other kingdoms have been proposed. Higher
level taxonomy is always somewhat controversial; while organisms do
group in a set of nested categories, it's hard to decide how big a group
you're going to call a genus, or family, or order, or kingdom. And new
information is found all the time that necessitates modifications in
taxonomy. Anyway, here's is one particular classification scheme:

We divide living things into three domains (also known as
“superkingdoms"):

\begin{itemize}
\item
  Eukaryota (organisms with internal membranes in their cells, including
  a nuclear envelope),
\item
  Eubacteria (organisms with no internal membranes in their cells, thus
  no nucleus), and
\item
  Archaea (bacteria-like organisms with a wildly different internal
  organization and chemistry).
\end{itemize}

\noindent The domains Eubacteria and Archaea each contain only a single kingdom. 
The domain of Eukaryota is further divided into a number of eukaryotic
kingdoms, including:

\begin{itemize}
\item
  Plantae or Viridiplantae (plants — multicellular organisms with
  cell walls and chlorophyll)
\item
  Metazoa or Animalia (animals — multicellular organisms without cell
  walls)
\item
  Fungi (multicellular organisms with cell walls that do not
  photosynthesize)
\item
  Protista (sometimes called protoctista) — all eukaryotes that don't
  fit in the other kingdoms. These include a number of single-celled
  eukaryotes, including \emph{Paramecium} and \emph{Amoeba}. Most
  scientists no longer recognize Protista as a kingdom but, because the
  classification status for the organisms in this group is in a state of
  flux, it will serve for the purpose of this course.
\item
  And several more kingdoms that you don't need to worry about.
\end{itemize}

Each kingdom is progressively broken up into smaller and smaller
subgroups. From a domain, the most generalized level of classification,
down to a species, the most specific level of classification, there are
8 different levels of classification.

\begin{multicols}{2}
\noindent Domain

\noindent Kingdom

\noindent Phylum\footnote{Referred to as a Division in plants, fungi, and some protists.}

\noindent Class

\noindent Order

\noindent Family

\noindent Genus

\noindent Species
\end{multicols}

Thus a kingdom contains a number of phyla (plural of phylum), a phylum
contains a number of classes, a class contains many families, etc.
Additionally, these levels can be further divided into sub- or super-
groups, as in a subclass, superfamily, etc. The name of each group is
capitalized, such as

\begin{itemize}
	\item Canidae (family of wolves, foxes, dogs, etc). Do not write canidae.

	\item Arthropoda (phylum containing crabs, insects, spiders, etc). Do not
write arthropoda.
\end{itemize}

Each organism then has seven names, that is, the organism can be
described by listing the names of all seven levels of classification of
that particular organism. However, organisms are referred to by just the
Generic and Specific names, called the Species name.

A species name is always written as the Genus + Species name. The genus
name is always capitalized, while the specific name is not. Further, the
name is italicized \emph{or} underlined but not both. For example: The
common garden tomato is written as \vspace{\baselineskip}

\emph{Lycopersicon esculentum} or \uline{Lycopersicon esculentum}.\vspace{\baselineskip}

Note too that the genus name alone is also italicized or underlined, like \vspace{\baselineskip}

\emph{Elacatinus} or \uline{Elacatinus}.\vspace{\baselineskip}

Here are a few examples of the names of all the levels of classification
for a few select organisms from the domain Eukarya:

\begin{longtable}[c]{@{}lllll@{}}
\toprule
\textbf{Kingdom} & Metazoa & Metazoa & Metazoa & Plantae\tabularnewline
\textbf{Phylum} & Chordata & Chordata & Arthropoda &
Anthophyta\tabularnewline
\textbf{Class} & Mammalia & Mammalia & Insecta &
Dicotyledones\tabularnewline
\textbf{Order} & Carnivora & Carnivora & Hymenoptera &
Fagales\tabularnewline
\textbf{Family} & Canidae & Canidae & Sphecidae &
Fagaceae\tabularnewline
\textbf{Genus} & \emph{Canis} & \emph{Canis} & \emph{Sceliphron} &
\emph{Quercus}\tabularnewline
\textbf{Species} & \emph{Canis lupus} & \emph{Canis familiaris} &
\emph{Sceliphron caementarium} & \emph{Quercus alba}\tabularnewline
 & & & &\tabularnewline
\textbf{common name} & Wolf & Dog & Mud dauber wasp & White
Oak\tabularnewline
\bottomrule
\end{longtable}


\subsection*{Assignment}


\begin{questions}

\question[4]
For each of the following organisms (some
identified fairly specifically, some naming broader groups), use the
web, your brain, and/or the library to \textbf{find out its top-level
classification}. If it's a eukaryote, tell the kingdom that it's in; if
it's in one of the other two domains, just tell the domain. Refer to the
examples above, and use the links. Hint: not all kingdoms (or domains)
are represented. (0.5 points per blank).

\fullwidth{%
\noindent\begin{longtable}[l]{@{}L{0.65in}L{1.75in}L{0.85in}L{1.75in}@{}}
\toprule
Organism & Kingdom (Domain for \textit{E. coli}) & Organism & Domain\tabularnewline
\midrule
& & & \\
alligator 	& \ifprintanswers{Animalia/Metazoa}\else\rule{1.7in}{0.4pt}\fi &
 cat 		& \ifprintanswers{Animalia/Metazoa}\else\rule{1.7in}{0.4pt}\fi \tabularnewline[1.5ex]
%
\textit{E. coli} 	& \ifprintanswers{(Eu)Bacteria}\else\rule{1.7in}{0.4pt}\fi & 
macaque 	& \ifprintanswers{Animalia/Metazoa}\else\rule{1.7in}{0.4pt}\fi \tabularnewline[1.5ex]
%
ant 		& \ifprintanswers{Animalia/Metazoa}\else\rule{1.7in}{0.4pt}\fi &
\emph{Homo sapiens} & \ifprintanswers{Animalia/Metazoa}\else\rule{1.7in}{0.4pt}\fi \tabularnewline[1.5ex]
%
bass 	& \ifprintanswers{Animalia/Metazoa}\else\rule{1.7in}{0.4pt}\fi &
maple 	& \ifprintanswers{Animalia/Metazoa}\else\rule{1.7in}{0.4pt}\fi  \tabularnewline[1.5ex]
\bottomrule
\end{longtable}
}


\fullwidth{%
You must now find out more detailed classification
information for some organisms to complete a table on the last page. 
As you saw in the table above, each species is a member of a genus,
which may contain several species. Each genus is part of a family,
which can contain several genera (plural of genus). Each family is part
of an order, an order is part of a phylum, a phylum is part of a
kingdom, a kingdom is part of a domain.

You can find out all of these taxonomic levels for a species by looking
it up in the National Center for Biotechnology Information (NCBI), part
of the National Institutes of Health. It's a huge searchable database
of freely available information, paid for by your tax dollars. For this
assignment, go to the Taxonomy Home Page
(\url{http://www.ncbi.nlm.nih.gov/Taxonomy/}). Go to it now, and follow along
in this worksheet. The taxonomy home page looks like this:

\begin{center}
	\includegraphics[width=0.9\textwidth]{01_taxonomy_browser01}
\end{center}

Type the common name of the species you want to look up in the “search
for" box (arrow). Then click “Search”. If you type in “broccoli” (and spell it right) you'll get
this page:
}

\newpage

\fullwidth{%
\begin{center}
	\includegraphics[width=0.9\textwidth]{01_taxonomy_browser02}
\end{center}

If you hold the cursor over one of the words in the “Lineage” list,
you'll see a little box pop up after a moment. Here I had the cursor on
“Brassicales,” and the box with “order” popped up. You
can use this technique to find all the taxonomic levels for broccoli. 
Just work your way up from the species name, looking for the taxonomic
levels in the table (some will say “no rank” — just skip them and
keep looking). Then copy the terms into the appropriate boxes in the
table on the next page, and do the same for the other organisms I asked about.

Fill in the table on the next page for each of the species shown. Sometimes you will
search and you will get a list of species names instead of a page like the
one for broccoli. Just click on the species that you want, and that
will take you to the right page.
}

\newpage

\question[14]
Fill in the following table. Be sure to use proper treatment of
species names. (0.5 pts per blank.)

\end{questions}

\begin{longtable}[l]{@{}L{0.75in}L{1.1in}L{1.1in}L{1.1in}L{1.1in}@{}}
\toprule
& & & & \\
\textbf{Kingdom} & \ifprintanswers{Metazoa}\else\rule{1in}{0.4pt}\fi %
			  & \ifprintanswers{Viridiplantae}\else\rule{1in}{0.4pt}\fi %
			  & \ifprintanswers{Metazoa}\else\rule{1in}{0.4pt}\fi %
			  & \ifprintanswers{Metazoa}\else\rule{1in}{0.4pt}\fi \tabularnewline[3ex]
			  %
\textbf{Phylum} 	& \ifprintanswers{Chordata}\else\rule{1in}{0.4pt}\fi %
			& \ifprintanswers{Streptophyta}\else\rule{1in}{0.4pt}\fi %
			& \ifprintanswers{Chordata}\else\rule{1in}{0.4pt}\fi %
			& \ifprintanswers{Chordata}\else\rule{1in}{0.4pt}\fi \tabularnewline[3ex]
%
\textbf{Class} 	& \ifprintanswers{Mammalia}\else\rule{1in}{0.4pt}\fi %
			& \ifprintanswers{Liliopsida}\else\rule{1in}{0.4pt}\fi %
			& \ifprintanswers{Mammalia}\else\rule{1in}{0.4pt}\fi %
			& \ifprintanswers{Mammalia}\else\rule{1in}{0.4pt}\fi \tabularnewline[3ex]
%
\textbf{Order} 	& Cetartiodactyla %
			& \ifprintanswers{Poales}\else\rule{1in}{0.4pt}\fi %
			& \ifprintanswers{Primates}\else\rule{1in}{0.4pt}\fi %
			& \ifprintanswers{Primates}\else\rule{1in}{0.4pt}\fi \tabularnewline[3ex]
%
\textbf{Family} & \ifprintanswers{Bovidae}\else\rule{1in}{0.4pt}\fi %
			& \ifprintanswers{Poaceae}\else\rule{1in}{0.4pt}\fi %
			& \ifprintanswers{Hominidae}\else\rule{1in}{0.4pt}\fi%
			& \ifprintanswers{Hominidae}\else\rule{1in}{0.4pt}\fi \tabularnewline[3ex]
%
\textbf{Genus} 	& \ifprintanswers{\textit{Bison}}\else\rule{1in}{0.4pt}\fi %
			& \ifprintanswers{\textit{Triticum}}\else\rule{1in}{0.4pt}\fi %
			& \ifprintanswers{\textit{Pan}}\else\rule{1in}{0.4pt}\fi %
			& \ifprintanswers{\textit{Homo}}\else\rule{1in}{0.4pt}\fi \tabularnewline[3ex]
%
\textbf{Species} & \ifprintanswers{\textit{Bison bison}}\else\rule{1in}{0.4pt}\fi %
			& \ifprintanswers{\textit{Triticum aestivum}}\else\rule{1in}{0.4pt}\fi %
			& \ifprintanswers{\textit{Pan troglodytes}}\else\rule{1in}{0.4pt}\fi %
			& \ifprintanswers{\textit{Homo sapiens}}\else\rule{1in}{0.4pt}\fi\tabularnewline[3ex]
%
\textbf{common name} & American bison %
			& bread wheat %
			& chimpanzee %
			& human\tabularnewline
\bottomrule
\end{longtable}


Note the name of our species. It's \emph{Homo sapiens}.
These are Latin words. “Homo” means “human", and “sapiens" means
“wise”. \emph{Homo sapiens} is \emph{not} plural — Latin words don't
form plurals by adding an “s” the way English words do. That means that
there is \emph{no such thing} as “a \textit{Homo sapien}”. That's just wrong—don't
write it, or you will have points counted off. 

By the way, the word “species” is similar, also being originally Latin.
One species, two species. No change in singular and plural forms. There
is an English word “specie”, but it doesn’t mean anything related to
biology. Look it up.



\end{document}  