%!TEX TS-program = lualatex
%!TEX encoding = UTF-8 Unicode

\documentclass[12pt]{exam}
\usepackage{graphicx}
	\graphicspath{{/Users/goby/Pictures/teach/163/lab/}} % set of paths to search for images

\usepackage{geometry}
\geometry{letterpaper, bottom=1in}                   

\usepackage{afterpage}
\usepackage{pdflscape}

\newlength{\myindent}
\setlength{\myindent}{\parindent}
\newcommand{\ind}{\hspace*{\myindent}}


%\geometry{landscape}                % Activate for for rotated page geometry
\usepackage[parfill]{parskip}    % Activate to begin paragraphs with an empty line rather than an indent
%\usepackage{amssymb, amsmath}
%\usepackage{mathtools}
%	\everymath{\displaystyle}

\usepackage{fontspec}
\setmainfont[Ligatures={TeX}, BoldFont={* Bold}, ItalicFont={* Italic}, BoldItalicFont={* BoldItalic}, Numbers={OldStyle,Proportional}]{Linux Libertine O}
\setsansfont[Scale=MatchLowercase,Ligatures=TeX, Numbers=OldStyle]{Linux Biolinum O}
\setmonofont[Scale=MatchLowercase]{Inconsolatazi4}
\usepackage{microtype}

\usepackage{unicode-math}
\setmathfont[Scale=MatchLowercase]{Asana Math}
%\setmathfont[Scale=MatchLowercase]{XITS Math}

% To define fonts for particular uses within a document. For example, 
% This sets the Libertine font to use tabular number format for tables.
\newfontfamily{\tablenumbers}[Numbers={Monospaced}]{Linux Libertine O}
\newfontfamily{\libertinedisplay}{Linux Libertine Display O}

\usepackage{longtable}

\usepackage{booktabs}
\usepackage{multirow}
\usepackage{multicol}

\usepackage[justification=raggedright, labelsep=period]{caption}
\captionsetup{singlelinecheck=off}
\captionsetup{skip=0.2em}

%\usepackage{tabularx}
%\usepackage{siunitx}
\usepackage{array}
\newcolumntype{L}[1]{>{\raggedright\let\newline\\\arraybackslash\hspace{0pt}}p{#1}}
\newcolumntype{C}[1]{>{\centering\let\newline\\\arraybackslash\hspace{0pt}}p{#1}}
\newcolumntype{R}[1]{>{\raggedleft\let\newline\\\arraybackslash\hspace{0pt}}p{#1}}

\newcolumntype{M}[1]{>{\centering\let\newline\\\arraybackslash\hspace{0pt}}m{#1}}


\usepackage{enumitem}
\setlist{leftmargin=*}
\setlist[1]{labelindent=\parindent}
\setlist[enumerate]{label=\textsc{\alph*}., ref=\textsc{\alph*}}

\usepackage{hyperref}
%\usepackage{hanging}

\usepackage[sc]{titlesec}


\renewcommand{\solutiontitle}{\noindent}
\unframedsolutions
\SolutionEmphasis{\bfseries}

\renewcommand{\questionshook}{%
	\setlength{\leftmargin}{-\leftskip}%
}
%Change \half command from 1/2 to .5
%\renewcommand*\half{.5}


\makeatletter
\def\SetTotalwidth{\advance\linewidth by \@totalleftmargin
\@totalleftmargin=0pt}
\makeatother



\pagestyle{headandfoot}
\firstpageheader{BI 063: Evolution and Ecology}{}{\ifprintanswers\textbf{KEY}\else Name: \enspace \makebox[2.5in]{\hrulefill}\fi}
\runningheader{}{}{\footnotesize{pg. \thepage}}
\footer{}{}{}
\runningheadrule

\newcommand*\AnswerBox[2]{%
    \parbox[t][#1]{0.92\textwidth}{%
    \begin{solution}#2\end{solution}}
    \vspace*{\stretch{1}}
}

\newenvironment{AnswerPage}[1]
    {\begin{minipage}[t][#1]{0.92\textwidth}%
    \begin{solution}}
    {\end{solution}\end{minipage}
    \vspace*{\stretch{1}}}

\newlength{\basespace}
\setlength{\basespace}{5\baselineskip}

\newcommand{\allele}[1]{\textit{#1}}

%\printanswers

\begin{document}

\subsection*{Evolution: natural selection and gene flow\footnote{Modified from Morgan and Carter 2005. \emph{Investigating Biology Laboratory Manual,} 5th Ed., Pearson Education Inc.}}

Evolution is be defined as genetic change in a population over time. The genetic change that happens in a population over time is the change in allele frequencies. Alleles are versions of genes that determine the traits of individuals, such as eye color and hair color. Frequency tells how common or uncommon different alleles are in a population. Common alleles have a high frequency while uncommon alleles have a low frequency.  

Biologists have identified five evolutionary processes that cause populations to evolve: natural selection, gene flow, genetic drift, mutation, and non-random mating. You will explore natural selection and gene flow in this lab.  The first part of this lab will teach you how to calculate allele frequencies in a population. The second part will teach you how to use allele frequencies to calculate genotype frequencies.  The last two parts will demonstrate how natural selection and gene flow change allele and genotype frequencies in a population over time. 

\subsubsection*{How to calculate allele frequencies}

For this lab, you will work under the following assumptions:

\begin{enumerate}

	\item The organisms are diploid. Diploid organisms have two alleles for each gene.
	
	\item Only two unique alleles exist for one gene in the entire population, such as \allele{Z} and \allele{z.}
	
	\item The population is very large.
	
	\item Mating is random.
	
\end{enumerate}

If only two alleles in a population, then only three genotypes are possible. If the population has alleles \allele{A} and \allele{a,} then the three possible genotypes are \allele{AA,} \allele{Aa,} and \allele{aa.}  If you know how many individuals of each genotype are present in the population, then calculating allele frequencies is very easy. 

\begin{enumerate}
	\item Count the total number of individuals in the population. Multiply the number of individuals by 2 to get the total number of alleles in the population for that gene. (Remember these are diploid individuals so each individual contributes two alleles to the population.) 
	
	Example: The population size is 100 individuals. Then the total number of alleles is 100 $\times$ 2 = 200.
	
	\item Count the number of individuals that have the \allele{AA} genotype. Multiply by 2 to get the number of \allele{A} alleles contributed by the \allele{AA} individuals.
	
	Example: The population has 64 \allele{AA} individuals. 64 $\times$ 2 = 128 \allele{A} alleles that come from \allele{AA} individuals.
	
	\item Count the number of individuals that have the \allele{aa} genotype. Multiply by 2 to get the number of \allele{a} alleles contributed by the \allele{aa} individuals.
	
	Example: The population has 4 \allele{aa} individuals. 4 $\times$ 2 = 8 \allele{a} alleles that come from \allele{aa} individuals.
	
	\item Count the number of individuals that have the \allele{Aa} genotype. Heterozygous individuals contribute one of each type of allele. Therefore, multiply the number of heterozygotes by 1 to get the number of \allele{A} alleles contributed by the \allele{Aa} individuals. That will also be the number of \allele{a} alleles contributed by the \allele{Aa} individuals.
	
	Example: The population has 32 \allele{Aa} individuals. 32 $\times$ 1 = 32 \allele{A} alleles and 32 \allele{a} alleles that come from \allele{Aa} individuals.
	
	\item Add together the number of alleles contributed by each genotype to get the total number of each allele in the population. 
	
	Example: The \allele{AA} genotype contributed 128 \allele{A} alleles and the \textit{Aa} genotype contributed 32 \textit{A} alleles. 128 $+$ 32 = 160 \allele{A} alleles.  The \allele{aa} genotype contributed 8 \allele{a} alleles and the \allele{Aa} genotype contributed 32 \allele{a} alleles. 8 $+$ 32 = 40 \allele{a} alleles. Notice that 160 $+$ 40 = 200, the total number of alleles in the population. \emph{If the numbers of each allele to not add up to the total number of alleles, carefully check your math to find and correct your error.}
	
	\item Calculate the frequencies by dividing the number of each allele by the total number of alleles in the population.
	
	Example: The frequency of the \allele{A} allele is 160/200 = 0.8. The frequency of the \allele{a} allele is 40/200 = 0.2.  The frequencies must add up to 1 to show that you have accounted for all of the alleles in the population. Another way to think about this is that 0.8 $\times$ 100\% = 80\% of the alleles in the population are the \allele{A} allele. 0.2 $\times$ 100\% = 20\% of the alleles in the population are the \allele{a} allele. Adding together 80\% and 20\% shows that 100\% of the alleles in the population have been accounted for. \emph{If the frequencies do not add up to 1, carefully check your math to find and correct your error.}
	
\end{enumerate}

\begin{questions}

\question
Calculate the allele frequencies for a population of 98 \allele{bb}, 84 \allele{Bb}, and 18 \allele{BB} individuals.

\begin{tabular}{@{}cc@{}}
	\toprule
	Allele & Frequency\tabularnewline
	\midrule
	 & \tabularnewline
	\allele{b} & \rule{0.5in}{0.4pt}\tabularnewline[2em]
	\allele{B} & \rule{0.5in}{0.4pt}\tabularnewline
	\bottomrule 
\end{tabular}

\vspace*{\baselineskip}

Ask your instructor to check your answers before proceeding to the next section. 

\newpage

\subsubsection*{How to calculate genotype frequencies}

If you know the allele frequencies in a population then you can estimate the genotype frequencies in the population. Like allele frequencies, genotype frequencies represent how common each genotype is in a population.  You can estimate genotype frequencies with a Punnett square, which you may already know how to use.  A Punnett square shows the possible genotypes that can be produced by two parents of known genotypes. For example, if two \allele{Cc} parents mate, then the possible genotypes of their offspring are

\begin{center}
	\begin{tabular}{cc|c|c|}
		\multicolumn{2}{c}{}	& \multicolumn{2}{c}{Parent 1}\tabularnewline
		\multicolumn{2}{c}{}	& \multicolumn{1}{c}{\allele{C}}	& \multicolumn{1}{c}{\allele{c}} \tabularnewline
		\cline{3-4}
		\multirow{2}{*}{Parent 2}	& \allele{C}	& \allele{CC}	& \allele{Cc} \tabularnewline
		\cline{3-4}
				&	\allele{c} & \allele{Cc}	& \allele{cc}	\tabularnewline
		\cline{3-4}
	\end{tabular}
\end{center}

\bigskip

The key to using Punnett squares to calculate genotype frequencies is to know that frequencies are equal to probabilities. You can see this using a coin.  A coin has a head side and a tail side (two sides total, of course, for a fair coin). The frequency of heads is 1/2 = 0.5 and the frequency for tails is 1/2 = 0.5.  If you flip the coin, the probability of landing heads-up is 0.5. The probability of landing tails up is also 0.5. Frequencies equal probabilities.

Imagine you flip two coins (or flip the same coin twice). The first coin (or flip) has a 0.5 probability of landing heads-up and a 0.5 probability of landing tails up. The second coin (or flip) has the same probabilities for heads or tails. Using H for heads and T for tails in a Punnett square, the possible results (left table) and the probabilities (right table) of two flips is

\begin{multicols}{2}
\begin{center}
	\begin{tabular}{cc|M{0.5in}|M{0.5in}|}
		\multicolumn{2}{c}{}	& \multicolumn{2}{c}{Coin 1}\tabularnewline
		\multicolumn{2}{c}{}	& \multicolumn{1}{c}{H}	& \multicolumn{1}{c}{T} \tabularnewline
		\cline{3-4}
		\multirow{2}{*}{Coin 2}	& H	& HH	 & TH \tabularnewline
		\cline{3-4}
			&	T & HT	&TT	\tabularnewline
		\cline{3-4}
	\end{tabular}
\end{center}

\columnbreak

\begin{center}
	\begin{tabular}{cc|M{0.5in}|M{0.5in}|}
		\multicolumn{2}{c}{}	& \multicolumn{2}{c}{Coin 1}\tabularnewline
		\multicolumn{2}{c}{}	& \multicolumn{1}{c}{0.5}	& \multicolumn{1}{c}{0.5} \tabularnewline
		\cline{3-4}
		\multirow{2}{*}{Coin 2}	& 0.5	& 0.25	 & 0.25 \tabularnewline
		\cline{3-4}
			&	0.5 & 0.25 	&0.25	\tabularnewline
		\cline{3-4}
	\end{tabular}
\end{center}

\end{multicols}

\bigskip

The probability of getting flipping two heads is the probability of landing heads-up for the first coin (0.5) times the probability of landing heads-up for the second coin (0.5): 0.5 $\times$ 0.5 = 0.25. The probability of two coins (or two flips) landing heads-up is 25\%. The probability of flipping two coins tails-up is also 0.5 $\times$ 0.5 = 0.25.   

Notice that you can get 1 head and 1 tail two different ways. The first flip lands heads-up and the second flip lands tails-up (left table, lower left cell). Or, the first flip can land tails-up and the second flip lands heads-up (left table, upper right cell). To get the total probability of getting 1 head and 1 tail from two flips, add together the two probabilities, e.g., 0.25 $+$ 0.25 = 0.5. Thus, if you make two flips, you are twice as likely to get 1 head and 1 tail than you are to get either 2 heads or 2 tails.

The same logic applies for allele frequencies if the population size is very large and if mating is random (recall two of the assumptions from above). Assume that allele \allele{D} has a frequency of 0.6 and allele \allele{d} has a frequency of 0.4. An egg from a randomly drawn female in the population has a 0.6 probability of having the \allele{D} allele and 0.4 probability of having the \allele{d} allele. A sperm from a randomly drawn male in the population also has the same 0.6 and 0.4 probabilities for the \allele{D} and \allele{d} alleles, respectively. In the Punnett square, the possible genotypes (left) and their probabilities (right) are

\begin{multicols}{2}
  \begin{center}
  	\begin{tabular}{cc|M{0.5in}|M{0.5in}|}
  		\multicolumn{2}{c}{}	& \multicolumn{2}{c}{Egg}\tabularnewline
  		\multicolumn{2}{c}{}	& \multicolumn{1}{c}{\allele{D}}	& \multicolumn{1}{c}{\allele{d}} \tabularnewline
  		\cline{3-4}
  		\multirow{2}{*}{Sperm}	& \allele{D}	& \allele{DD}	 & \allele{dD} \tabularnewline
  		\cline{3-4}
  			&	\allele{d} & \allele{Dd}	& \allele{dd}	\tabularnewline
  		\cline{3-4}
  	\end{tabular}
  \end{center}
  
  \columnbreak
  
  \begin{center}
  	\begin{tabular}{cc|M{0.5in}|M{0.5in}|}
  		\multicolumn{2}{c}{}	& \multicolumn{2}{c}{Egg}\tabularnewline
  		\multicolumn{2}{c}{}	& \multicolumn{1}{c}{0.6}	& \multicolumn{1}{c}{0.4} \tabularnewline
  		\cline{3-4}
  		\multirow{2}{*}{Sperm}	& 0.6	& 0.36	 & 0.24 \tabularnewline
  		\cline{3-4}
  			&	0.4 & 0.24 	&0.16	\tabularnewline
  		\cline{3-4}
  	\end{tabular}
  \end{center}
\end{multicols}

The probability of getting the \allele{DD} genotype is 0.36. The probability of getting the \allele{dd} genotype is 0.16. The probability of of getting the \allele{Dd} genotype is 0.24 $+$ 0.24 = 0.48. Remember that probabilities and frequencies are the same thing?  In a very large, randomly mating population with allele frequencies 0.6 and 0.4, the genotype frequencies will be very close to 0.36, 0.48, and 0.16.

\question
Calculate the possible genotypes and their frequencies in a population with allele \allele{E} that has a frequency of 0.2 and \allele{e} that has a frequency of 0.8. Fill in the tables below with your results. 

\begin{multicols}{2}
  \begin{center}
  	\begin{tabular}{cc|M{0.5in}|M{0.5in}|}
  		\multicolumn{2}{c}{}	& \multicolumn{2}{c}{Egg}\tabularnewline
  		\multicolumn{2}{c}{}	& \multicolumn{1}{c}{\allele{E}}	& \multicolumn{1}{c}{\allele{e}} \tabularnewline
  		\cline{3-4}
  		\multirow{2}{*}{Sperm}	& \allele{E}	& 	 &  \tabularnewline[2em]
  		\cline{3-4}
  			&	\allele{e} & 	& 	\tabularnewline[2em]
  		\cline{3-4}
  	\end{tabular}
  \end{center}
  
  \columnbreak
  
  \begin{center}
  	\begin{tabular}{cc|M{0.5in}|M{0.5in}|}
  		\multicolumn{2}{c}{}	& \multicolumn{2}{c}{Egg}\tabularnewline
  		\multicolumn{2}{c}{}	& \multicolumn{1}{c}{0.1}	& \multicolumn{1}{c}{0.9} \tabularnewline
  		\cline{3-4}
  		\multirow{2}{*}{Sperm}	& 0.2		& 	 &  \tabularnewline[2em]
  		\cline{3-4}
  			&	0.8 &  	&		\tabularnewline[2em]
  		\cline{3-4}
  	\end{tabular}
  \end{center}
\end{multicols}

Ask your instructor to check your results to be sure you understand how to calculate genotype frequencies from allele frequencies. 

\subsubsection*{Evolution by natural selection}

Natural selection is the differences in survival and reproductive success (relative fitness) among individuals in a population based on the interaction of their phenotype (traits) with the environment. This exercise is based on a famous study of the peppered moth (\textit{Bison betularia}), which lives in Great Britain. The peppered moth has two phenotypes, a light-colored form with dark speckles (hence, the common name) and a dark form that is nearly black. The peppered moth rests during the day on trees. The bark of the trees was also lightly colored so the light form moths were hard for predators to find. The dark moths were much easier for predators to find because the dark did not blend in with the light-colored tree bark. 

In the early 1800s, most peppered moths were the light form, probably because they were protected by their \emph{cryptic coloration} (camouflage). The dark form was very rare because they did not blend in with the bark. Following the Industrial Revolution, soot and other pollution from factories caused the bark of the trees to become much darker. Over time, the frequency of the dark form increased and the frequency of the light form decreased. 

The color of the peppered moth is determined by one gene with two alleles. The \allele{D} allele causes pigment production but the \allele{d} allele does not. If at least one dark allele is present then pigment will be produced, so moths that are homozygous \allele{DD} or heterozygous \allele{Dd} genotypes are dark. Only moths that are homozygous \allele{dd} are light. 

For this exercise, you will simulate how allele frequencies changed in a population of peppered moths due to the Industrial Revolution. 

\question\label{ques:selection_prediction}
Predict how the frequency of each allele should change from before the Industrial Revolution to after the Industrial Revolution?

\AnswerBox{2\baselineskip}{The \allele{d} allele should become less frequent over time. The \allele{D} allele should increase in frequency.}

\textsc{Procedure}

\medskip

You will work in pairs. One of you will sample alleles from the population while the other will record the results. You should take turns doing these tasks.  In the kit on your table, you have two containers with light and dark beads, some paper sacks, and some glass finger bowls. Each color of bead represents a different allele. For this simulation, you will use a starting frequency of 0.9 light alleles (the light colored bead) and 0.1 dark alleles (the dark bead). You will need enough beads to represent a population size of 50 individuals.

\begin{enumerate}

	\item Calculate the total number of alleles and write it in the blank. \hfill \rule{0.5in}{0.4pt}\\ \emph{Remember that peppered moths are diploid.} 
	
	\item Calculate the number of light alleles you need for a frequency of 0.9. \hfill \rule{0.5in}{0.4pt} \\ Count out that number of light beads and place them in a paper sack. 
	
	\item Calculate the number of dark alleles you need for a frequency of 0.1. \hfill \rule{0.5in}{0.4pt} \\ Count out that number of dark beads and add them to the sack with the light beads. Shake the sack to mix the alleles.
	
	\item This is Generation 0. Using 0.9 and 0.1 for the starting allele frequencies, calculate the genotype frequencies for Generation 0 and record them in Table~\ref{tab:selection_results} on page~\pageref{tab:selection_results}. \emph{Ask your instructor to check your calculations so that you start with the correct number of alleles.}
	
	\item \label{selection_sample_start} Reach into the bag and draw two alleles at random. Record the genotype (\allele{DD,} \allele{Dd,} or \allele{dd}) on a separate sheet of paper. Return the two alleles back to the population. (Put them back in the bag with the other beads).
	
	\textsc{Note:} Returning the beads to the bag is called \textbf{sampling with replacement.} Returning the alleles to the population simulates a very large population. Drawing the alleles at random simulates random mating.
	
	\item Repeat Step~\ref{selection_sample_start} until you have sampled the genotypes for 50 individuals. Record the genotype for each individual you sample. Remember to return the alleles to the population after each sample. 
	
	\item Add up the number of individuals for each genotype. These individuals are the offspring from Generation 0. Some of these individuals will not survive, thanks to the darker trees caused by the Industrial Revolution. Dark moths are less visible to predators so only 10\% are eaten by predators. In contrast, 50\% of the light moths are eaten. 
	
	How many light moths must you remove from the offspring to reflect the change in survivorship? How many dark moths? Can natural selection distinguish between dark moths with \allele{DD} and \allele{dd} genotypes? How will you decide which dark moths to remove? Remove the appropriate number of individuals of each phenotype.
	
	\item \label{selection_sample_stop} The surviving offspring are now Generation~1. Calculate new allele frequencies for this generation. Reestablish the populations with these new allele frequencies for the 100 alleles (see example below). Record the new allele frequencies in Table~\ref{tab:selection_results}.
	
	\textit{Ask your instructor to check your calculation to be sure you start Generation 1 with the correct allele frequencies.}
	
	\textsc{Example:} Assume that after predation, you have the following number of survivors:
	
	Number of individuals: 14 \allele{dd}, 21 \allele{Dd}, and 3 \allele{DD}.\\
	Number of alleles: $49$ \allele{d} and $27$ \allele{D}.\\
	Total number of alleles: $76$

	Frequency of \allele{d:} $49/76 = 0.64$\\
	Frequency of \allele{D:} $24/76 = 0.36$ (Round to two digits after the decimal.)

	Adjust the number of beads in your population so that you have 64 beads that represent the \allele{d} allele and 36 beads that represent the \allele{D} allele.  
	
	\item Repeat Steps~\ref{selection_sample_start}–\ref{selection_sample_stop} for five more generations (a total of six generations). Record the allele frequencies for each generation in Table~\ref{tab:selection_results}. Calculate the genotype frequencies for the final generation. 
	
\end{enumerate}

\question
Describe how the genotype frequencies changed over time for your population.

\newpage

\question
Sketch a graph of the frequencies of \allele{D} and \allele{d} changed over time.

\vspace*{\stretch{3}}

%\vskip0pt plus 1filll

%\newpage

\begin{table}[b!]
\begin{longtable}[l]{@{}C{0.75in}C{0.75in}C{0.75in}C{0.75in}C{0.75in}C{0.75in}@{}}
  \caption{Allele and genotype frequencies for peppered moths.}
  \label{tab:selection_results}\tabularnewline
  \toprule
  &
  \multicolumn{2}{c}{Allele Frequency}	&
  \multicolumn{3}{c}{Genotype Frequency}\tabularnewline
%
  \cmidrule(lr){2-3} 
  \cmidrule(l){4-6}
%
  Generation	&
  \allele{D}		&
  \allele {d} 	&
  \allele{DD} 	&
  \allele {Dd} 	&
  \allele {dd}	\tabularnewline
%
  \midrule
  & & & & & \tabularnewline
%
0		&
0.1	&
0.9	&
\rule{0.5in}{0.4pt}	&
\rule{0.5in}{0.4pt}	&
\rule{0.5in}{0.4pt}	\tabularnewline[2em]
%
1	&
\rule{0.5in}{0.4pt}	&
\rule{0.5in}{0.4pt}	&
& % \rule{0.5in}{0.4pt}	&
& % \rule{0.5in}{0.4pt}	&
\tabularnewline[2em]
 % \rule{0.5in}{0.4pt} \tabularnewline[2em]
%
2	&
 \rule{0.5in}{0.4pt} &
 \rule{0.5in}{0.4pt}	&
& % \rule{0.5in}{0.4pt}	&
& % \rule{0.5in}{0.4pt}	&
\tabularnewline[2em]
 % \rule{0.5in}{0.4pt} \tabularnewline[2em]
	3	&
 \rule{0.5in}{0.4pt} &
 \rule{0.5in}{0.4pt}	&
& % \rule{0.5in}{0.4pt}	&
& % \rule{0.5in}{0.4pt}	&
\tabularnewline[2em]
 % \rule{0.5in}{0.4pt} \tabularnewline[2em]
	4	&
 \rule{0.5in}{0.4pt} &
 \rule{0.5in}{0.4pt}	&
 & % \rule{0.5in}{0.4pt}	&
& % \rule{0.5in}{0.4pt}	&
\tabularnewline[2em]
 % \rule{0.5in}{0.4pt} \tabularnewline[2em]
	5	&
 \rule{0.5in}{0.4pt} &
 \rule{0.5in}{0.4pt}	&
& % \rule{0.5in}{0.4pt}	&
& % \rule{0.5in}{0.4pt}	&
\tabularnewline[2em]
 % \rule{0.5in}{0.4pt} \tabularnewline[2em]
6	&
\rule{0.5in}{0.4pt}	&
\rule{0.5in}{0.4pt}	&
\rule{0.5in}{0.4pt}	&
\rule{0.5in}{0.4pt}	&
\rule{0.5in}{0.4pt}	\tabularnewline[2em]
%7	&
%\rule{0.5in}{0.4pt}	&
%\rule{0.5in}{0.4pt}	&
%\rule{0.5in}{0.4pt}	&
%\rule{0.5in}{0.4pt}	&
%\rule{0.5in}{0.4pt}	\tabularnewline[2em]
%8	&
%\rule{0.5in}{0.4pt}	&
%\rule{0.5in}{0.4pt}	&
%\rule{0.5in}{0.4pt}	&
%\rule{0.5in}{0.4pt}	&
%\rule{0.5in}{0.4pt}	\tabularnewline[2em]
\bottomrule 
\end{longtable}
\end{table}

\question
Did the results agree with the prediction you made in question~\ref{ques:selection_prediction}? Explain.

\vspace{3\baselineskip}

\question
What mode of selection best matches your results? Explain.

\AnswerBox{1\baselineskip}{Directional selection. The population phenotype is changing in one direction, from light phenotype to dark phenotype.}

\subsubsection*{Evolution by gene flow}

Gene flow is defined as the movement of individuals among populations.  If an individual moves to a different population and reproduces, then its alleles are introduced into the new population.  This can alter allele frequencies of both populations. You will simulate gene flow using two populations with different allele frequencies.

For this exercise, you will simulate the movement of individuals between two populations. Each population will have 50 individuals each. Population 1 will have starting allele frequencies of 0.9 (\allele{d} allele) and 0.1 (\allele{D} allele). Population 2 will have a starting allele frequency of 0.5 for both alleles. Every generation, 10 individuals will move from Population 1 to Population 2, while 10 individuals move from Population 2 to Population 1.

\question\label{ques:migration_prediction}
Predict how the frequency of each allele will change for \emph{each} of the two populations due to gene flow. 

\AnswerBox{3\baselineskip}{Population 1: the frequency of \allele{d} should decrease and \allele{D} should increase. Population 2: the frequency of \allele{d} should increase and \allele{D} should decrease.}

\textsc{Procedure}

\medskip

\begin{enumerate}

	\item Count 90 light and 10 dark beads and place them together in one bag. You can use the population from the peppered moth simulation, but be sure to return it to the starting conditions of 90 light and 10 dark beads. Label the bag Population 1.
	
	\item Count 50 dark and 50 light beads and place them together in another bag. Label the bag Population 2.  Record the starting allele frequencies in the Generation 0 row in Table~\ref{tab:migration_results} on page~\pageref{tab:migration_results}.
	
	\item Select 10 individuals at random (how many alleles?) from each population and exchange them between the two populations. (Take the beads from the Population 1 bag and add them to the Population 2 bag. Take the beads from the Population 2 bag and add them to the Population 1 bag. Exchange them \emph{after} counting out the 10 individuals from each population.)
	
	\item Select 50 individuals from each population, \emph{sampling with replacement.} Record the genotypes and the number of \allele{D} and \allele{d} alleles for each population.
	
	\item Calculate the new allele and genotype frequencies in the two populations following migration. Record your results in the Generation~1 row in Table~\ref{tab:migration_results}. The population size of each population should still be 50 individuals.
	
	\item Repeat this procedure for 5 more generations (6 total). 

\end{enumerate}

\question
Sketch a graph of the frequencies of \allele{D} and \allele{d} changed over time.

\AnswerBox{4\baselineskip}{Their results will hopefully agree.}

\question
Describe how the genotype frequencies changed over time for your population.

\AnswerBox{2\baselineskip}{The \allele{dd} frequency should slowly decrease. The \allele{Dd} frequency increases first, followed by \allele{DD} frequency.}

\question
Did the results agree with the prediction you made in question~\ref{ques:migration_prediction}? Explain.

\AnswerBox{4\baselineskip}{Their results will hopefully agree.}


%\afterpage{%
\begin{landscape}

\begin{longtable}[l]{@{}%
	C{0.75in}
	C{0.55in}
	C{0.55in}
	C{0.55in}
	C{0.55in}
	C{0.55in}
	C{0.55in}
	C{0.55in}
	C{0.55in}
	C{0.55in}
	C{0.55in}@{}}
	\caption{Allele and genotype frequencies for migration.}
	\label{tab:migration_results}\tabularnewline
  \toprule
  &
  \multicolumn{5}{c}{Population 1} &
  \multicolumn{5}{c}{Population 2}\tabularnewline
%
  \cmidrule(lr){2-6} \cmidrule(lr){7-11}
 %
  & 
  \multicolumn{2}{c}{Allele Frequency}		&
  \multicolumn{3}{c}{Genotype Frequency}	&
  \multicolumn{2}{c}{Allele Frequency}		&
  \multicolumn{3}{c}{Genotype Frequency}\tabularnewline
%
  \cmidrule(lr){2-3}
  \cmidrule(lr){4-6}
  \cmidrule(lr){7-8}
  \cmidrule(lr){9-11}
%
  Generation	&
  \allele{D}		&
  \allele {d}		&
  \allele{DD}	&
  \allele {Dd}	&
  \allele {dd}	&
  \allele{D}		&
  \allele {d}		&
  \allele{DD}	&
  \allele {Dd}	&
  \allele {dd}	\tabularnewline
%
  \midrule
 & & & & & & & & & & \tabularnewline
%Row 1: Generation 0
0		& 
0.1 	& 
0.9	& 
\rule{0.45in}{0.4pt}	& 
\rule{0.45in}{0.4pt}	& 
\rule{0.45in}{0.4pt}	& 
0.5 	&
0.5	&
\rule{0.45in}{0.4pt}	&
\rule{0.45in}{0.4pt}	&
\rule{0.45in}{0.4pt}	\tabularnewline[2em]
%Row 2: Generation 1
1		&
\rule{0.45in}{0.4pt}	&
\rule{0.45in}{0.4pt}	&
& %\rule{0.45in}{0.4pt}	&
& %\rule{0.45in}{0.4pt}	&
& %\rule{0.45in}{0.4pt}	&
\rule{0.45in}{0.4pt} 	&
\rule{0.45in}{0.4pt}	&
& %\rule{0.45in}{0.4pt}	&
& %\rule{0.45in}{0.4pt}	&
\tabularnewline[2em]
% \rule{0.45in}{0.4pt}	\tabularnewline[2em]
%Row 3: Generation 2
2	&
\rule{0.45in}{0.4pt}	&
\rule{0.45in}{0.4pt}	&
& %\rule{0.45in}{0.4pt}	&
& %\rule{0.45in}{0.4pt}	&
& %\rule{0.45in}{0.4pt}	&
\rule{0.45in}{0.4pt} 	&
\rule{0.45in}{0.4pt}	&
& %\rule{0.45in}{0.4pt}	&
& %\rule{0.45in}{0.4pt}	&
\tabularnewline[2em]
% \rule{0.45in}{0.4pt}	\tabularnewline[2em]
%Row 4: Generation 3
3	&
\rule{0.45in}{0.4pt}	&
\rule{0.45in}{0.4pt}	&
& %\rule{0.45in}{0.4pt}	&
& %\rule{0.45in}{0.4pt}	&
& %\rule{0.45in}{0.4pt}	&
\rule{0.45in}{0.4pt} 	&
\rule{0.45in}{0.4pt}	&
& %\rule{0.45in}{0.4pt}	&
& %\rule{0.45in}{0.4pt}	&
\tabularnewline[2em]
% \rule{0.45in}{0.4pt}	\tabularnewline[2em]
%Row 5: Generation 4
4	&
\rule{0.45in}{0.4pt}	&
\rule{0.45in}{0.4pt}	&
& %\rule{0.45in}{0.4pt}	&
& %\rule{0.45in}{0.4pt}	&
& %\rule{0.45in}{0.4pt}	&
\rule{0.45in}{0.4pt} 	&
\rule{0.45in}{0.4pt}	&
& %\rule{0.45in}{0.4pt}	&
& %\rule{0.45in}{0.4pt}	&
\tabularnewline[2em]
% \rule{0.45in}{0.4pt}	\tabularnewline[2em]
%Row 6: Generation 5
5	&
\rule{0.45in}{0.4pt}	&
\rule{0.45in}{0.4pt}	&
& %\rule{0.45in}{0.4pt}	&
& %\rule{0.45in}{0.4pt}	&
& %\rule{0.45in}{0.4pt}	&
\rule{0.45in}{0.4pt} 	&
\rule{0.45in}{0.4pt}	&
& %\rule{0.45in}{0.4pt}	&
& %\rule{0.45in}{0.4pt}	&
\tabularnewline[2em]
% \rule{0.45in}{0.4pt}	\tabularnewline[2em]
%Row 7: Generation 6
6	&
 \rule{0.45in}{0.4pt}	&
 \rule{0.45in}{0.4pt}	&
 \rule{0.45in}{0.4pt}	&
 \rule{0.45in}{0.4pt}	&
 \rule{0.45in}{0.4pt}	&
 \rule{0.45in}{0.4pt}	&
 \rule{0.45in}{0.4pt}	&
 \rule{0.45in}{0.4pt}	&
 \rule{0.45in}{0.4pt}	&
 \rule{0.45in}{0.4pt}	\tabularnewline[2em]
%Row 8: Generation 7
%	7	&
% \rule{0.45in}{0.4pt}	&
% \rule{0.45in}{0.4pt}	&
% \rule{0.45in}{0.4pt}	&
% \rule{0.45in}{0.4pt}	&
% \rule{0.45in}{0.4pt}	&
% \rule{0.45in}{0.4pt}	&
% \rule{0.45in}{0.4pt}	&
% \rule{0.45in}{0.4pt}	&
% \rule{0.45in}{0.4pt}	&
% \rule{0.45in}{0.4pt}	\tabularnewline[2em]
%%Row 9: Generation 8
%8	&
% \rule{0.45in}{0.4pt}	&
% \rule{0.45in}{0.4pt}	&
% \rule{0.45in}{0.4pt}	&
% \rule{0.45in}{0.4pt}	&
% \rule{0.45in}{0.4pt}	&
% \rule{0.45in}{0.4pt}	&
% \rule{0.45in}{0.4pt}	&
% \rule{0.45in}{0.4pt}	&
% \rule{0.45in}{0.4pt}	&
% \rule{0.45in}{0.4pt	}	\tabularnewline[2em]
	\bottomrule 
\end{longtable}

\end{landscape}
%\clearpage
%}

\end{questions}
%
%\subsubsection*{Reestablishing a population with new allele frequencies}
%
%During the natural selection simulation, the number of individuals remaining after you account for predation will be lower than the starting number of 50 individuals. You will need to adjust the number of beads in your population to bring the total population size back to 50 individuals (100 alleles). For example, after one sample and predation, your results may be:
%




\end{document}  