%!TEX TS-program = lualatex
%!TEX encoding = UTF-8 Unicode

\documentclass[12pt]{exam}
\usepackage{graphicx}
	\graphicspath{{/Users/goby/Pictures/teach/163/labs}
	{img/}} % set of paths to search for images

\usepackage{geometry}
\geometry{letterpaper, left=1.5in, bottom=1in}                   
%\geometry{landscape}                % Activate for for rotated page geometry
\usepackage[parfill]{parskip}    % Activate to begin paragraphs with an empty line rather than an indent
\usepackage{amssymb, amsmath}
\usepackage{mathtools}
	\everymath{\displaystyle}

\usepackage{fontspec}
\setmainfont[Ligatures={TeX}, BoldFont={* Bold}, ItalicFont={* Italic}, BoldItalicFont={* BoldItalic}, Numbers={Proportional}]{Linux Libertine O}
\setsansfont[Scale=MatchLowercase,Ligatures=TeX]{Linux Biolinum O}
%\setmonofont[Scale=MatchLowercase]{Inconsolatazi4}
\usepackage{microtype}

\usepackage{unicode-math}
\setmathfont[Scale=MatchLowercase]{Asana Math}
%\setmathfont[Scale=MatchLowercase]{XITS Math}

% To define fonts for particular uses within a document. For example, 
% This sets the Libertine font to use tabular number format for tables.
\newfontfamily{\tablenumbers}[Numbers={Monospaced}]{Linux Libertine O}
\newfontfamily{\libertinedisplay}{Linux Libertine Display O}

\usepackage{booktabs}
\usepackage{multicol}
\usepackage[normalem]{ulem}

%\usepackage{tabularx}
\usepackage{longtable}
%\usepackage{siunitx}
\usepackage{array}
\newcolumntype{L}[1]{>{\raggedright\let\newline\\\arraybackslash\hspace{0pt}}p{#1}}
\newcolumntype{C}[1]{>{\centering\let\newline\\\arraybackslash\hspace{0pt}}p{#1}}
\newcolumntype{R}[1]{>{\raggedleft\let\newline\\\arraybackslash\hspace{0pt}}p{#1}}

\usepackage{enumitem}
\setlist{leftmargin=*}
\setlist[1]{labelindent=\parindent}
\setlist[enumerate]{label=\textsc{\alph*}.}

%\usepackage{hyperref}
%\usepackage{placeins} %PRovides \FloatBarrier to flush all floats before a certain point.
%\usepackage{hanging}

\usepackage[sc]{titlesec}


\renewcommand{\solutiontitle}{\noindent}
\unframedsolutions
\SolutionEmphasis{\bfseries}

\renewcommand{\questionshook}{%
	\setlength{\leftmargin}{-\leftskip}%
}

%Change \half command from 1/2 to .5
%\renewcommand*\half{.5}


\makeatletter
\def\SetTotalwidth{\advance\linewidth by \@totalleftmargin
\@totalleftmargin=0pt}
\makeatother


\pagestyle{headandfoot}
\firstpageheader{BI 063: Evolution and Ecology}{}{\ifprintanswers\textbf{KEY}\else Name: \enspace \makebox[2.5in]{\hrulefill}\fi}
\runningheader{}{}{\footnotesize{pg. \thepage}}
\footer{}{}{}
\runningheadrule

\newcommand*\AnswerBox[2]{%
    \parbox[t][#1]{0.92\textwidth}{%
    \begin{solution}#2\end{solution}}
    \vspace*{\stretch{1}}
}

\newenvironment{AnswerPage}[1]
    {\begin{minipage}[t][#1]{0.92\textwidth}%
    \begin{solution}}
    {\end{solution}\end{minipage}
    \vspace*{\stretch{1}}}

\newlength{\basespace}
\setlength{\basespace}{5\baselineskip}

%\printanswers

\begin{document}

\subsection*{The scientific method: digestion}

At the beginning of the twentieth century, it was known that the
pancreas, a small organ near the stomach, secreted digestive juices into
the duodenum, which connects the stomach with the small intestine. This
happened whenever partially digested food entered the duodenum from the
stomach. The question was whether the signal that started the pancreas
working was transmitted by the nervous system. 

In 1901, two physiologists, W.~M.~Bayliss and E.~H.~Starling, tried to 
answer the question by cutting all the nerves going to and from the 
duodenum of an experimental animal. They also inserted tubes to detect the 
flow of digestive juices from the pancreas to the duodenum. The result of this
experiment was that, when food entered the duodenum
of the experimental animal, the pancreas secreted digestive juices.\vspace*{\baselineskip}

Apply the scientific method to the above report. Determine what the various
components are and list them below (they may be a sentence or two or
more). \textbf{Write answers below in your own words. Do not just copy
sentences from above.}

\begin{questions}

\question
What is the real world observation?

\AnswerBox{5\baselineskip}{%
When food entered into the duodenum from the stomach, the pancreas
secreted digestive juices.}

\question
What is the hypothesis?.

\AnswerBox{5\baselineskip}{%
The nervous system transmitted the signal that caused the pancreas
to secrete digestive juices.}


%\newpage
%\vspace*{1\baselineskip}

\question
What is the prediction made the hypothesis?

\AnswerBox{5\baselineskip}{%
If the nerves to the duodenum are cut, then the pancreas
should not secrete digestive juices when food enters the stomach.}

\newpage

\question
What are the results that were collected?

\AnswerBox{5\baselineskip}{%
Bayliss and Starling cut the nerves leading to the duodenum. When food
entered the duodenum, digestive juices were secreted.}

\question
Do the results and predictions agree? Explain why.

\AnswerBox{5\baselineskip}{%
The prediction was digestive juices would not be secreted
because the nerves were cut. The results do not agree with this prediction. }

\question
What can you conclude about the hypothesis?

\AnswerBox{5\baselineskip}{%
The hypothesis (nerves as the signal) was falsified.}

\question[Checkout]
A competing hypothesis was that digestion was triggered by chemical signals in the blood. Based on the description of the experiment given earlier, can you confidently state that the competing hypothesis was supported by the experimental results? Explain.

\AnswerBox{7\baselineskip}{%
No because only the nervous system was tested. A separate experiment would be needed to performed to test the chemical signal hypothesis.}

\end{questions}

The research of Bayliss and Starling was an important early step towards understanding digestion. Their research did suggest that chemical signals in the blood were an important part of digestion. Subsequent research has showed that the digestive process is much more complex than described here. 

\end{document}  