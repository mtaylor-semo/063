%!TEX TS-program = lualatex
%!TEX encoding = UTF-8 Unicode

\documentclass[12pt, hidelinks]{exam}
\usepackage{graphicx}
	\graphicspath{{/Users/goby/Pictures/teach/163/lab/}
	{img/}} % set of paths to search for images

\usepackage{geometry}
\geometry{letterpaper, left=1.5in, bottom=1in}                   
%\geometry{landscape}                % Activate for for rotated page geometry
\usepackage[parfill]{parskip}    % Activate to begin paragraphs with an empty line rather than an indent
\usepackage{amssymb, amsmath}
\usepackage{mathtools}
	\everymath{\displaystyle}

\usepackage{fontspec}
\setmainfont[Ligatures={TeX}, BoldFont={* Bold}, ItalicFont={* Italic}, BoldItalicFont={* BoldItalic}, Numbers={OldStyle}]{Linux Libertine O}
\setsansfont[Scale=MatchLowercase,Ligatures=TeX]{Linux Biolinum O}
\setmonofont[Scale=MatchLowercase]{Linux Libertine Mono O}
\newfontfamily\liningnum[Numbers=Lining]{Linux Libertine O}
\usepackage{microtype}


% To define fonts for particular uses within a document. For example, 
% This sets the Libertine font to use tabular number format for tables.
 %\newfontfamily{\tablenumbers}[Numbers={Monospaced}]{Linux Libertine O}
% \newfontfamily{\libertinedisplay}{Linux Libertine Display O}

\usepackage{booktabs}
\usepackage{multicol}

\usepackage{tikz}
\tikzstyle{block} = [rectangle, draw, fill=white, rounded corners,
                 minimum size=2em]
\tikzstyle{branch} = [thick, draw]


\usepackage{longtable}
%\usepackage{siunitx}
\usepackage{array}
\newcolumntype{L}[1]{>{\raggedright\let\newline\\\arraybackslash\hspace{0pt}}p{#1}}
\newcolumntype{C}[1]{>{\centering\let\newline\\\arraybackslash\hspace{0pt}}p{#1}}
\newcolumntype{R}[1]{>{\raggedleft\let\newline\\\arraybackslash\hspace{0pt}}p{#1}}

\usepackage{enumitem}
\usepackage{hyperref}
%\usepackage{placeins} %PRovides \FloatBarrier to flush all floats before a certain point.
\usepackage{hanging}

\usepackage[sc]{titlesec}

%% Commands for Exam class
\renewcommand{\solutiontitle}{\noindent}
\unframedsolutions
\SolutionEmphasis{\bfseries}

\renewcommand{\questionshook}{%
	\setlength{\leftmargin}{-\leftskip}%
}

%Change \half command from 1/2 to .5
\renewcommand*\half{.5}

\pagestyle{headandfoot}
\firstpageheader{\textsc{bi}\,063 Evolution and Ecology}{}{\ifprintanswers\textbf{KEY}\else Name: \enspace \makebox[2.5in]{\hrulefill}\fi}
\runningheader{}{}{\footnotesize{pg. \thepage}}
\footer{}{}{}
\runningheadrule

\newcommand*\AnswerBox[2]{%
    \parbox[t][#1]{0.92\textwidth}{%
    \begin{solution}#2\end{solution}}
%    \vspace*{\stretch{1}}
}

\newenvironment{AnswerPage}[1]
    {\begin{minipage}[t][#1]{0.92\textwidth}%
    \begin{solution}}
    {\end{solution}\end{minipage}
    \vspace*{\stretch{1}}}

\newlength{\basespace}
\setlength{\basespace}{5\baselineskip}

%% To hide and show points
\newcommand{\hidepoints}{%
	\pointsinmargin\pointformat{}
}

\newcommand{\showpoints}{%
	\nopointsinmargin\pointformat{(\thepoints)}
}

\newcommand{\bumppoints}[1]{%
	\addtocounter{numpoints}{#1}
}

\newcommand{\dna}{\textsc{dna}}
%
%\makeatletter
%\def\SetTotalwidth{\advance\linewidth by \@totalleftmargin
%\@totalleftmargin=0pt}
%\makeatother


%\printanswers


\begin{document}

\subsection*{Using shared characters to make phylogenetic trees}

An important aspect of evolutionary biology is to determine the relationships among species, genera, families, and other taxonomic levels. Relationships can be estimated with many different (and complementary) methods.  Most methods result in a \emph{phylogenetic tree,} similar to a family genealogy. Below are trees that show a hypothetical family genealogy (left) and relationships among a group of organisms (right). 

\begin{center}
	\includegraphics[width=\textwidth]{05a_genealogy_phylogeny}
\end{center}

Just like the genealogy shows the relationships among family members, a phylogenetic tree shows relationships among organisms.  There is one important difference between family genealogies and phylogenetic trees. Family genealogies show \emph{known} relationships. Phylogenetic trees are \emph{hypotheses} about relationships. You will learn more about how to interpret phylogenetic trees later. First, you will learn how to assemble a data set that can be used to make phylogenetic tree.

As you learned in lecture, species share traits that they inherited from a common ancestor. Similar traits shared among species can be used to make hypothesis about relationships, in the form of a phylogenetic tree. \emph{Not all characters shared between species are suitable for making phylogenetic trees.} The choice of characters to use is based on careful study. For these exercises, you will use hypothetical characters or characters that have been tested and found to be suitable during these exercises.

One way to make phylogenetic trees using characters is by noting whether a character is present or absent in a group of species. If the character is present in multiple species, then those species are hypothesized to share a common ancestor.  The presence of a character can be noted by a “{\liningnum 1” and its absence is noted by a “0}.” (Any symbol can be used to indicate presence and absence, but {\liningnum 1}s and {\liningnum 0}s are easily analyzed by computers.) Below is a small data set of species and characters.

{\liningnum
\begin{center}
	\begin{tabular}{llll}
	\toprule
		Species	&	C1	&	C2	&	C3	\tabularnewline
	\midrule
		A	&	1	&	0	&	0	\tabularnewline
		B	&	1	&	1	&	1	\tabularnewline
		C	& 1	&	1	&	0	\tabularnewline
	\bottomrule
	\end{tabular}
\end{center}
}

Character C{\liningnum 1} is shared among species A, B, and C. That means all three species share a common ancestor. Character C{\liningnum 2} is shared between only species B and C. That means B and C also share a more recent common ancestor that is not shared with species A. Notice that character C{\liningnum 3} is not shared; it is found only in species B. \emph{Unshared characters are not useful for making phylogenetic trees,} even if the character is of biological interest.

%\newpage

\vspace{\baselineskip}

\begin{questions}

\question
You will now make a similar presence/absence data set from this list of characters found in several species of birds. 

{\liningnum
\begin{tabular}{ll}
\toprule
Character	&	Description	\tabularnewline
\midrule
C1	&	Has color on its throat.	\tabularnewline
C2	&	Flies faster than an unladen swallow \tabularnewline
C3	&	Knows how to tweet. \tabularnewline
C4	&	Knows that the bird is a word. \tabularnewline
C5	&	Rocks in the tree tops all day long. \tabularnewline
C6	&	Has hipster blue feet. Ironically. \tabularnewline
\bottomrule
\end{tabular}
}

\vspace{\baselineskip}

Here is the list of bird species and the characters each has from the list above. The single-letter  code after each species name corresponds to the table on the next page.

\begin{longtable}[l]{@{}lL{3.5in}@{}}
\toprule
Species	&	Characters Present	\tabularnewline
\midrule
Alfredo's Twitterer \textsc{(a)} & Knows how to tweet. Knows that the bird is a word. Rocks in the tree tops all day long.\tabularnewline
Blue-throated Garbler \textsc{(b)} & Has color on its throat. Flies faster than an unladen swallow. Knows that the bird is a word. Rocks in the tree tops all day long. \tabularnewline
California Garbler \textsc{(c)} & Flies faster than an unladen swallow. Knows that the bird is a word. Rocks in the tree tops all day long. \tabularnewline
Field Twitterer \textsc{(f)} & Knows how to tweet. Knows that the bird is a word. Rocks in the tree tops all day long. \tabularnewline
Red-footed Chump \textsc{(r)} & Rocks in the tree tops all day long. Has hipster blue feet. Ironically. \tabularnewline
Yellow-throated Garbler \textsc{(y)} & Has color on its throat. Flies faster than an unladen swallow. Knows that the bird is a word. Rocks in the tree tops all day long. \tabularnewline
\bottomrule
\end{longtable}

\newpage

Use the data above to complete the presence/absence data table. Write a {\liningnum 1} if the character is present in a species and a {\liningnum 0} if the character is absent in that species. Notice that the order of the birds in the first column is different from the order given above. Character C{\liningnum 6} has been filled in for you as a guide. You may work in pairs.

{\liningnum
\begin{longtable}[c]{@{}cC{0.32in}C{0.32in}C{0.32in}C{0.32in}C{0.32in}C{0.32in}@{}}
\toprule
Species & C1 & C2 & C3 & C4 & C5 & C6\tabularnewline
\midrule
	&	&	&	&	&	\tabularnewline
B	&
	\rule{0.3in}{0.4pt}	& 
	\rule{0.3in}{0.4pt}	& 
	\rule{0.3in}{0.4pt}	&
	\rule{0.3in}{0.4pt}	&
	\rule{0.3in}{0.4pt}	&
	0	\tabularnewline[0.15in]
%
Y	& 
	\rule{0.3in}{0.4pt}	&
	\rule{0.3in}{0.4pt}	&
	\rule{0.3in}{0.4pt}	&
	\rule{0.3in}{0.4pt}	&
	\rule{0.3in}{0.4pt}	&
	0	\tabularnewline[0.15in]
%
C	&
	\rule{0.3in}{0.4pt}	&
	\rule{0.3in}{0.4pt}	&
	\rule{0.3in}{0.4pt}	&
	\rule{0.3in}{0.4pt}	& 
	\rule{0.3in}{0.4pt}	&
	0	\tabularnewline[0.15in]
%
F	&
	\rule{0.3in}{0.4pt}	&
	\rule{0.3in}{0.4pt}	&
	\rule{0.3in}{0.4pt}	&
	\rule{0.3in}{0.4pt}	&
	\rule{0.3in}{0.4pt}	&
	0	\tabularnewline[0.15in]
%
A	&	
	\rule{0.3in}{0.4pt}	&
	\rule{0.3in}{0.4pt}	&
	\rule{0.3in}{0.4pt}	&
	\rule{0.3in}{0.4pt}	&
	\rule{0.3in}{0.4pt}	&
	0	\tabularnewline[0.15in]
%
R	&
	\rule{0.3in}{0.4pt}	&
	\rule{0.3in}{0.4pt}	&
	\rule{0.3in}{0.4pt}	&
	\rule{0.3in}{0.4pt}	&
	\rule{0.3in}{0.4pt}	&
	1	\tabularnewline[0.15in]
\bottomrule
\end{longtable}
}

\end{questions}

Your instructor will review the results after the class has compiled the presence/absence data. You will then learn how to use the data to make a phylogenetic tree.


\end{document}  