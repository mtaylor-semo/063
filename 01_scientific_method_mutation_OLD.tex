%!TEX TS-program = lualatex
%!TEX encoding = UTF-8 Unicode

\documentclass[12pt, addpoints]{exam}
\usepackage{graphicx}
	\graphicspath{{/Users/goby/Pictures/teach/163/lab/}
	{img/}} % set of paths to search for images

\usepackage{geometry}
\geometry{letterpaper, left=1.5in, bottom=1in}                   
%\geometry{landscape}                % Activate for for rotated page geometry
%\usepackage[parfill]{parskip}    % Activate to begin paragraphs with an empty line rather than an indent
\usepackage{amssymb, amsmath}
\usepackage{mathtools}
	\everymath{\displaystyle}

\usepackage{fontspec}
\setmainfont[Ligatures={TeX}, BoldFont={* Bold}, ItalicFont={* Italic}, BoldItalicFont={* BoldItalic}, Numbers={Proportional}]{Linux Libertine O}
\setsansfont[Scale=MatchLowercase,Ligatures=TeX]{Linux Biolinum O}
%\setmonofont[Scale=MatchLowercase]{Inconsolatazi4}
\usepackage{microtype}

\usepackage{unicode-math}
\setmathfont[Scale=MatchLowercase]{Asana Math}
%\setmathfont[Scale=MatchLowercase]{XITS Math}

% To define fonts for particular uses within a document. For example, 
% This sets the Libertine font to use tabular number format for tables.
\newfontfamily{\tablenumbers}[Numbers={Monospaced}]{Linux Libertine O}
\newfontfamily{\libertinedisplay}{Linux Libertine Display O}

\usepackage{booktabs}
\usepackage{multicol}
\usepackage[normalem]{ulem}

%\usepackage{tabularx}
\usepackage{longtable}
%\usepackage{siunitx}
\usepackage{array}
\newcolumntype{L}[1]{>{\raggedright\let\newline\\\arraybackslash\hspace{0pt}}p{#1}}
\newcolumntype{C}[1]{>{\centering\let\newline\\\arraybackslash\hspace{0pt}}p{#1}}
\newcolumntype{R}[1]{>{\raggedleft\let\newline\\\arraybackslash\hspace{0pt}}p{#1}}

\usepackage{enumitem}
\usepackage{hyperref}
%\usepackage{placeins} %PRovides \FloatBarrier to flush all floats before a certain point.
\usepackage{hanging}

\usepackage[sc]{titlesec}


\renewcommand{\solutiontitle}{\noindent}
\unframedsolutions
\SolutionEmphasis{\bfseries}

%Change \half command from 1/2 to .5
\renewcommand*\half{.5}


\makeatletter
\def\SetTotalwidth{\advance\linewidth by \@totalleftmargin
\@totalleftmargin=0pt}
\makeatother


\pagestyle{headandfoot}
\firstpageheader{BI 063: Evolution and Ecology}{}{\ifprintanswers\textbf{KEY}\else Name: \enspace \makebox[2.5in]{\hrulefill}\fi}
\runningheader{}{}{\footnotesize{pg. \thepage}}
\footer{}{}{}
\runningheadrule

\newcommand*\AnswerBox[2]{%
    \parbox[t][#1]{0.92\textwidth}{%
    \begin{solution}#2\end{solution}}
%    \vspace*{\stretch{1}}
}

\newenvironment{AnswerPage}[1]
    {\begin{minipage}[t][#1]{0.92\textwidth}%
    \begin{solution}}
    {\end{solution}\end{minipage}
    \vspace*{\stretch{1}}}

\newlength{\basespace}
\setlength{\basespace}{5\baselineskip}

%\printanswers

\begin{document}

\subsection*{A scientific method: mutations (\numpoints\ points)}

It has long been known that changes in the genes of organisms can occur.
Such changes are commonly called “mutations.” In the 1940s, it was not
yet known how mutations occur. Part of the answer to this question was
given by a famous experiment performed by Salvador Luria and Max
Delbrück.

Research that took place before Luria and Delbrück's experiment 
showed that some types of viruses (bacteriophages, or “phages”) 
could attack and kill some types of bacteria. It is relatively easy to grow 
bacteria in covered dishes containing nourishment in which bacteria 
generally thrive. These are called “bacterial cultures.”

Luria and Delbrück discovered that in some bacterial cultures a few of
the bacteria survive attacks by phages. Moreover, descendants of the
surviving bacteria tend also to survive phage attacks. This shows that
the genes of some of the bacteria had undergone mutations that made them
resistant to the phage, and that these resistant bacteria passed their
mutant genes onto their offspring.

The question remained as to whether the mutations that made the bacteria
resistant were caused by the attacking virus itself, or whether they
merely happened by chance. The experiment at issue was designed to
answer this question regarding the cause of the mutations.

Luria and Delbrück considered what would happen if a number (say twenty)
of bacterial cultures, each with a similar small number of bacteria,
were allowed to grow for a short time, then all were infected with the
same quantity of the phage, and then were allowed to grow some more. If
the phages were producing the mutations, they argued, then all the
bacterial cultures should end up with roughly the same number of
resistant bacteria.

On the other hand, if the mutations were arising by chance, it follows
that those bacterial cultures in which the chance mutation happened to
occur early in the experiment would end up with many more mutant
bacteria than those cultures in which the mutation happened to occur
late in the experiment. The earlier mutant bacteria would have longer to
multiply. Those cultures in which the chance mutation happened to occur
at some intermediate time would end up with an intermediate number of
bacteria. If it is a matter of pure chance when the mutations occur, one
would, therefore, expect that by the end of the experiment there would
be a large variation in the numbers of mutant bacteria in the different
bacterial cultures. Figure 1 below shows the predictions from the two
hypotheses:

\noindent\includegraphics[width=\textwidth]{02b_delbruck_luria}

\noindent{\small Figure 1. Graphic representation of possible outcomes from the 
Luria and Delbrück hypotheses. Each cluster represents a single bacterial plate 
descending from a single bacterium (at the top). Bacterial colonies were
exposed to the phage in the final generation (at the bottom). Time proceeds
downward from the top. Figure modified from image by Madeleine Price Ball, 
Wikimedia Commons, after Luria and Delbrück, 1943. }
\vspace*{\baselineskip}

Luria and Delbrück prepared a number of bacterial cultures, then
introduced the phage, and later found that the actual number of
resistant bacteria differed widely from one bacterial culture to the
next.

Apply the scientific method to the above report. Determine what the various
components are and list them on the next page. These will be graded for
correctness and the clarity of your explanations. \textbf{Write answers below in
 your own words. Do not just copy sentences from above.}

\begin{questions}

\question[1]
What is the real world observation?

\AnswerBox{4\baselineskip}{%
Some bacteria developed resistance to viruses. The resistance
was inherited by their offspring.}

\question[2]

The passage suggests two hypotheses. Write each of them here, as
hypotheses a and b (H\textsubscript{a} and H\textsubscript{b}).

\begin{enumerate}[label=\alph*.]

	\item H\textsubscript{a}: 
	\ifprintanswers{%
		\textbf{The viruses that infected the bacteria caused the mutations.}}
	\fi \vspace*{2\baselineskip}
	
	
	\item H\textsubscript{b}: 
	\ifprintanswers{%
		\textbf{The mutations that caused resistance occurred by chance.}}
	\fi \vspace*{2\baselineskip}

\end{enumerate}

\newpage

\question[2]
What is the prediction made based on each of the hypotheses you stated
above?

\begin{enumerate}[label=\alph*.]

	\item H\textsubscript{a}:
	\ifprintanswers
		\textbf{If the viruses bacteria caused the mutations, then each 
		culture should have about the same number of resistant colonies.}
		\vspace*{1\baselineskip}
	\else
		\vspace*{2\baselineskip}
	\fi

	\item H\textsubscript{b}:
	\ifprintanswers
		\textbf{If mutations occur by chance, then the number of resistant 
		colonies will vary among cultures.}
		\vspace*{1\baselineskip}
	\else
		\vspace*{2\baselineskip}
	\fi

\end{enumerate}


\question[1]
What are the results that were collected?

\AnswerBox{5\baselineskip}{%
Luria and Delbrück counted the number of resistant colonies in each culture 
at the end of the experiment. They found that the number of colonies varied
among cultures.}

\question[2]
Do the results and predictions agree (for each hypothesis)? Explain why or why not for each prediction.

\begin{enumerate}[label=\alph*.]

	\item H\textsubscript{a}:
	\ifprintanswers
		\textbf{The prediction was that the number of resistant colonies should
		be about they same but the number varied. The results do not agree 
		with this prediction. }
		\vspace*{1\baselineskip}
	\else
		\vspace*{2\baselineskip}
	\fi

	\item H\textsubscript{b}: 
	\ifprintanswers
		\textbf{The prediction was that the number of resistant colonies should
		vary among cultures. The results showed varying amounts of resistance
		so the results do agree with this prediction. }
		\vspace*{1\baselineskip}
	\else
		\vspace*{2\baselineskip}
	\fi

\end{enumerate}

\question[2]
What can you conclude about each of the hypotheses?

\AnswerBox{5\baselineskip}{%
The first hypothesis (viruses as the cause) was falsified.\\
The second hypothesis (random mutations as the cause) was supported.
}

\end{questions}

\end{document}  