%!TEX TS-program = lualatex
%!TEX encoding = UTF-8 Unicode

\documentclass[12pt, addpoints, hidelinks]{exam}
\usepackage{graphicx}
	\graphicspath{{/Users/goby/Pictures/teach/163/lab/}
	{img/}} % set of paths to search for images

\usepackage{geometry}
\geometry{letterpaper, left=1.5in, bottom=1in}                   
%\geometry{landscape}                % Activate for for rotated page geometry
\usepackage[parfill]{parskip}    % Activate to begin paragraphs with an empty line rather than an indent
\usepackage{amssymb, amsmath}
\usepackage{mathtools}
	\everymath{\displaystyle}

\usepackage{fontspec}
\setmainfont[Ligatures={TeX}, BoldFont={* Bold}, ItalicFont={* Italic}, BoldItalicFont={* BoldItalic}, Numbers={OldStyle}]{Linux Libertine O}
\setsansfont[Scale=MatchLowercase,Ligatures=TeX]{Linux Biolinum O}
\setmonofont[Scale=MatchLowercase]{Inconsolatazi4}
\usepackage{microtype}


% To define fonts for particular uses within a document. For example, 
% This sets the Libertine font to use tabular number format for tables.
 %\newfontfamily{\tablenumbers}[Numbers={Monospaced}]{Linux Libertine O}
% \newfontfamily{\libertinedisplay}{Linux Libertine Display O}

\usepackage{booktabs}
\usepackage{multicol}
\usepackage[normalem]{ulem}

\usepackage{longtable}
\usepackage{caption}
	\captionsetup{labelsep=period, singlelinecheck=off, skip=1pt}
%	\captionsetup{labelsep=period}
	
%\usepackage{siunitx}
\usepackage{array}
\newcolumntype{L}[1]{>{\raggedright\let\newline\\\arraybackslash\hspace{0pt}}p{#1}}
\newcolumntype{C}[1]{>{\centering\let\newline\\\arraybackslash\hspace{0pt}}p{#1}}
\newcolumntype{R}[1]{>{\raggedleft\let\newline\\\arraybackslash\hspace{0pt}}p{#1}}

\usepackage{enumitem}
\setlist{leftmargin=*}
\setlist[1]{labelindent=\parindent}
\setlist[enumerate]{label=\textsc{\alph*}.}
\setlist[itemize]{label=\color{gray}\textbullet}

\usepackage{hyperref}
%\usepackage{placeins} %PRovides \FloatBarrier to flush all floats before a certain point.
\usepackage{hanging}

\usepackage[sc]{titlesec}

%% Commands for Exam class
\renewcommand{\solutiontitle}{\noindent}
\unframedsolutions
\SolutionEmphasis{\bfseries}

\renewcommand{\questionshook}{%
	\setlength{\leftmargin}{-\leftskip}%
}

%Change \half command from 1/2 to .5
\renewcommand*\half{.5}

\pagestyle{headandfoot}
\firstpageheader{\textsc{bi}\,063 Evolution and Ecology}{}{\ifprintanswers\textbf{KEY}\else Name: \enspace \makebox[2.5in]{\hrulefill}\fi}
\runningheader{}{}{\footnotesize{pg. \thepage}}
\footer{}{}{}
\runningheadrule

\newcommand*\AnswerBox[2]{%
    \parbox[t][#1]{0.92\textwidth}{%
    \begin{solution}#2\end{solution}}
%    \vspace*{\stretch{1}}
}

\newenvironment{AnswerPage}[1]
    {\begin{minipage}[t][#1]{0.92\textwidth}%
    \begin{solution}}
    {\end{solution}\end{minipage}
    \vspace*{\stretch{1}}}

\newlength{\basespace}
\setlength{\basespace}{5\baselineskip}

\makeatletter 
\newlength\LongtableWidth 
\newcommand*{\org@longtable}{} 
\let\org@longtable\longtable 
\def\longtable{%
 \begingroup 
   \advance\c@LT@tables\@ne 
   \edef\x{LT@\romannumeral\c@LT@tables}% 
   \global\LongtableWidth\z@ 
   \@ifundefined{\x}{% 
   	% longtable width not available 
   }{% 
     \def\LT@entry##1##2{% 
       \global\advance\LongtableWidth##2\relax 
     }% 
     \@nameuse{\x}%
    }% 
    	% debug output 
    \typeout{* \x: \the\LongtableWidth}%
  \endgroup 
  \ifdim\LongtableWidth>\z@ 
  \setlength{\LTcapwidth}{\LongtableWidth}%
  \fi 
  \org@longtable 
}
\makeatother

%\printanswers

\usepackage{multirow}

\begin{document}

\subsection*{The Chips are down: natural selection (\numpoints\ points)}

The process of natural selection occurs because organisms vary in their
heritable characteristics, and because some variants survive and
reproduce better than others. As a result, the genetic structure of a
population changes through time, which is a factor in evolution.
Although evolution may be defined in terms of genetic change, natural
selection occurs by the interaction of the environment and whole
organisms. Individuals that survive longer and produce more offspring in
an environment are said to have greater fitness, compared to individuals
that don't live as long and produce fewer offspring in that same
environment.

This inquiry uses a model to see how competition between members of a
species to avoid predation, leads to change in that species
over time. We also model the role of predation as the environmental
selection pressure.

In this model, you will be the predator species. Pieces
paper will represent a single prey species (the ``Chips''). The Chips come
in different colors to represent natural variation in the prey species.
(Think of dogs. They all belong to the same species, but may look very
different, e.g., a Chihuahua vs. a Great Dane.) You will see the effect
 of the variants' ability to avoid predation on their survival and reproduction rates.

\subsubsection*{Material}

\begin{itemize}
\item
  One big sheet of newspaper with small printing (the classified pages
  are the best). This represents the environment that the Chips ``live''
  in.
\item
  An envelope containing the Chips (1 cm square bits of paper):
  50 white, 50 pink, 50 from newsprint pages.
\end{itemize}

\vspace*{\baselineskip}

\begin{questions}

\question
  Write a hypothesis about the relationship between the how well
  camouflaged Chips are and how well they will survive to reproduce.

	\AnswerBox{2\baselineskip}{\textbf{\emph{This question has no point value.}}}
	
\question
  For your hypothesis to be supported, predict what you expect to
  happen to the number of each Chip type by the end of the fourth
  generation?

	``newsprint'' Chips: \ifprintanswers\textbf{\emph{This question has no point value.}}\fi
	
	\vspace*{1\baselineskip}	

	``pink'' Chips: 

	\vspace*{1\baselineskip}	

	``white'' Chips:

%	\vspace*{2\baselineskip}	


\subsubsection*{How to play}

\begin{enumerate}

\item
  Work in groups of four. One member of the group will be the 
  ``Keeper'' who will set up the  playing board each time.  The 
  other three members of the group will be Predators. Each 
  predator will remove 5 Chips before the Chips have a chance 
  to reproduce. All four members of the team collect and analyze the data.

\item
  Lay out the big piece of newspaper on the desk.
  
\item
  Sort the Chips into piles by color, with at least 50 in each pile.
  Each Chip represents an individual.
  
\item
  The Keeper begins by taking 10 of each type of Chip, for a total of
  30.
  
\item\label{start_here}
  The Predators cover their eyes or turn away \emph{until it is their
  individual turn}.
  
\item
  The Keeper spreads the Chips out on the environment, mixing the colors
  well and spreading out the Chips.
  
\item
  The first Predator may look at the board and as quickly as possible,
  remove 5 Chips. These Chips are now dead and unable to reproduce.
  
\item
  The next Predator may now look at the board and remove 5 Chips as
  quickly as possible. These Chips are also dead and unable to reproduce.
  
\item
  The last Predator may look at the board and remove 5 Chips as quickly
  as possible. These Chips are also dead and unable to reproduce.
  
\item
  Shake the remaining Chips off the environment and count these
  survivors according to their type.
  
\item
  Record the number of surviving Chips in the table.
  
\item
  Each of these survivors reproduces one more Chip in their same color
  to make a population total of 30.
  
\item\label{end_here}
  Record the total number of each color of Chip as the next round begins
  (after reproduction).
  
\item
  Play this game (steps \ref{start_here}–\ref{end_here}) for a total of four rounds, recording
  your data in Table 1 each time.
  
\item After all groups of have finished, compile the class data in Table 2.
\end{enumerate}

%\begin{questions}

\newpage

\begin{longtable}[c]{@{}R{0.75in}L{2in}C{0.75in}C{0.75in}C{0.75in}@{}}
\caption{Effect of natural selection on different Chip varieties (group results).}\\
\toprule
 & & \multicolumn{3}{c}{Paper Chip Variants}\tabularnewline
\cmidrule(lr){3-5}
%\midrule
 Generation & & Newsprint & White & Pink\tabularnewline
\midrule
%
%& & & & \tabularnewline
\multirow{2}{*}{1} 			& 
	{\small Starting number} 	&
	10 					&
	10 					&
	10\tabularnewline[0.5cm]
	 %
	& 
	{\small \# left after 1st predation} 	&
	 \rule{0.7in}{0.4pt} 				& 
	 \rule{0.7in}{0.4pt} 				&	
	 \rule{0.7in}{0.4pt} \tabularnewline
%
\midrule
%
&&&&\tabularnewline
%
\multirow{2}{*}{2} 					& 
	{\small \# after 1st reproduction}	& 
	\rule{0.7in}{0.4pt} 				& 
	\rule{0.7in}{0.4pt} 				& 
	\rule{0.7in}{0.4pt} \tabularnewline[0.5cm]
	%
	& 
	{\small \# left after 2nd predation} 	& 
	\rule{0.7in}{0.4pt} 				& 
	\rule{0.7in}{0.4pt} 				&
	\rule{0.7in}{0.4pt} \tabularnewline
%
\midrule
%
&&&&\tabularnewline
%
\multirow{2}{*}{3} 					& 
	{\small \# after 2nd reproduction} 	& 
	\rule{0.7in}{0.4pt} 				& 
	\rule{0.7in}{0.4pt} 				&
	\rule{0.7in}{0.4pt} \tabularnewline[0.5cm]
	%
	& 
	{\small \# left after 3rd predation} 	&
	\rule{0.7in}{0.4pt} 				& 
	\rule{0.7in}{0.4pt} 				&
	\rule{0.7in}{0.4pt} \tabularnewline
%
\midrule
%
&&&&\tabularnewline
%
\multirow{2}{*}{4} 					& 
	{\small \# after 3rd reproduction} 	& 
	\rule{0.7in}{0.4pt} 				& 
	\rule{0.7in}{0.4pt} 				&
	\rule{0.7in}{0.4pt} \tabularnewline[0.5cm]
	%
	& 
	{\small \# left after 4th predation} 	& 
	\rule{0.7in}{0.4pt} 				& 
	\rule{0.7in}{0.4pt} 				&
	\rule{0.7in}{0.4pt} \tabularnewline
%
\midrule
%
&&&&\tabularnewline
%
Final 						&
{\small \# after 4th reproduction}	& 
\rule{0.7in}{0.4pt} 				& 
\rule{0.7in}{0.4pt} 				&
\rule{0.7in}{0.4pt} \tabularnewline
%
\midrule
%
&&&&\tabularnewline
%
 							& 
Calculate \% change 			& 
\rule{0.7in}{0.4pt} 				& 
\rule{0.7in}{0.4pt} 				& 
\rule{0.7in}{0.4pt} \tabularnewline
%
\bottomrule
%Footnote
\multicolumn{5}{L{5.5in}}{\footnotesize \% change = 100*{[}(Final number after the last reproduction $-$ Starting number) / Starting number){]}}
\end{longtable}

\vspace*{2\baselineskip}

\begin{longtable}[c]{@{}rC{0.75in}C{0.75in}C{0.75in}@{}}
\caption{Effect of natural selection on different Chip varieties (class results).}\tabularnewline

%\multicolumn{4}{l}{Effect of natural selection on different Chip varieties (class results).}\tabularnewline
\toprule
& \multicolumn{3}{c}{Paper Chip Variants}\tabularnewline
\cmidrule(lr){2-4}
%\midrule
& Newsprint & White & Pink\tabularnewline
\midrule
& & &\tabularnewline
Total starting number & \rule{0.7in}{0.4pt} & \rule{0.7in}{0.4pt} &\rule{0.7in}{0.4pt}\tabularnewline[0.5cm]

Total number after 4th reproduction & \rule{0.7in}{0.4pt} & \rule{0.7in}{0.4pt} & \rule{0.7in}{0.4pt}\tabularnewline[0.5cm]

Calculate \% change & \rule{0.7in}{0.4pt} &\rule{0.7in}{0.4pt} &\rule{0.7in}{0.4pt}\tabularnewline
\bottomrule
\end{longtable}

\newpage

\question
  Examine the data in Table 1. Does any Chip variant have more survivors
  than the others? If so, which?

	\AnswerBox{2\baselineskip}{Usually newsprint and white. \emph{No point value.}}

\question
  Does your individual data agree with the class data? Is it similar or
  different? Why?

	\AnswerBox{2\baselineskip}{Usually similar. \emph{No point value.}}


\question[1]
  Based on the experience of your group, does it take Predators a longer or shorter period of time to find one
  prey individual as they proceed through the generations? Give an
  explanation for this.
  
  	\AnswerBox{2\baselineskip}{Usually starts to take longer. There are fewer pink chips, which are easier to spot.}

\question[1]
 Were your hypothesis in Question 1 and your predictions in Question 2
  supported?

	\AnswerBox{2\baselineskip}{Answer depends on their hypotheses and predictions.}

\question[2]
  Based on the class data in Table 2, which of the Chip variants was the best
  competitor (least affected by natural selection) in this habitat? Why?

	\AnswerBox{2\baselineskip}{Usually the newsprint. They match the habitat the best (better camouflage) so escape predators better.}

\question[2]
  Now think about natural selection acting on the predator. What do you
  suppose will happen to predators that are a little bit better at
  finding camouflaged chips? What about predators that are not as good at
  finding camouflaged chips?

	\AnswerBox{3\baselineskip}{Predators that are better at finding food will survive longer and reproduce more. Predators that are not as good will reproduce less. Difference in relative fitness.}

\question[2]
  Apply the concepts of natural selection to the relationship between a
  flowering plant and its pollinators. Explain what you think would
  happen to the flowers if pollinators are attracted more to one color
  of flower than another color?

	\AnswerBox{3\baselineskip}{The plants with the preferred flower color will reproduce more due to more pollination and thus have higher relative fitness. The preferred flower color will become more common over time. The other flower color will become less common.}
	
\end{questions}

\end{document}  