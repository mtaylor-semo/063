%!TEX TS-program = lualatex
%!TEX encoding = UTF-8 Unicode

\documentclass[12pt]{exam}
\usepackage{graphicx}
	\graphicspath{{/Users/goby/Pictures/teach/163/lab/}
	{img/}} % set of paths to search for images

\usepackage{geometry}
\geometry{letterpaper, left=1.5in, bottom=1in}                   
%\geometry{landscape}                % Activate for for rotated page geometry
\usepackage[parfill]{parskip}    % Activate to begin paragraphs with an empty line rather than an indent
\usepackage{amssymb, amsmath}
\usepackage{mathtools}
	\everymath{\displaystyle}

\usepackage{fontspec}
\setmainfont[Ligatures={TeX}, BoldFont={* Bold}, ItalicFont={* Italic}, BoldItalicFont={* BoldItalic}, Numbers={OldStyle}]{Linux Libertine O}
\setsansfont[Scale=MatchLowercase,Ligatures=TeX]{Linux Biolinum O}
\setmonofont[Scale=MatchLowercase]{Inconsolatazi4}
\usepackage{microtype}


% To define fonts for particular uses within a document. For example, 
% This sets the Libertine font to use tabular number format for tables.
 %\newfontfamily{\tablenumbers}[Numbers={Monospaced}]{Linux Libertine O}
% \newfontfamily{\libertinedisplay}{Linux Libertine Display O}

\usepackage{booktabs}
\usepackage{multicol}
\usepackage[normalem]{ulem}

\usepackage{longtable}
%\usepackage{siunitx}
\usepackage{array}
\newcolumntype{L}[1]{>{\raggedright\let\newline\\\arraybackslash\hspace{0pt}}p{#1}}
\newcolumntype{C}[1]{>{\centering\let\newline\\\arraybackslash\hspace{0pt}}p{#1}}
\newcolumntype{R}[1]{>{\raggedleft\let\newline\\\arraybackslash\hspace{0pt}}p{#1}}

\usepackage{enumitem}
\setlist{leftmargin=*}
\setlist[1]{labelindent=\parindent}

\usepackage{hyperref}
%\usepackage{placeins} %PRovides \FloatBarrier to flush all floats before a certain point.
\usepackage{hanging}

\usepackage[sc]{titlesec}

%% Commands for Exam class
\renewcommand{\solutiontitle}{\noindent}
\unframedsolutions
\SolutionEmphasis{\bfseries}

\renewcommand{\questionshook}{%
	\setlength{\leftmargin}{-\leftskip}%
}

%Change \half command from 1/2 to .5
\renewcommand*\half{.5}

\pagestyle{headandfoot}
\firstpageheader{\textsc{bi}\,063 Evolution and Ecology}{}{\ifprintanswers\textbf{KEY}\else Name: \enspace \makebox[2.5in]{\hrulefill}\fi}
\runningheader{}{}{\footnotesize{pg. \thepage}}
\footer{}{}{}
\runningheadrule

\newcommand*\AnswerBox[2]{%
    \parbox[t][#1]{0.92\textwidth}{%
    \begin{solution}#2\end{solution}}
%    \vspace*{\stretch{1}}
}

\newenvironment{AnswerPage}[1]
    {\begin{minipage}[t][#1]{0.92\textwidth}%
    \begin{solution}}
    {\end{solution}\end{minipage}
    \vspace*{\stretch{1}}}

\newlength{\basespace}
\setlength{\basespace}{5\baselineskip}


%\usepackage{mdframed}
%\mdfsetup{%
%	innerleftmargin=0pt,%
%	innerrightmargin=0pt,
%	innertopmargin=0pt,
%	innerbottommargin=0pt,
%	hidealllines=true
%}%end mdfsetup

%
%\makeatletter
%\def\SetTotalwidth{\advance\linewidth by \@totalleftmargin
%\@totalleftmargin=0pt}
%\makeatother


%\printanswers


\begin{document}

\subsection*{Journal 1: the evidence so far}

\emph{Read this entire handout throughly. You are responsible for meeting all requirements of this assignment, as detailed below.}

This is your first journal assignment, which is a formal writing assignment. 
This journal entry will require
some writing, as you need to explain the effect of each anatomical and
embryological comparison on your hypothesis. It isn't really that hard,
though. Look at your phylogenetic hypothesis to see which organisms you
predicted are related (or not). Your answer to this assignment should
take \emph{at least} two typed pages (double space, please!), and
probably a little more, depending on your hypothesis.

The evidence you have examined to test your hypothesis is listed below. Also listed is the component point value for this assignment.

\begin{enumerate}

	\item Homology of the front legs between wasps and praying mantis (5 points).
	
	\item Homology of leaves among plants (5 points).

	\item Homology of vertebrate skeletal features (10 points):

	\emph{Manus} of the bat and macaque (identified by most
class members as a homology). You should also consider the structure and
function of the manus of the other organisms in your hypothesis. Think
about the manus of other organisms viewed in class.

	\emph{Phalanges} on the manus of a macaque, an
alligator, and a salamander. You compared the number of phalanges in the digits of these organisms
to several other organisms on your hypothesis. These were identified by
most of the class members as homologous structures.

	\emph{Clavicle} of the cat and macaque, identified by most class members
as homology.

\emph{Radius and ulna} in the pigeon, human,
and bison, and other organisms (identified by most class members as homology).

	\item Embryological and larval features (10 points):
	
\emph{Tail, limb buds and pharyngeal arches} in vertebrates.

\emph{Trochophore larvae} in marine worms and land snails.

\end{enumerate}


\subsubsection*{Assignment}

Before you begin, review all of your
homology/analogy assignments, and all of the anatomical assignments and
embryo assignments. Compare the evidence from the lab exercises to the predictions made by your phylogenetic trees. Does the evidence support your
hypothesis? Falsify your hypothesis? Give no conclusive evidence? Does the evidence
require you to change your hypothesis in any way? Refer to the Scientific Method
Overview for how to use evidence to test hypotheses.

Some parts of your hypothesis may be supported and
some parts may be falsified by the anatomical and embryological evidence.
You must discuss both supported and unsupported parts of your
hypothesis. For supported parts, you must discuss which parts of your
hypothesis are supported and how the evidence supports your hypothesis.
For falsified parts of your hypothesis, you must explain which parts are
not supported, how the evidence falsifies your hypothesis, and then
explain how your hypothesis must be revised to fit the evidence examined
so far. If we did not consider evidence for some organisms, you do not
need to include those organisms in your discussion.

\emph{Clarity of your written thoughts is critical for your journal entry. Writing that is not clear is evidence of
 thinking that is not clear and will be evaluated accordingly.}

\subsubsection*{Grading of Journal 1}

\begin{enumerate}

\item Evaluation of the evidence (30 points)

Each type of evidence that you evaluated in class has been given a point
value (see the evidence above). The points will be assigned based on
inclusion and thorough discussion of the evidence, and how the evidence
affected your hypothesis. Did the evidence support the predictions made
by your hypothesis? Did the evidence falsify your hypothesis? If the
evidence falsified your hypothesis, then you must state how you revised
your hypothesis to agree with the evidence. Clarity counts towards the
total points in each category.

\item Spelling, grammar, and mechanics (5 points)

Spelling, grammar and
mechanics (sentence structure, etc) are important in any writing but especially informal writing. Use spell check. Proof your assignment carefully before you turn it in. You are entitled to \emph{one free
mistake.} After that, each mistake is a 1 point deduction.

\emph{Remember}: This course is part of the University Studies
program. The University Studies program objectives that are especially
relevant to this course are 1) the demonstration of your ability for
critical thinking, reasoning and analyzing, and 2) the demonstration of
effective written communication. Keep this in mind as you write. Points will be deducted for signs of
weak reasoning and analysis and poorly written communication.

\item New phylogenetic tree (20 points)

You must also submit a revised phylogenetic tree with this assignment. 
Your new phylogenetic tree must be consistent with every piece of
evidence evaluated and determined to be a homology. It must also be consist with what you write
for your journal entry assignment above. You will lose 3
points for each missed homology. You will also lose 3 points for each missing organism,
a missing or incomplete time scale, or any other common errors. \emph{See the first phylogenetic tree exercise for things you should not do.}

\end{enumerate}


\subsubsection*{Due Date} 

Your journal entry and new phylogenetic tree are due at the start of lab next week. Late submissions will not be accepted without a valid excuse. If necessary, review the syllabus for lab policy.

Please do not type a cover page. Just put your name and \textsc{bi} 063 at the
top, and then start typing your journal assignment.

\end{document}  