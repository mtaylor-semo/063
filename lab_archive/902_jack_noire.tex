%!TEX TS-program = lualatex
%!TEX encoding = UTF-8 Unicode

\documentclass[12pt, addpoints]{exam}
\usepackage{graphicx}
	\graphicspath{{/Users/goby/Pictures/teach/163/lab/}
	{img/}} % set of paths to search for images

\usepackage{geometry}
\geometry{letterpaper, left=1.5in, bottom=1in}                   
%\geometry{landscape}                % Activate for for rotated page geometry
%\usepackage[parfill]{parskip}    % Activate to begin paragraphs with an empty line rather than an indent
\usepackage{amssymb, amsmath}
\usepackage{mathtools}
	\everymath{\displaystyle}

\usepackage{fontspec}
\setmainfont[Ligatures={TeX}, BoldFont={* Bold}, ItalicFont={* Italic}, BoldItalicFont={* BoldItalic}, Numbers={Proportional}]{Linux Libertine O}
\setsansfont[Scale=MatchLowercase,Ligatures=TeX]{Linux Biolinum O}
%\setmonofont[Scale=MatchLowercase]{Inconsolatazi4}
\usepackage{microtype}

\usepackage{unicode-math}
\setmathfont[Scale=MatchLowercase]{Asana Math}
%\setmathfont[Scale=MatchLowercase]{XITS Math}

% To define fonts for particular uses within a document. For example, 
% This sets the Libertine font to use tabular number format for tables.
\newfontfamily{\tablenumbers}[Numbers={Monospaced}]{Linux Libertine O}
\newfontfamily{\libertinedisplay}{Linux Libertine Display O}

\usepackage{booktabs}
\usepackage{multicol}
\usepackage[normalem]{ulem}

%\usepackage{tabularx}
\usepackage{longtable}
%\usepackage{siunitx}
\usepackage{array}
\newcolumntype{L}[1]{>{\raggedright\let\newline\\\arraybackslash\hspace{0pt}}p{#1}}
\newcolumntype{C}[1]{>{\centering\let\newline\\\arraybackslash\hspace{0pt}}p{#1}}
\newcolumntype{R}[1]{>{\raggedleft\let\newline\\\arraybackslash\hspace{0pt}}p{#1}}

\usepackage{enumitem}
\usepackage{hyperref}
%\usepackage{placeins} %PRovides \FloatBarrier to flush all floats before a certain point.
\usepackage{hanging}

\usepackage[sc]{titlesec}


\renewcommand{\solutiontitle}{\noindent}
\unframedsolutions
\SolutionEmphasis{\bfseries}

%Change \half command from 1/2 to .5
\renewcommand*\half{.5}


\makeatletter
\def\SetTotalwidth{\advance\linewidth by \@totalleftmargin
\@totalleftmargin=0pt}
\makeatother


\pagestyle{headandfoot}
\firstpageheader{BI 063: Evolution and Ecology}{}{\ifprintanswers\textbf{KEY}\else Name: \enspace \makebox[2.5in]{\hrulefill}\fi}
\runningheader{}{}{\footnotesize{pg. \thepage}}
\footer{}{}{}
\runningheadrule

\newcommand*\AnswerBox[2]{%
    \parbox[t][#1]{0.92\textwidth}{%
    \begin{solution}#2\end{solution}}
%    \vspace*{\stretch{1}}
}

\newenvironment{AnswerPage}[1]
    {\begin{minipage}[t][#1]{0.92\textwidth}%
    \begin{solution}}
    {\end{solution}\end{minipage}
    \vspace*{\stretch{1}}}

\newlength{\basespace}
\setlength{\basespace}{5\baselineskip}

%\printanswers

\begin{document}

\subsection*{Jack Noire and the law of nature (\numpoints\ points)}

The universe is a deck of cards, and I am Mother Nature. I have
stacked the deck according to a natural law of my own devising. You are
a scientist; your goal is to discover the natural law that describes the
order of the cards in the deck.

I will reveal one card at a time to you. As each card is revealed,
you evaluate your current hypotheses, and write new hypotheses to fit
the new evidence. Only normal cards appear (no green ones, no jokers, etc.) 
but more than one deck could be used (so there could be multiple copies of a
given card).

\vspace*{\baselineskip}

\noindent\textsc{Consider First:}

\begin{itemize}
\item
  From one observation, are you prepared to guess what the law of nature
  might be?
\item
  Why or why not?
\item
  Are there many possibilities?
\end{itemize}

Identify eight potential laws (hypotheses) below, and for each potential law, state
in the appropriate column what next card (or group/class/category of
cards) would (a) support and (b) falsify it.


\begin{longtable}[l]{@{}|L{3.02in}|C{0.75in}|C{0.75in}|C{0.5in}|@{}}
\hline
 & \multicolumn{2}{@{}c|@{}}{Card(s) that would} & \\
Potential law (hypothesis) & Support & Falsify & conc. \tabularnewline
\hline
& & &\tabularnewline[0.8cm]
\hline
& & &\tabularnewline[0.8cm]
\hline
& & &\tabularnewline[0.8cm]
\hline
& & &\tabularnewline[0.8cm]
\hline
& & &\tabularnewline[0.8cm]
\hline
& & &\tabularnewline[0.8cm]
\hline
& & &\tabularnewline[0.8cm]
\hline
& & &\tabularnewline[0.8cm]
\hline
\end{longtable}

\noindent\textsc{Test 1: look at the next card and evaluate your hypotheses.} 
On the chart above, put your conclusion in the last column (conc.):\vspace*{0.33\baselineskip}

Which, if any, potential laws have been \textit{proven}? Indicate with
a P. \vspace*{0.33\baselineskip}

Which, if any, potential laws have been \textit{supported}? Indicate
with an S.\vspace*{0.33\baselineskip}

Which, if any, potential laws have been \textit{falsified}? Indicate
with an F.\vspace*{1\baselineskip}

\noindent\textsc{Copy any hypotheses that were supported into the table below}. 
Skip any hypotheses that were falsified. Hopefully, you did not say that some
hypotheses were proven. Remember that science is tentative. Your hypotheses
can be supported or falsified but never proven. If you did, change your P to
an S and include it below.\vspace*{\baselineskip}

\noindent\textsc{Have any new potential laws been suggested by the data?} If so, 
put the new hypotheses into the table below. For each potential law, state in the 
appropriate column what next card  (or group/class/category of cards) would 
(a) support and (b) falsify it.


\begin{longtable}[l]{@{}|L{3.02in}|C{0.75in}|C{0.75in}|C{0.5in}|@{}}
\hline
 & \multicolumn{2}{@{}c|@{}}{Card(s) that would} & \\
Potential law (hypothesis) & Support & Falsify & conc. \tabularnewline
\hline
& & &\tabularnewline[0.8cm]
\hline
& & &\tabularnewline[0.8cm]
\hline
& & &\tabularnewline[0.8cm]
\hline
& & &\tabularnewline[0.8cm]
\hline
& & &\tabularnewline[0.8cm]
\hline
& & &\tabularnewline[0.8cm]
\hline
& & &\tabularnewline[0.8cm]
\hline
& & &\tabularnewline[0.8cm]
\hline
\end{longtable}

\newpage

\noindent\textsc{Test 2: look at the next card and evaluate your hypotheses.} 
On the chart above, put your conclusion in the last column (conc.):\vspace*{0.33\baselineskip}

Which, if any, potential laws have been \textit{supported}? Indicate
with an S.\vspace*{0.33\baselineskip}

Which, if any, potential laws have been \textit{falsified}? Indicate
with an F.\vspace*{1\baselineskip}

\noindent\textsc{Copy any hypotheses that were supported into the table below}. 
Skip any hypotheses that were falsified. \vspace*{\baselineskip}

\noindent\textsc{Have any new potential laws been suggested by the data?} If so, 
put the new hypotheses into the table below. For each potential law, state in the 
appropriate column what next card  (or group/class/category of cards) would 
(a) support and (b) falsify it.

\begin{longtable}[l]{@{}|L{3.02in}|C{0.75in}|C{0.75in}|C{0.5in}|@{}}
\hline
 & \multicolumn{2}{@{}c|@{}}{Card(s) that would} & \\
Potential law (hypothesis) & Support & Falsify & conc. \tabularnewline
\hline
& & &\tabularnewline[0.8cm]
\hline
& & &\tabularnewline[0.8cm]
\hline
& & &\tabularnewline[0.8cm]
\hline
& & &\tabularnewline[0.8cm]
\hline
& & &\tabularnewline[0.8cm]
\hline
& & &\tabularnewline[0.8cm]
\hline
& & &\tabularnewline[0.8cm]
\hline
& & &\tabularnewline[0.8cm]
\hline
\end{longtable}

\noindent\textsc{Test 3: look at the next card and evaluate your hypotheses.} 
On the chart above, put your conclusion in the last column (conc.):\vspace*{0.33\baselineskip}

Which, if any, potential laws have been \textit{supported}? Indicate
with an S.\vspace*{0.33\baselineskip}

Which, if any, potential laws have been \textit{falsified}? Indicate
with an F.\vspace*{1\baselineskip}

\newpage

\noindent\textsc{Copy any hypotheses that were supported into the table below}. 
Skip any hypotheses that were falsified. \vspace*{\baselineskip}

\noindent\textsc{Have any new potential laws been suggested by the data?} If so, 
put the new hypotheses into the table below. For each potential law, state in the 
appropriate column what next card  (or group/class/category of cards) would 
(a) support and (b) falsify it.

\begin{longtable}[l]{@{}|L{3.02in}|C{0.75in}|C{0.75in}|C{0.5in}|@{}}
\hline
 & \multicolumn{2}{@{}c|@{}}{Card(s) that would} & \\
Potential law (hypothesis) & Support & Falsify & conc. \tabularnewline
\hline
& & &\tabularnewline[0.8cm]
\hline
& & &\tabularnewline[0.8cm]
\hline
& & &\tabularnewline[0.8cm]
\hline
& & &\tabularnewline[0.8cm]
\hline
& & &\tabularnewline[0.8cm]
\hline
& & &\tabularnewline[0.8cm]
\hline
& & &\tabularnewline[0.8cm]
\hline
& & &\tabularnewline[0.8cm]
\hline
\end{longtable}

\noindent\textsc{Test 4: look at the next card and evaluate your hypotheses.} 
On the chart above, put your conclusion in the last column (conc.):\vspace*{0.33\baselineskip}

Which, if any, potential laws have been \textit{supported}? Indicate
with an S.\vspace*{0.33\baselineskip}

Which, if any, potential laws have been \textit{falsified}? Indicate
with an F.\vspace*{1\baselineskip}

Mother Nature may run the game longer than four rounds. Continue to track whether
your hypotheses have been supported or falsified. Continue to think of new hypotheses
that fit all evidence seen so far (the exposed cards) and that predict the next card.
Mother Nature will finish this game before the end of lab.

\newpage

\noindent\textsc{Jack Noire follow-up assignment}\hfill Name: \rule{2.25in}{0.4pt}

%\vspace*{2\baselineskip}

\vspace*{1\baselineskip}

\begin{questions}

\question[10]
\noindent List five aspects of hypothesis testing and the process of science that are
illustrated by Jack Noire.

\ifprintanswers 
	\textbf{More answers than those below are possible. I emphasize these points as the game proceeds.}
	\vspace*{1\baselineskip}
\else
	\vspace*{-1.5\baselineskip}
\fi

\begin{parts}

\part \ifprintanswers\textbf{Hypotheses are supported / falsified, never proven}\fi\vspace*{5\baselineskip}

\part \ifprintanswers\textbf{New evidence can falsify a previously supported hypothesis.}\fi\vspace*{5\baselineskip}

\part \ifprintanswers\textbf{New evidence can lead to new hypotheses.}\fi\vspace*{5\baselineskip}

\part \ifprintanswers\textbf{Many hypotheses can be consistent with the same evidence.}\fi\vspace*{5\baselineskip}

\part \ifprintanswers\textbf{Hypotheses make specific predictions (of experimental outcomes).}\fi\vspace*{5\baselineskip}


\end{parts}

\end{questions}

\end{document}  