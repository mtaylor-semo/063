%!TEX TS-program = lualatex
%!TEX encoding = UTF-8 Unicode

\documentclass[12pt, addpoints, hidelinks]{exam}
\usepackage{graphicx}
	\graphicspath{{/Users/goby/Pictures/teach/163/lab/}
	{img/}} % set of paths to search for images

\usepackage{geometry}
\geometry{letterpaper, left=1.5in, bottom=1in}                   
%\geometry{landscape}                % Activate for for rotated page geometry
\usepackage[parfill]{parskip}    % Activate to begin paragraphs with an empty line rather than an indent
\usepackage{amssymb, amsmath}
\usepackage{mathtools}
	\everymath{\displaystyle}

\usepackage{fontspec}
\setmainfont[Ligatures={TeX}, BoldFont={* Bold}, ItalicFont={* Italic}, BoldItalicFont={* BoldItalic}, Numbers={OldStyle}]{Linux Libertine O}
\setsansfont[Scale=MatchLowercase,Ligatures=TeX]{Linux Biolinum O}
\setmonofont[Scale=MatchLowercase]{Inconsolatazi4}
\usepackage{microtype}


% To define fonts for particular uses within a document. For example, 
% This sets the Libertine font to use tabular number format for tables.
 %\newfontfamily{\tablenumbers}[Numbers={Monospaced}]{Linux Libertine O}
% \newfontfamily{\libertinedisplay}{Linux Libertine Display O}

\usepackage{booktabs}
\usepackage{multicol}
\usepackage[normalem]{ulem}

\usepackage{longtable}
%\usepackage{siunitx}
\usepackage{array}
\newcolumntype{L}[1]{>{\raggedright\let\newline\\\arraybackslash\hspace{0pt}}p{#1}}
\newcolumntype{C}[1]{>{\centering\let\newline\\\arraybackslash\hspace{0pt}}p{#1}}
\newcolumntype{R}[1]{>{\raggedleft\let\newline\\\arraybackslash\hspace{0pt}}p{#1}}

\usepackage{enumitem}
\usepackage{hyperref}
%\usepackage{placeins} %PRovides \FloatBarrier to flush all floats before a certain point.
\usepackage{hanging}

\usepackage[sc]{titlesec}

%% Commands for Exam class
\renewcommand{\solutiontitle}{\noindent}
\unframedsolutions
\SolutionEmphasis{\bfseries}

\renewcommand{\questionshook}{%
	\setlength{\leftmargin}{-\leftskip}%
}

%Change \half command from 1/2 to .5
\renewcommand*\half{.5}

\pagestyle{headandfoot}
\firstpageheader{\textsc{bi}\,063 Evolution and Ecology}{}{\ifprintanswers\textbf{KEY}\else Name: \enspace \makebox[2.5in]{\hrulefill}\fi}
\runningheader{}{}{\footnotesize{pg. \thepage}}
\footer{}{}{}
\runningheadrule

\newcommand*\AnswerBox[2]{%
    \parbox[t][#1]{0.92\textwidth}{%
    \begin{solution}#2\end{solution}}
%    \vspace*{\stretch{1}}
}

\newenvironment{AnswerPage}[1]
    {\begin{minipage}[t][#1]{0.92\textwidth}%
    \begin{solution}}
    {\end{solution}\end{minipage}
    \vspace*{\stretch{1}}}

\newlength{\basespace}
\setlength{\basespace}{5\baselineskip}

%% To hide and show points
\newcommand{\hidepoints}{%
	\pointsinmargin\pointformat{}
}

\newcommand{\showpoints}{%
	\nopointsinmargin\pointformat{(\thepoints)}
}

\newcommand{\bumppoints}[1]{%
	\addtocounter{numpoints}{#1}
}

%
%\makeatletter
%\def\SetTotalwidth{\advance\linewidth by \@totalleftmargin
%\@totalleftmargin=0pt}
%\makeatother


%\printanswers


\begin{document}

\subsection*{Descriptive statistics (\numpoints\ points)}

In biological research, we often test for cause and effect between
independent and dependent variables.

\begin{enumerate}[label=\textsc{\alph*.}, leftmargin=*]
\item
  \emph{Independent variable:} a manipulated variable in an experiment
  or study whose presence or degree determines a change in the dependent
  variable
\item
  \emph{Dependent variable:} an observed variable in an experiment or
  study whose change(s) are determined by the presence or degree of one
  or more independent variables.
\end{enumerate}

Dr. Noymer was interested in the effect on populations of the global flu
epidemic that happened in 1918. He hypothesized that babies, with their
weak immune systems, would die in a higher percentage than adults aged
25–34 when you compare the year with the epidemic (1918) to the year
before (1917).


\begin{questions}

\question[2]
What is the independent variable? 
\AnswerBox{2\baselineskip}{%
Age.
}

\question[2]
What is the dependent variable?
\AnswerBox{2\baselineskip}{%
Whether the babies/adults got sick or not.
}

\question[2]
Write a null hypothesis based on Noymer's conjecture.
\AnswerBox{2\baselineskip}{%
The increased percentage of babies and adults getting sick will not be different between 1917 and 1918.
}

\question[2]
Write a research (=alternative) hypothesis.
\AnswerBox{2\baselineskip}{%
That babies will show a greater increase in sickness than adults from 1917 to 1918.
}

Noymer's results, adapted from \emph{Age-specific death rates (per
100,000), Influenza \& Pneumonia, USA} (Noymer, 2007), are given in Table 1. They could not tell flu and pneumonia apart, so they were both counted together.

\begin{quote}
Table 1. U.S. deaths per 100,000 attributed to influenza and
pneumonia during 1917–1918. \end{quote}

\begin{longtable}[c]{@{}rrrR{0.75in}@{}}
\toprule
Age & 1917 & 1918 & Increase in deaths\tabularnewline
\midrule
\endhead
\textless{}1 & 2944.5 & 4540.9 & 152\%\tabularnewline
1-\/-4 & 422.7 & 1436.2 &\tabularnewline
5-\/-14 & 47.9 & 352.7 &\tabularnewline
15-24 & 78 & 1175.7 &\tabularnewline
25-34 & 117.7 & 1998 & 1707\%\tabularnewline
35-44 & 193.2 & 1097.6 &\tabularnewline
45-54 & 292.3 & 686.8 &\tabularnewline
\bottomrule
\end{longtable}

\newpage

\question[2]
What do these data say about the null hypothesis?
\AnswerBox{2\baselineskip}{%
They appear to falsify it. The percent increase was much larger for adults than babies.
}

\question[3]
Was Noymer's research hypothesis proven true? Supported? Supported
weakly? Falsified? Explain.
\AnswerBox{2\baselineskip}{%
Falsified. He predicted that more babies would get sick. Instead, more adults got sick.
}

\subsubsection*{Measures of central tendency: a key descriptive statistic}

A statistic is a one-number description of a set of data, or numbers
used as measurements or counts—lengths of arms, number of days, number
of fish in a catch—or, rarely, a number in that set.

Suppose that a biologist has used a 1–4 scale to describe the density of
ground-level vegetation in different habitats as part of a survey for
the local conservation district. 1~=~no vegetation, 2~=~a few plants
(mostly soil), 3~=~mostly plants and 4~=~no visible soil. Twenty samples
were taken randomly in each habitat.

The mean (=\,average) score of a forest habitat was 1.5, and for a meadow, the score
was 3.0. When reading the report, the administrator concludes that the
density of vegetation was two times higher in the meadow than in the
forest.

\question
 Is mean the appropriate statistic to use here to convey biologically meaningful information about forest and meadow vegetation?
 \AnswerBox{3\baselineskip}{No. The categories do not measure the actual amount of vegetation present.}
 
 \question
Should the biologist accept the administrator's conclusion? Why might it be challenged?
\AnswerBox{3\baselineskip}{No. The information does not reliably indicate the actual amount of vegetation present.}

One way to determine whether two things are really different is through various statistics. The most basic statistics include measures of \emph{central tendency.} A measure of central tendency is one number that describes most of the values in a data set. One measure of central tendency you are probably familiar with is the average, or mean. 

For this part of the exercise, you will calculate the mean foot length for all the students in class. Use a meter stick to measure the length of your left bare foot. We will collect the data for the entire class to fill in the table on the next page. We'll record the data separately for males and females.

\newpage


Place the left foot length measurements of males and females for the
class in the table below.

\begin{longtable}[c]{@{}|l|l|l|@{}}

\hline
Sample number & Female foot length (cm) & Male foot length (cm)\tabularnewline
\hline
1 & & \tabularnewline
\hline
2 & & \tabularnewline
\hline
3 & & \tabularnewline
\hline
4 & & \tabularnewline
\hline
5 & & \tabularnewline
\hline
6 & & \tabularnewline
\hline
7 & & \tabularnewline
\hline
8 & & \tabularnewline
\hline
9 & & \tabularnewline
\hline
10 & & \tabularnewline
\hline
11 & & \tabularnewline
\hline
12 & & \tabularnewline
\hline
13 & & \tabularnewline
\hline
14 & & \tabularnewline
\hline
15 & & \tabularnewline
\hline
16 & & \tabularnewline
\hline
17 & & \tabularnewline
\hline
18 & & \tabularnewline
\hline
19 & & \tabularnewline
\hline
20 & & \tabularnewline
\hline
Mean $(\bar{x})$ & & \tabularnewline
\hline
\end{longtable}

\question[4]\label{mean_foot}
Calculate the mean foot length for each sex (2 points each. \textsc{Hint}:
total the values and divide by the number of values)


\begin{multicols}{2}

In science we are often interested in the \emph{distribution} of our
data. A distribution is a graph where we put the values on the X axis,
and the frequency of the values on the Y axis. You are probably familiar
with a ``bell curve'' or normal distribution, such as the generic one
shown to the right.

\columnbreak

\includegraphics[width=0.45\textwidth]{09_normal_distribution}

\end{multicols}

Three methods are commonly used to describe the central tendency of a data set. 

\begin{enumerate}[label=\textsc{\alph*.}, leftmargin=*]

	\item \emph{Mean:} Arithmetic mean or the
arithmetic average. Add all the scores together and divide by the number
of scores.

	\item \emph{Median:} In an ordered set of numbers, the number in the middle:
take the number of scores and divide by 2, count down the distribution
to find the median. The score at the 50th percentile.

	\item \emph{Mode:} the most frequent score in the set of
scores.

\end{enumerate}

Each measure has specific uses but, as a rule of thumb, the mean is used when distribution is symmetric (like the normal distribution above), with few outliers. The mean is sensitive to each number in the distribution, so is greatly affected by outliers.  The median is less sensitive to outliers, but less stable than the mean because slight variations in scores in the data set can change the median quite a bit. The mode is the least stable measure of central tendency.

Using the foot data you measured earlier, answer the following.

\vspace*{0.5\baselineskip}

\question
What are the medians for males \rule{1in}{0.4pt} and
females \rule{1in}{0.4pt}?

\vspace*{0.5\baselineskip}

\question
What are the modes for males \rule{1in}{0.4pt} and
females \rule{1in}{0.4pt}?

\vspace*{0.5\baselineskip}

\question
What were the means for males \rule{1in}{0.4pt} and
females \rule{1in}{0.4pt} that you calculated for question~\ref{mean_foot}?



\question
When the mean, median and mode are all the same value (or nearly
so) we say we have a normal distribution. Based on your calculations above, were the foot length data normally distributed? (clearly not,
sort of, pretty good, spot on)

\AnswerBox{1\baselineskip}{Whatever}

\subsubsection*{Calculating standard deviation}

The \emph{standard deviation} $(s)$tells how far on average any \emph{one} measurement
is from the mean of \emph{all} measurements in the data. The smaller the standard deviation, the closer most
measurements are on average to the mean. When the standard deviation is large,
more of the measurements are more widely spread out on average from the mean. That is, a data set with a small standard deviation is less variable than a data with a large standard deviation.

\subsubsection*{Practice Problem}

Your instructor will give you a subset of the foot data. Calculate the mean and standard deviation of the
subsample foot data by hand. Fill in the table below to record the steps.

\newpage

%\vspace*{\baselineskip}

\textsc{Subsampled foot data:}

\question
Mean $(\bar{x})$: \rule{1in}{0.4pt} \hspace*{0.5in} Sample size (\emph{n}): \rule{1in}{0.4pt}

\begin{tabular}{@{}|L{1in}|L{2in}|L{2in}|@{}}
\hline
Foot length ($x$)	& Difference from the mean $(x-\bar{x})$	& Squared difference from the mean $(x-\bar{x})^2$ \tabularnewline
\hline
 	&		& \tabularnewline[1em]
\hline
	&		& \tabularnewline[1em]
\hline
	&		& \tabularnewline[1em]
\hline
	&		& \tabularnewline[1em]
\hline
	&		& \tabularnewline[1em]
\hline
	&		& \tabularnewline[1em]
\hline
	&		& \tabularnewline[1em]
\hline
	&		& \tabularnewline[1em]
\hline
	&		& \tabularnewline[1em]
\hline
	&		& \tabularnewline[1em]
\hline
Sum the values in the third column.	&	Sum of squares: $\Sigma{(x-\bar{x})^2}$ & \tabularnewline[1em]
\hline
Divide the sum by the sample size minus 1	&	Variance: $s^2 = \frac{\Sigma{(x-\bar{x})^2}}{n-1}$ & \tabularnewline[1em]
\hline
Take the square root.	&	Standard deviation: $s = \sqrt{\frac{\Sigma{(x-\bar{x})^2}}{n-1}}$ & \tabularnewline[1em]
\hline
\end{tabular}

\vspace*{\baselineskip}

\subsubsection*{Not a practice problem}

\question[8]
Repeat this process using all of the foot data gathered in lab today. Do the calculations on the other side of this page.

\end{questions}

\newpage

Mean $(\bar{x})$: \rule{1in}{0.4pt} 

\vspace*{\baselineskip}

Sample size (\emph{n}): \rule{1in}{0.4pt}

\vspace*{\baselineskip}

Sum of squares: \rule{1in}{0.4pt}

\vspace*{\baselineskip}

Variance $(s^2)$: \rule{1in}{0.4pt}

\vspace*{\baselineskip}

Standard deviation $(s)$: \rule{1in}{0.4pt}

\vspace*{\baselineskip}
Space for calculations:

\end{document}  