%!TEX TS-program = lualatex
%!TEX encoding = UTF-8 Unicode

\documentclass[12pt]{exam}


%\printanswers


\usepackage{graphicx}
	\graphicspath{{/Users/goby/Pictures/teach/063/}
	{img/}} % set of paths to search for images

\usepackage{geometry}
\geometry{letterpaper, left=1.5in, bottom=1in}                   
%\geometry{landscape}                % Activate for for rotated page geometry
\usepackage[parfill]{parskip}    % Activate to begin paragraphs with an empty line rather than an indent
\usepackage{amssymb, amsmath}
\usepackage{mathtools}
	\everymath{\displaystyle}

\usepackage[table]{xcolor}

\usepackage{fontspec}
\setmainfont[Ligatures={TeX}, BoldFont={* Bold}, ItalicFont={* Italic}, BoldItalicFont={* BoldItalic}, Numbers={OldStyle, Proportional}]{Linux Libertine O}
\setsansfont[Scale=MatchLowercase,Ligatures=TeX]{Linux Biolinum O}
\setmonofont[Scale=MatchLowercase]{Inconsolatazi4}
\newfontfamily{\tablenumbers}[Numbers={Monospaced,Lining}]{Linux Libertine O}
\usepackage{microtype}

\usepackage{unicode-math}
\setmathfont[Scale=MatchLowercase]{TeX Gyre Termes Math}

\usepackage{amsbsy}
%\usepackage{bm}

% To define fonts for particular uses within a document. For example, 
% This sets the Libertine font to use tabular number format for tables.
 %\newfontfamily{\tablenumbers}[Numbers={Monospaced}]{Linux Libertine O}
% \newfontfamily{\libertinedisplay}{Linux Libertine Display O}

\usepackage{multicol}
%\usepackage[normalem]{ulem}

\usepackage{longtable}
\usepackage{caption}
	\captionsetup{format=plain, justification=raggedright, singlelinecheck=off,labelsep=period,skip=3pt} % Removes colon following figure / table number.
%\usepackage{siunitx}
\usepackage{booktabs}
\usepackage{array}
\newcolumntype{L}[1]{>{\raggedright\let\newline\\\arraybackslash\hspace{0pt}}m{#1}}
\newcolumntype{C}[1]{>{\centering\let\newline\\\arraybackslash\hspace{0pt}}m{#1}}
\newcolumntype{R}[1]{>{\raggedleft\let\newline\\\arraybackslash\hspace{0pt}}m{#1}}

\usepackage{enumitem}
\setlist{leftmargin=*}
\setlist[1]{labelindent=\parindent}
\setlist[enumerate]{label=\textsc{\alph*}.}
\setlist[itemize]{label=\color{gray}\textbullet}
\usepackage{hyperref}
%\usepackage{placeins} %PRovides \FloatBarrier to flush all floats before a certain point.
%\usepackage{hanging}

\usepackage[sc]{titlesec}

\usepackage{afterpage}

%% Commands for Exam class
\renewcommand{\solutiontitle}{\noindent}
\unframedsolutions
\SolutionEmphasis{\bfseries}

\renewcommand{\questionshook}{%
	\setlength{\leftmargin}{-\leftskip}%
}

%Change \half command from 1/2 to .5
\renewcommand*\half{.5}

\pagestyle{headandfoot}
\firstpageheader{\textsc{bi}\,063 Evolution and Ecology}{}{\ifprintanswers\textbf{KEY}\fi}
\runningheader{}{}{\footnotesize{pg. \thepage}}
\footer{}{}{}
\runningheadrule

\newcommand*\AnswerBox[2]{%
    \parbox[t][#1]{0.92\textwidth}{%
    \begin{solution}#2\end{solution}}
%    \vspace*{\stretch{1}}
}

\newenvironment{AnswerPage}[1]
    {\begin{minipage}[t][#1]{0.92\textwidth}%
    \begin{solution}}
    {\end{solution}\end{minipage}
    \vspace*{\stretch{1}}}

\newlength{\basespace}
\setlength{\basespace}{5\baselineskip}

%% To hide and show points
\newcommand{\hidepoints}{%
	\pointsinmargin\pointformat{}
}

\newcommand{\showpoints}{%
	\nopointsinmargin\pointformat{(\thepoints)}
}

\newcommand{\bumppoints}[1]{%
	\addtocounter{numpoints}{#1}
}

\newcommand*\meanY{\overline{Y\kern1.67pt}\kern-1.67pt}
\newcommand*\meansubY{\overline{Y}}
%\newcommand*\meanY{\overline{Y}}
\newcommand*\ttest{\emph{t}-test}
\newcommand*\Popa{Population~\textsc{a}}
\newcommand*\Popb{Population~\textsc{b}}
\newcommand*\popa{population~\textsc{a}} %lower case
\newcommand*\popb{population~\textsc{b}} %lower case
\newcommand*\Corbicula{\textit{Corbicula}}
\newcommand*\AnswerBlank{\rule{0.75in}{0.4pt}\kern0.67pt.}
%
%\makeatletter
%\def\SetTotalwidth{\advance\linewidth by \@totalleftmargin
%\@totalleftmargin=0pt}
%\makeatother


\begin{document}


\subsubsection*{Instructions for resource partitioning lab}

Download from your lab Canvas page the lab handout and the Excel data file. The data are from a previous semester. If you cannot open the Excel file in your spreadsheet software, email your lab instructor for an alternate solution.

\begin{itemize}
\item 13: Resource parititioning
\item 13\_bombus\_data.xlsx
\end{itemize}

Answer all questions and make all graphs required in the lab handout, following modifications to specific questions listed below.

\begin{enumerate}

\item Do not go to the website listed for the data. Use the 13\_bombus\_data.xlsx file downloaded from your lab Moodle page.

\item Type your answers to \emph{all} questions into a Word document. Modifications to specific questions are detailed below.

 For questions like Question 1 where you have to fill in a table, you can make a table in Word if you know how or just group the information together for each species on a line. For example, question~1 asks you to calculate the mean and standard deviation for proboscis length for each species of \textsc{Bombus.} You could enter the results as

\textit{B.~appositus}- Mean: \#\#\#, S.D.: \#\#\#.

Substitute the actual values for the \#\#\# symbols. Enter the results for each species on a new line.

\item Make \emph{all} graphs in Excel or similar spreadsheet. Modifications to specific graphs are detailed below. 

\emph{Read the questions carefully so that you use the correct data tab for each graph.}

You must upload your final results in Excel. Do not draw your graphs by hand as indicated in the lab handout. \textbf{Important:} Find resources on the internet for instructions on making graphs if you are not sure how to make a particular type of graph, such as a scatterplot.

\end{enumerate}

\subsection*{Modifications to specific questions}

\begin{enumerate}[resume]

\item Question~2: Make a column chart for the means. Do not add error bars.

\item Question~8: Follow the instructions carefully to set up your data correctly for Question~9.

\item Question~9: Make column charts as required. It may help you to arrange your five graphs vertically like shown on page~5 of the handout to compare the size classes for the five species of \textit{Bombus.}

\item Questions~16 and 17: You can make line graphs in Excel with points. Look up on the internet how to change the line types (dashes, dots, colors). You can use different line types of colors to distinguish among the \textsc{Bombus} species.


\end{enumerate}

When you are finished, upload your Word (or similar) document and your spreadsheet with your graphs to the drop box.

\end{document}  