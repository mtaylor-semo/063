%!TEX TS-program = lualatex
%!TEX encoding = UTF-8 Unicode

\documentclass[12pt, addpoints, hidelinks]{exam}
\usepackage{graphicx}
	\graphicspath{{/Users/goby/Pictures/teach/163/lab/}
	{img/}} % set of paths to search for images

\usepackage{geometry}
\geometry{letterpaper, left=1.5in, bottom=1in}                   
%\geometry{landscape}                % Activate for for rotated page geometry
%\usepackage[parfill]{parskip}    % Activate to begin paragraphs with an empty line rather than an indent
\usepackage{amssymb, amsmath}
\usepackage{mathtools}
	\everymath{\displaystyle}

\usepackage{fontspec}
\setmainfont[Ligatures={TeX}, BoldFont={* Bold}, ItalicFont={* Italic}, BoldItalicFont={* BoldItalic}, Numbers={OldStyle}]{Linux Libertine O}
\setsansfont[Scale=MatchLowercase,Ligatures=TeX]{Linux Biolinum O}
%\setmonofont[Scale=MatchLowercase]{Inconsolatazi4}
\usepackage{microtype}

\usepackage{unicode-math}
\setmathfont[Scale=MatchLowercase]{Asana Math}
%\setmathfont[Scale=MatchLowercase]{XITS Math}

% To define fonts for particular uses within a document. For example, 
% This sets the Libertine font to use tabular number format for tables.
\newfontfamily{\tablenumbers}[Numbers={Monospaced}]{Linux Libertine O}
\newfontfamily{\libertinedisplay}{Linux Libertine Display O}

\usepackage{booktabs}
\usepackage{multicol}
\usepackage[normalem]{ulem}

%\usepackage{tabularx}
\usepackage{longtable}
%\usepackage{siunitx}
\usepackage{array}
\newcolumntype{L}[1]{>{\raggedright\let\newline\\\arraybackslash\hspace{0pt}}p{#1}}
\newcolumntype{C}[1]{>{\centering\let\newline\\\arraybackslash\hspace{0pt}}p{#1}}
\newcolumntype{R}[1]{>{\raggedleft\let\newline\\\arraybackslash\hspace{0pt}}p{#1}}

\usepackage{enumitem}
\usepackage{hyperref}
%\usepackage{placeins} %PRovides \FloatBarrier to flush all floats before a certain point.
\usepackage{hanging}

\usepackage[sc]{titlesec}


\renewcommand{\solutiontitle}{\noindent}
\unframedsolutions
\SolutionEmphasis{\bfseries}

%Change \half command from 1/2 to .5
\renewcommand*\half{.5}


\makeatletter
\def\SetTotalwidth{\advance\linewidth by \@totalleftmargin
\@totalleftmargin=0pt}
\makeatother


\pagestyle{headandfoot}
\firstpageheader{\textsc{bi} 063: Evolution and Ecology}{}{\ifprintanswers\textbf{KEY}\else Name: \enspace \makebox[2.5in]{\hrulefill}\fi}
\runningheader{}{}{\footnotesize{pg. \thepage}}
\footer{}{}{}
\runningheadrule

\newcommand*\AnswerBox[2]{%
    \parbox[t][#1]{0.92\textwidth}{%
    \begin{solution}#2\end{solution}}
%    \vspace*{\stretch{1}}
}

\newenvironment{AnswerPage}[1]
    {\begin{minipage}[t][#1]{0.92\textwidth}%
    \begin{solution}}
    {\end{solution}\end{minipage}
    \vspace*{\stretch{1}}}

\newlength{\basespace}
\setlength{\basespace}{5\baselineskip}

%\printanswers

\begin{document}

\subsection*{Phylogenetic trees as hypotheses (\numpoints\ points)}

You are about to receive an assignment to make your first hypothesis
about the relationships of different organisms. An efficient way to 
represent your hypothesis is by a “tree” diagram,
called a phylogenetic tree. \emph{Phylogenetic trees are hypotheses 
about evolutionary relationships} that are compact and easy to
interpret. You will use phylogenetic trees to represent your own
hypothesis about the relationships (if any) among 21 different
species. This exercise will prepare you to interpret and construct a
tree diagram of your own.

Consider the three phylogenetic trees below. According to the hypothesis, life originated
in 1930; the first organisms were a plant-like ancestor, and an unspecified
ancestor of ducks, geese, whales, and fish. Note that, according to the hypothesis, 
the ancestor of cows, humans, and dogs didn't appear until after 1940, and that humans
and dogs had a common ancestor as late as 1990 or so.

\vspace*{1\baselineskip}

\noindent\includegraphics[width=\textwidth]{03c_intro_trees}

\begin{questions}

\question[2]
Define these two terms in your own words:
\begin{parts}
	\part phylogeny:

	\AnswerBox{4\baselineskip}{A hypothesis about the relationships and ancestry of organisms.}

	\part common ancestor (what meaning of the word “common” is being used?):

	\AnswerBox{3\baselineskip}{An ancestor that is shared between two or more organisms. The organisms descend from that shared ancestor. Common refers to shared.}
	
\end{parts}

\newpage

\question[2]
When did ducks and geese have their last common ancestor according to
this hypothesis? Ducks and whales? Ferns and grass? Explain how you
know.

\AnswerBox{5\baselineskip}{Ducks and geese: about 1990. Ducks and whales: about 1980. Ferns and grass: 1960. Those times are when the ancestor splits into the two descendants.}
%\vspace*{\stretch{1}}

\question[2]
What kinds of animals existed in 1985 according to this hypothesis?
In 1945?

\AnswerBox{5\baselineskip}{1985: Ancestor of trees/grass, ferns; ancestor of ducks/geese; whales, fish, cows, ancestor of humans/dogs. 1945: plant-like ancestor, ancestor of ducks/geese/whales/fish. Maybe nothing for cows/humans/dogs. Allow some leeway for them lining a date across the page.}

\question[2]
Which should be more basically similar according to this hypothesis:
geese and whales, or dogs and whales? Which actually are more similar?
Explain.

\AnswerBox{5\baselineskip}{The hypothesis predicts that whales and geese should be more similar because of common ancestor. Whales and dogs are actually more similar because of shared mammalian characters.}

\question[2]
What does the vertical axis represent? Be careful.

\AnswerBox{5\baselineskip}{Time in years (or other time based scale. It does \emph{not} represent generations.}

\question[2]
Can you think of any evidence that you personally have witnessed that
would falsify the hypothesis represented by this phylogenetic tree? (Any
portion of it, including dates?) Explain.

\AnswerBox{5\baselineskip}{Open ended. Photos of grandparents or plants or animals older than 1940, etc. Students older than 20. Teachers older than 20. }

\end{questions}

\end{document}  