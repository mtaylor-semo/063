%!TEX TS-program = lualatex
%!TEX encoding = UTF-8 Unicode

\documentclass[12pt, hidelinks]{exam}

%\printanswers

\usepackage{graphicx}
	\graphicspath{{/Users/goby/Pictures/teach/163/lab/}
	{img/}} % set of paths to search for images

\usepackage{geometry}
\geometry{letterpaper, left=1.5in, bottom=1in}                   
%\geometry{landscape}                % Activate for for rotated page geometry
\usepackage[parfill]{parskip}    % Activate to begin paragraphs with an empty line rather than an indent
\usepackage{amssymb, amsmath}
\usepackage{mathtools}
	\everymath{\displaystyle}

\usepackage{fontspec}
\setmainfont[Ligatures={TeX}, BoldFont={* Bold}, ItalicFont={* Italic}, BoldItalicFont={* BoldItalic}, Numbers={Proportional, OldStyle}]{Linux Libertine O}
\setsansfont[Scale=MatchLowercase,Ligatures=TeX, Numbers={Proportional,OldStyle}]{Linux Biolinum O}
\setmonofont[Scale=MatchLowercase]{Linux Libertine Mono O}
\newfontfamily{\liningnum}[Numbers=Lining]{Linux Libertine O}
\usepackage{microtype}%

\usepackage[table]{xcolor}

\usepackage[bold-style=ISO]{unicode-math}
\setmathfont[Scale=MatchLowercase]{Tex Gyre Pagella Math}


\usepackage{pifont} % Use the x mark in the final section.

\usepackage{booktabs}
\usepackage{multicol}


\usepackage{caption}
\captionsetup{format=plain, justification=raggedright, singlelinecheck=off,labelsep=period,skip=3pt} % Removes colon following figure / table number.

%\usepackage{caption}
%\captionsetup{font=small} 
%\captionsetup{singlelinecheck=false}
%\captionsetup[figure]{labelsep=period, format=plain}

\usepackage{longtable}
\usepackage{caption}
\captionsetup{format=plain, justification=raggedright, singlelinecheck=off,labelsep=period,skip=3pt} 

\usepackage{array}
\newcolumntype{L}[1]{>{\raggedright\let\newline\\\arraybackslash\hspace{0pt}}p{#1}}
\newcolumntype{C}[1]{>{\centering\let\newline\\\arraybackslash\hspace{0pt}}p{#1}}
\newcolumntype{R}[1]{>{\raggedleft\let\newline\\\arraybackslash\hspace{0pt}}p{#1}}

\usepackage{enumitem}
\setlist{leftmargin=*}
\setlist[1]{labelindent=\parindent}
\setlist[enumerate]{label=\textsc{\alph*}.}
\setlist[itemize]{label=\color{white}\textbullet}

\usepackage{hyperref}
%\usepackage{placeins} %PRovides \FloatBarrier to flush all floats before a certain point.
\usepackage{hanging}

\usepackage[sc]{titlesec}

%% Commands for Exam class
\renewcommand{\solutiontitle}{\noindent}
\unframedsolutions
\SolutionEmphasis{\bfseries}

\renewcommand{\questionshook}{%
	\setlength{\leftmargin}{-\leftskip}%
}

%Change \half command from 1/2 to .5
\renewcommand*\half{.5}

\pagestyle{headandfoot}
\firstpageheader{\textsc{bi}\,063 Evolution and Ecology}{}{\ifprintanswers\textbf{KEY}\else Name: \enspace \makebox[2.5in]{\hrulefill}\fi}
\runningheader{}{}{\footnotesize{pg. \thepage}}
\footer{}{}{}
\runningheadrule

\newcommand*\AnswerBox[2]{%
    \parbox[t][#1]{0.92\textwidth}{%
    \begin{solution}#2\end{solution}}
    \vspace{\stretch{1}}
}

\newenvironment{AnswerPage}[1]
    {\begin{minipage}[t][#1]{0.92\textwidth}%
    \begin{solution}}
    {\end{solution}\end{minipage}
    \vspace{\stretch{1}}}

\newlength{\basespace}
\setlength{\basespace}{5\baselineskip}


\newcommand\chisq{$\chi^2$}
\newcommand*\meanY{\overline{Y}\kern0.67pt}

\newcommand*\AnswerBlank[1]{%
	\ifprintanswers%
		\textbf{#1}
	\else%
		\rule{0.75in}{0.4pt}\kern0.67pt.\fi%
	}

%\newcommand*\AnswerBlank{\rule{0.75in}{0.4pt}\kern0.67pt.}
\newcommand*\xcell[1]{cell~\liningnum{#1}}

%
%\makeatletter
%\def\SetTotalwidth{\advance\linewidth by \@totalleftmargin
%\@totalleftmargin=0pt}
%\makeatother



\begin{document}

\subsection*{Mark and recapture: estimating population size}

Ecologists are often interested in estimating the total number of
individuals in a population and whether that number is changing over
time. If a population is sedentary (does not move very much), then
population size can be estimated by quadrat
or other sampling techniques that allow direct counting of individuals. 

Population sizes of mobile organisms, however,
cannot always be estimated that way. Instead, a frequently used method 
for sampling populations of mobile individuals is the
\textbf{mark-recapture} technique. In its simplest form, this method consists of
capturing some individuals, marking them in some way, releasing them
back into the population, and, after an appropriate time interval,
resampling the population. The number of individuals in the original
population can be estimated with the formula,

\begin{equation} \label{eq:mark_recapture}
N_1 = \dfrac{M_1 n_2}{M_{12}},
\end{equation}

where,

\begin{itemize}
	\item $N_1$ is the estimated population size,
	\item $M_1$ is the number of marked individuals released into the
	population from the first capture,
	\item $n_2$ is the number of
	individuals taken in the second sample, and
	\item $M_{12}$ is the
	number of individuals in the second sample that were marked during the
	first sample.
\end{itemize}

Other terms you will need later in this lab include,

\begin{itemize}\label{eq:terms}
	\item $N_2$ is the estimate of population size at the second
census,

	\item $M_2$ is the number of individuals marked and released
during the second census,

	\item $n_3$ is the number of individuals captured in the third
census, and

	\item $M_{23}$ is the number of individuals marked during the second
census that are recaptured during the third census.
\end{itemize}

The accuracy of this technique depends upon some
rather strict assumptions:

\begin{itemize}[label=\color{black}\textbullet]

\item
  An individual caught and tagged in the first capture must be neither
  more nor less likely than other members of the population to be caught
  on the second capture. This means that tagged individuals must not
  have a higher mortality rate, tagged individuals mingle freely with
  other members of the population when released, sufficient time has
  elapsed between captures to allow dispersal of tagged individuals, and
  tagged individuals must not become ``trap-happy'' or ``trap-shy''.

\item
  Tags do not become lost or unrecognizable.

\item
  There is no emigration or death of tagged individuals. Emigration and
  death rates for unmarked members of the population must be equal to
  immigration and birth rates.
  
  \item An equal time  of time has elapsed between each sample period.

\end{itemize}

This exercise will help you learn some properties of the mark-recapture method.

\subsection*{Methodology}

\subsubsection*{First capture-mark-release}\label{sec:first_mark}

Each pair of students has a box that contains a population of spiders. %, with a few cockroaches mixed in. 
Yes, \textsc{ok}, they are beans. Pretend. %Spiders are the larger white beans. Cockroaches are the speckled beans.

\begin{questions}

\question
How many spiders do you think are present in the box? That is, what is your initial \emph{estimate} of the population size ($N$)?

\bigskip

Estimated $N\colon$ \rule{0.75in}{0.4pt}

You will now estimate population size using the following mark and recapture technique. \textbf{Follow each step exactly.}

\begin{enumerate}
	\item Elect a sampler for your group who will do all sampling for 
	this exercise. Choose your spider sampler wisely. The better
	the sampler, the better your data!
	
	\item Designate another person to be the timer. Use a watch or smart phone to count seconds.
	
	\item The sampler will capture all the spiders s/he can within a
	\textbf{45 second period.} 
	
	\item Sample only \emph{one spider at a time.}  \textbf{Do not look into the box when you sample to avoid sample bias.} Each spider must be
	placed on the table before the next spider can be captured. %Cockroaches  must be returned to the box before he next spider can be captured.
	
	\item After the sample time has ended, use the black marking pen to mark each sampled spider with a small, distinct dot \emph{on one side only.}
	
	\item Record the number of individuals as $M_1$.
	
	\bigskip
	
	$M_1\colon$ \rule{0.75in}{0.4pt}
	
	\item Mix the marked individuals back in with the remaining 
spiders in the box. Mix well! Make sure the individuals are evenly distributed.

\end{enumerate}

\subsubsection*{Second capture-mark-release}\label{sec:second_mark}

\begin{enumerate}

	\item Repeat the sampling process for another 45 seconds.  \textbf{Do not look into the box when you sample to avoid sample bias.} As before, sampled spiders must be placed on the table before capturing the next individual. %Cockroaches must be returned to the box.
	
	\item Record the total number of spiders captured ($n_2$) and the
	number of marked spiders recaptured ($M_{12}$). 
	
	\bigskip
	
	$n_2\colon$ \rule{0.75in}{0.4pt} \bigskip
	
	$M_{12}\colon$ \rule{0.75in}{0.4pt}
	
	\item Mark all the spiders taken in this second sample with a small dot. If the individual was sampled earlier, place the dot \emph{on the same side} as the first dot. Record the 	number of spiders marked this round as $M_2$. (This value will
	be the same as $n_2$). Add the marked spiders back to the box and mix them well.
	
	\bigskip
	
	$M_2\colon$ \rule{0.75in}{0.4pt}
	
	You will need this value again below.
	
\end{enumerate}

\question
Calculate the population size ($N_1$) of
spiders \emph{after the second capture}, with equation~\ref{eq:mark_recapture},

\[ N_1 = \dfrac{M_1 n_2}{M_{12}}\]

\medskip

$N_1\colon$ \rule{0.75in}{0.4pt}

\question
Ask your instructor for the actual population size for your box. Was your estimate close (±20 individuals) to the actual population size, way off (±40 individuals), or somewhere between? Why do you think you did not get closer to the actual number?

\AnswerBox{2\baselineskip}{Answer here.}

\subsubsection*{Estimating change of population size}

Changes in population size can sometimes be estimated with a third round of capture-mark-release, as long as the assumptions described above are met, such as equal time intervals between samples. In this case, you calculate
$\Delta P$, \emph{the change in population size per individual} (proportional change in
population size). You calculate the relative change in
population sizes as,

\begin{equation}
	\Delta P = \dfrac{\mathrm{Change~in~population~size}}{\mathrm{Population~size~from~first~census}} = \dfrac{N_2 - N_1}{N_1},
\end{equation}

where, from equation~(\ref{eq:mark_recapture}), 

\begin{equation*}
N_1 = \dfrac{M_1 n_2}{M_{12}},
\end{equation*}

and

\begin{equation}
N_2 = \dfrac{M_2 n_3}{M_{23}}.
\end{equation}

Refer to page~\pageref{eq:terms} for the definition of each term.  

\subsubsection*{Third capture-mark-release}\label{sec:third-mark}

\begin{enumerate}

	\item Ask your instructor to change your population size. Mix spiders \emph{thoroughly}
after the instructor has adjusted the numbers.

	\item Capture your spiders as before, recording the total number of spiders in
the third capture ($n_3$) and the number of
spiders marked in the second capture and then recaptured in the third
sample ($M_{23}$); all spiders with \emph{two marks} used in the
second capture).  

	\begin{multicols}{2}
		$M_2\colon$ \rule{0.75in}{0.4pt} (from above) \bigskip

		$n_3\colon$ \rule{0.75in}{0.4pt} 
	
		\columnbreak 
		
		$M_{23}\colon$ \rule{0.75in}{0.4pt}\bigskip
	
	\end{multicols}
	
\end{enumerate}

\question
Calculate with this formula the population size ($N_2$) of
spiders \emph{after the third capture},

\[ N_2 = \dfrac{M_2 n_3}{M_{23}}. \]


\question
Was this population size estimate ($N_2$) similar to your original
estimate ($N_1$)? If not, how do your estimates differ?

\ifprintanswers \textbf{Answers will vary} \fi \vspace*{3\baselineskip}

\question
Calculate your estimated rate of population change ($\Delta P$) with the formula below to determine
how many spiders were added or subtracted from your population by the
instructor. 

\[ \Delta P = \dfrac{N_2 - N_1}{N_1}. \]

The $\Delta P$ value you calculated will be a positive or negative decimal number. 


\question
To estimate how many spiders were added (positive number) or removed (negative number), use

Number added or removed = Original \emph{known} number of spiders in the box $\times$~$\Delta P$.

\textbf{Use the number given to you by your instructor, not your estimate.}\bigskip


\question \label{ques:spider_change}
How many spiders were added or removed? \rule{0.75in}{0.4pt}


Once you have calculated the number of spiders added or removed, tell your answer to your
instructor to determine if you were able to accurately detect the size change
in your population.

\question
Was your estimate regarding the change in population correct? Explain.

\AnswerBox{1\baselineskip}{Answers will vary}


\subsubsection*{Does sampling time affect population size estimates?}

In the first part of this lab, the spider sampler had only 45 seconds to
capture (sample) spiders from the population. To examine the effect of
sampling time on your estimate of population size, repeat the first and second 
capture-mark-release sections (pages~\pageref{sec:first_mark}–\pageref{sec:second_mark}) 
again to estimate $N_1$ (you do not need $M_{2}$).
\emph{However, this time allow the spider sampler 60 seconds to capture
spiders}. \textbf{Mark the spiders with a short line (\raisebox{2pt}{\rule{4mm}{1.5pt}}\,; not a dot)} (on the other side if already marked) so you do not mix up the marks from previous captures. Record your new $M_1$, $n_2$, and $M_{12}$ values as you go. \bigskip

	\begin{multicols}{2}
	$M_1\colon$  \rule{0.75in}{0.4pt} \bigskip
	
	$n_2\colon$ \rule{0.75in}{0.4pt}
	
	\columnbreak
	
	$M_{12}\colon$ \rule{0.75in}{0.4pt} \bigskip
	
\end{multicols}

\question
Use Equation~\ref{eq:mark_recapture} to calculate the population size ($N_1$) of
spiders after the second capture. \bigskip

$N_1\colon$ \rule{0.75in}{0.4pt}

\question
Did the accuracy of your estimate of population size improve, given more time for sampling? 

\vspace*{2\baselineskip}


\subsubsection*{Swap with another group}

The estimate of population size can be affected by many factors, including the sampling effort. To see this effect, swap your box with another group. Make sure you swap with a group whose box has a different color code.


Repeat the sampling procedure with your new set of spiders, allowing 60 seconds for capture time. \textbf{Mark the spiders with an \ding{56} (not a line or dot)} so you do not mix up the marks from previous captures. Record your new $M_1$, $n_2$, and $M_{12}$ values as you go. %\bigskip

	\begin{multicols}{2}
	$M_1\colon$  \rule{0.75in}{0.4pt} \bigskip
	
	$n_2\colon$ \rule{0.75in}{0.4pt}
	
	\columnbreak
	
	$M_{12}\colon$ \rule{0.75in}{0.4pt} \bigskip
	
\end{multicols}

\question
Use Equation~\ref{eq:mark_recapture} to calculate the population size ($N_1$) of
spiders after the second capture. \bigskip

$N_1\colon$ \rule{0.75in}{0.4pt}

\question
How did your estimate compare with that of the other group? Were your estimates similar to each other? Was the estimate of one group much closer (say, by 10 or more individuals) to the true number? Both groups together should discuss reasons for differences.

\AnswerBox{\basespace}{Answers will vary.}

\newpage

\question[Checkout]
What does this tell you about the effect of sampling time on estimating
population size using this mark and recapture technique? If your estimate this time was
similar to your original estimate, describe what you imagine the effect
of having more time to sample would be on your estimate of population
size in a field setting, in which organisms are more likely to be spread
out across a larger space.

\AnswerBox{\basespace}{More time would increase the chance of sampling marked individuals, giving you a more reliable estimate of population size.}


\question[Checkout]
If migration occurred in a natural population being studied, how
would this influence the reliability of your estimate of population size
determined using the mark-and-recapture technique?

\AnswerBox{\basespace}{Reliability would decrease. If immigration occured between samples, then the estimate would be too low. If emigration occured between samples, then the esimate would be too high. Emigration of marked individuals would also affect the estimate because not as many could be captured.}

\subsubsection*{Final steps: sort your spiders}

To receive full credit for the day, you must help reset the boxes for the next lab.

\begin{enumerate}
	\item Remove all marked spiders and place them in the container provided by your instructor. Your instructor will then provide you with additional spiders.
	
	\item Count 100 spiders, place them in the container marked with blue, and seal the lid. 
	
	\item Count 130 spiders, place them in the container marked with green, and seal the lid.
	
	\item Count 30 spiders, place them in the bag with the blue tape, and seal the bag.
	
	\item Count 39 spiders, place them in the bag with the gray tape, and seal the bag.
	
	\item Pat yourself on the back for helping!
	
\end{enumerate}
%\question[Checkout]
%You decide to test out a new type of marking technique on a lizard
%population you are interested in characterizing. While other researchers
%have used nail polish to mark individuals, you decided to use acrylic
%paint (e.g., Crayola washable kids paint) as that is what you have on
%hand. You are able to mark 20 individuals. Given the area you sampled,
%you are confident you have captured a large proportion of the population
%(as in competition will limit the number of individuals in any one location
%because they set up territories of a known size). While you intended to
%go out to recapture after a few days, the weather has prevented you from
%doing so from massive rains and flooding. So, it ultimately takes a week
%after releasing your paint-marked individuals for you to return to the
%same location to see how many you can recapture. This time you catch 25
%lizards, but none of them are marked.
%
%\medskip
%
%What are the two possible scenarios that could have led to this result?
%
%\vspace*{\basespace}
%
%\question
%Based on your study design, which scenario is most likely to explain the
%absence of marked individuals in the study area upon your return? (Which
%assumption of the model was likely violated?) Explain your reasoning.

\end{questions}

\newpage

\ifprintanswers \bfseries

\subsubsection*{Note to instructors}

Boxes with blue paint have 100 beans. Boxes with green paint have 130 beans.

Baggies with blue paper have 30 beans. Baggies with green paper have 39 beans. 

Mnemonic: Blue before green alphabetically, 100 before 130 numerically.




One pair of students at a table gets the box with 100 beans while the other pair gets the box with 130 beans. Do not tell them the number of beans until after they have estimated $\symbf{N_1}$ the first time.


When it is time to increase the population size, it will increase by 30\%.  Add all of the beans from the bag with blue paper to the boxes marked with blue paint to go from 100 to 130 individuals. Add all of the beans from the bag with the green paper to the boxes marked with green paint to go from 130 to 169 individuals.

Do not tell them the how many beans you added or the new numbers until the groups have swapped and both have estimated the new $\symbf{N_1}$.

\fi	


\end{document}  