%!TEX TS-program = lualatex
%!TEX encoding = UTF-8 Unicode

\documentclass[12pt, addpoints, hidelinks]{exam}
\usepackage{graphicx}
	\graphicspath{{/Users/goby/Pictures/teach/163/lab/}
	{img/}} % set of paths to search for images

\usepackage{geometry}
\geometry{letterpaper, left=1.5in, bottom=1in}                   
%\geometry{landscape}                % Activate for for rotated page geometry
\usepackage[parfill]{parskip}    % Activate to begin paragraphs with an empty line rather than an indent
\usepackage{amssymb, amsmath}
\usepackage{mathtools}
	\everymath{\displaystyle}

\usepackage{fontspec}
\setmainfont[Ligatures={TeX}, BoldFont={* Bold}, ItalicFont={* Italic}, BoldItalicFont={* BoldItalic}, Numbers={OldStyle}]{Linux Libertine O}
\setsansfont[Scale=MatchLowercase,Ligatures=TeX]{Linux Biolinum O}
\setmonofont[Scale=MatchLowercase]{Inconsolatazi4}
\usepackage{microtype}


% To define fonts for particular uses within a document. For example, 
% This sets the Libertine font to use tabular number format for tables.
 %\newfontfamily{\tablenumbers}[Numbers={Monospaced}]{Linux Libertine O}
% \newfontfamily{\libertinedisplay}{Linux Libertine Display O}

\usepackage{booktabs}
\usepackage{multicol}
\usepackage[normalem]{ulem}

\usepackage{longtable}
%\usepackage{siunitx}
\usepackage{array}
\newcolumntype{L}[1]{>{\raggedright\let\newline\\\arraybackslash\hspace{0pt}}p{#1}}
\newcolumntype{C}[1]{>{\centering\let\newline\\\arraybackslash\hspace{0pt}}p{#1}}
\newcolumntype{R}[1]{>{\raggedleft\let\newline\\\arraybackslash\hspace{0pt}}p{#1}}

\usepackage[hang,flushmargin]{footmisc}

\usepackage{enumitem}
\usepackage{hyperref}
%\usepackage{placeins} %PRovides \FloatBarrier to flush all floats before a certain point.
\usepackage{hanging}

\usepackage[sc]{titlesec}

%% Commands for Exam class
\renewcommand{\solutiontitle}{\noindent}
\unframedsolutions
\SolutionEmphasis{\bfseries}

\renewcommand{\questionshook}{%
	\setlength{\leftmargin}{-\leftskip}%
}

%Change \half command from 1/2 to .5
\renewcommand*\half{.5}

\pagestyle{headandfoot}
\firstpageheader{\textsc{bi}\,063 Evolution and Ecology}{}{\ifprintanswers\textbf{KEY}\else Name: \enspace \makebox[2.5in]{\hrulefill}\fi}
\runningheader{}{}{\footnotesize{pg. \thepage}}
\footer{}{}{}
\runningheadrule

\newcommand*\AnswerBox[2]{%
    \parbox[t][#1]{0.92\textwidth}{%
    \begin{solution}#2\end{solution}}
%    \vspace*{\stretch{1}}
}

\newenvironment{AnswerPage}[1]
    {\begin{minipage}[t][#1]{0.92\textwidth}%
    \begin{solution}}
    {\end{solution}\end{minipage}
    \vspace*{\stretch{1}}}

\newlength{\basespace}
\setlength{\basespace}{5\baselineskip}


%\printanswers


\begin{document}

\subsection*{Invertebrate larvae (\numpoints\ points)}

Many animals go through an embryonic type of stage called a larval stage. 
A larva (plural: larvae) is a free living developmental stage that is morphologically distinct from the adult. In
many cases, the larva is a feeding stage which later metamorphoses into
an adult. Examples of larvae you probably know about are caterpillars
which metamorphose into butterflies and tadpoles which metamorphose into
frogs. For many marine invertebrates the larval stage is also a
dispersal stage. Their larvae will float near the surface of the ocean,
drift with the currents, and feed in the plankton. In later stages they
drop out of the plankton, settle, and metamorphose into adults.

Below are pictures of larvae of several marine
invertebrates.\footnote{Images A,B,D,E from Young, CM. 2002. Atlas of Marine Invertebrate Larvae, Image C from Jackson et al. 2007. BMC Evolutionary Biology 7:160.} They have been adjusted so they are
the same size and and on the same background.

\begin{longtable}[c]{@{}lll@{}}
\toprule
\includegraphics[height=4cm]{06_invert_larva_A_feather_duster} 	&
\includegraphics[height=4cm]{06_invert_larva_B_clam}			& \tabularnewline
%
A \ifprintanswers\textbf{\large feather duster}\fi				 	&
B \ifprintanswers\textbf{\large clam}\fi						& \tabularnewline[4ex]
\midrule
\includegraphics[height=4cm]{06_invert_larva_C_marine_snail} 	&
\includegraphics[height=4cm]{06_invert_larva_D_chiton} 			&
\includegraphics[height=4cm]{06_invert_larva_E_marine_worm} 	\tabularnewline
C  \ifprintanswers\textbf{\large marine snail}\fi 						&
D \ifprintanswers\textbf{chiton}\fi 						&
E \ifprintanswers\textbf{marine worm}\fi						\tabularnewline[4ex]
\bottomrule
\end{longtable}

\begin{questions}

\question
Can you determine which is the marine snail? How about the
clam, feather duster (worm), Marine worm, and chiton?

\newpage

\subsubsection*{Trochophore larvae}

\question

All of these larvae are called trochophore larvae because
they share certain features. Can you describe at least one
prominent feature shared by all of the above larvae?

\AnswerBox{3\baselineskip}{Students usually recognize the cilia band.}

\begin{minipage}{0.75\textwidth}%
One feature you probably noted was the band of hair-like structures
going around the organism. This is known as the \emph{prototroch} (labeled pt
in the diagram at right). The hair-like structures are
called cilia and these allow the larva to swim. They work like tiny
oars.
\end{minipage}\hfill
\begin{minipage}{0.25\textwidth}%
\centering\includegraphics[width=0.9in]{06_prototroch}\\%
{\footnotesize Trochophore stage\footnotemark}%
\end{minipage}
\footnotetext{Figure from \url{http://scaa.usask.ca/gallery/lacalli/tutorial/tutorial_all.php}. retrieved: February 26, 2008}

\question[2]
Based on what you know so far, would you consider the
prototroch on these larvae an analogy or homology? Explain.

\AnswerBox{3\baselineskip}{Analogy. Cilia serves common swimming function.}


\subsubsection*{Marine and terrestrial snails}

\begin{minipage}{0.75\textwidth}%
Let's focus on marine snail development for a moment. Marine snails go 
through an extra step in their development.
They start as an egg, hatch as a trochophore larvae, metamorphose into a
\emph{veliger larvae} and finally metamorphose into an adult snail.
The veliger looks like an adult snail but has a ciliated velum.
\end{minipage}\hfill
\begin{minipage}{0.25\textwidth}%
\centering\includegraphics[width=0.75in]{06_veliger}\\
{\footnotesize Veliger stage\footnotemark}
\end{minipage}
\footnotetext{Figure from Hymen, LH. 1967 The Invertebrates Vol VI.}

\question[2]
What do you think the velum is for? (Remember it has cilia.) Explain.

\AnswerBox{4\baselineskip}{Most will velum is for swimming say for swimming.}


\question
Now let's look at a terrestrial snail (which spends its
entire life on land). \emph{Would you expect a terrestrial snail to
have a ciliated larval stage?} Why or why not? (Remember the
function of cilia).

\AnswerBox{4\baselineskip}{Most will say they would not expect land snails to have ciliated stage.}

\begin{minipage}{0.75\textwidth}%
It turns out terrestrial snails do have these two larval stages but the
trochophore and veliger larval stages \emph{stay in the egg.} The veliger stage
metamorphoses in to the adult before the snail hatches. On the left you
can see a diagram of a snail veliger larva still in its egg.
\end{minipage}\hfill
\begin{minipage}{0.25\textwidth}%
\centering\includegraphics[width=1in]{06_land_snail_egg}\\
{\footnotesize Land snail egg}
\end{minipage}

\question[2]
Does a ciliated larva serve a function in a terrestrial snail? Explain.

\AnswerBox{4\baselineskip}{No. Swimming is not necessary in an egg so cilia unncessary.}

\question[2]
What does this evidence say about trochophore larvae in
general? Is having a trochophore larvae a homology or analogy? Explain.

\AnswerBox{4\baselineskip}{Homology. Even organisms that do not swim as larvae have a trochophore stage. }


\question[2]
Now look at your hypothesis and explain how this evidence
affects your hypothesis. 

\AnswerBox{4\baselineskip}{Answer depends on hypothesis. }

\end{questions}

\end{document}  