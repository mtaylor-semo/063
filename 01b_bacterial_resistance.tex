%!TEX TS-program = lualatex
%!TEX encoding = UTF-8 Unicode

\documentclass[12pt]{exam}
\usepackage{graphicx}
	\graphicspath{{/Users/goby/Pictures/teach/163/lab/}} % set of paths to search for images

\usepackage{geometry}
\geometry{letterpaper, bottom=0.9in}                   

\usepackage{afterpage}
\usepackage{pdflscape}

\newlength{\myindent}
\setlength{\myindent}{\parindent}
\newcommand{\ind}{\hspace*{\myindent}}


%\geometry{landscape}                % Activate for for rotated page geometry
\newlength{\litindent}
\setlength{\litindent}{\parindent}

\usepackage[parfill]{parskip}    % Activate to begin paragraphs with an empty line rather than an indent
%\usepackage{amssymb, amsmath}
%\usepackage{mathtools}
%	\everymath{\displaystyle}

\usepackage{fontspec}
\setmainfont[Ligatures={TeX}, BoldFont={* Bold}, ItalicFont={* Italic}, BoldItalicFont={* BoldItalic}, Numbers={OldStyle,Proportional}]{Linux Libertine O}
\setsansfont[Scale=MatchLowercase,Ligatures=TeX, Numbers=OldStyle]{Linux Biolinum O}
\setmonofont[Scale=MatchLowercase]{Inconsolatazi4}
\usepackage{microtype}

\usepackage{unicode-math}
\setmathfont[Scale=MatchLowercase]{Asana Math}
%\setmathfont[Scale=MatchLowercase]{XITS Math}

% To define fonts for particular uses within a document. For example, 
% This sets the Libertine font to use tabular number format for tables.
\newfontfamily{\tablenumbers}[Numbers={Monospaced}]{Linux Libertine O}
\newfontfamily{\libertinedisplay}{Linux Libertine Display O}

\usepackage{longtable}

\usepackage{booktabs}
\usepackage{multirow}
\usepackage{multicol}

\usepackage[justification=raggedright, labelsep=period]{caption}
\captionsetup{singlelinecheck=off}
\captionsetup{skip=0.2em}

%\usepackage{tabularx}
%\usepackage{siunitx}
\usepackage{array}
\newcolumntype{L}[1]{>{\raggedright\let\newline\\\arraybackslash\hspace{0pt}}p{#1}}
\newcolumntype{C}[1]{>{\centering\let\newline\\\arraybackslash\hspace{0pt}}p{#1}}
\newcolumntype{R}[1]{>{\raggedleft\let\newline\\\arraybackslash\hspace{0pt}}p{#1}}

\newcolumntype{M}[1]{>{\centering\let\newline\\\arraybackslash\hspace{0pt}}m{#1}}

\usepackage{tikz}

\usepackage{enumitem}
\setlist{leftmargin=*}
\setlist[1]{labelindent=\parindent}
\setlist[enumerate]{label=\textsc{\alph*}., ref=\textsc{\alph*}}

%\usepackage{hyperref}
\usepackage{hanging}

\usepackage[sc]{titlesec}


\renewcommand{\solutiontitle}{\noindent}
\unframedsolutions
\SolutionEmphasis{\bfseries}

\renewcommand{\questionshook}{%
	\setlength{\leftmargin}{-\leftskip}%
}
%Change \half command from 1/2 to .5
%\renewcommand*\half{.5}


\makeatletter
\def\SetTotalwidth{\advance\linewidth by \@totalleftmargin
\@totalleftmargin=0pt}
\makeatother



\pagestyle{headandfoot}
\firstpageheader{BI 063: Evolution and Ecology}{}{\ifprintanswers\textbf{KEY}\else Name: \enspace \makebox[2.5in]{\hrulefill}\fi}
\runningheader{}{}{\footnotesize{pg. \thepage}}
\footer{}{}{}
\runningheadrule

\newcommand*\AnswerBox[2]{%
    \parbox[t][#1]{0.92\textwidth}{%
    \begin{solution}#2\end{solution}}
    \vspace{\stretch{1}}
}

\newenvironment{AnswerPage}[1]
    {\begin{minipage}[t][#1]{0.92\textwidth}%
    \begin{solution}}
    {\end{solution}\end{minipage}
    \vspace{\stretch{1}}}

\newlength{\basespace}
\setlength{\basespace}{5\baselineskip}

\newcommand{\allele}[1]{\textit{#1}}

%\printanswers

\begin{document}

\subsection*{Bacterial resistance}

Bacteria are frequently exposed to chemical agents, such as antibiotic drugs and antimicrobial soaps, as humans attempt to control bacterial disease and contamination. Many bacteria have evolved resistance to common agents intended to control bacteria. For example, about 94\% of the different strains of \textit{Staphylococcus aureus}, a common source of staph infections, were killed by penicillin when the drug was introduced in the early 1940s. In less than a decade, only 50\% of strains were susceptible. Within another decade, severe outbreaks of \textit{S. aureus} in hospitals were common (Livermore 2000). \textit{Staphylococcus aureus} and other bacteria evolved resistance to penicillin through natural selection. As bacteria were exposed to penicillin, many individuals were killed but not all. Surviving individuals were still able to reproduce so they had higher relative fitness. As the survivors reproduced, their offspring inherited resistance to penicillin. 

Imagine you are infected by just one cell of a harmful bacterium. Bacteria are capable of dividing (reproducing) every 20 minutes. Each division produces two cells from one parent cell. In 10 hours, the total number of bacteria from 30 divisions would be $2^{30} = 1,073,741,824,$ more than one billion cells. (No wonder illness seems to set in so quickly.) Now that you are feeling poorly, you take an antibiotic labeled as 99.9\% effective. The antibiotic might kill most cells but as many as 1,073,741 cells (0.01\% of 1,073,741,824 cells) might not killed by the antibiotic. \emph{Each} of those one million cells could go on to produce another one billion cells within 24 hours (the antibiotic can slow the growth of resistant bacteria), each with some resistance to the antibiotic. The problem is magnified because some people take antibiotics only until they feel better instead of the full time prescribed. An incomplete dose allows even more resistant bacterial cells to survive that might have been killed if the full dose had been taken. 

For this lab, you will expose a harmless soil bacterium, \textit{Bacillus cereus,} to different concentrations of Triclosan, an antimicrobial drug that was once used in hand soaps at concentrations of 0.1–0.4\%. You will test the ability of four concentrations of Triclosan (0.1\%, 0.3\%, 0.5\%, and 0.7\%) to inhibit growth of \textit{B. cereus.} You will grow the bacterium on agar plates. You will place a paper disk soaked with Triclosan in the center of the agar plate. The drug will seep into the agar, potentially inhibiting the growth of the bacteria around the disk. The inhibiting effect will decrease farther away from the disk so some bacteria will be able to survive if they are far enough away. You will measure the diameter of the zone of inhibition to determine the inhibitory effect of the drug. The Triclosan is dissolved in 50\% ethanol so you will include a control of 50\% ethanol without the drug.

\begin{questions}

\question \label{ques:hypothesis}
Write a research hypothesis. How do you think different concentrations will affect the growth of \textit{B. cereus?}

%\vspace*{3\baselineskip}
\newpage

\question
Turn your research hypothesis into a prediction. Remember that your prediction should predict the results of the experiment that would support your hypothesis. 

\vspace*{3\baselineskip}

\question \label{ques:hypothesis}
Write the corresponding null hypothesis for this experiment. 

\vspace*{3\baselineskip}

\question
Turn your null hypothesis into a prediction. That is, predict the results of the experiment that would falsify your hypothesis. 

\vspace*{3\baselineskip}

\question
Identify the explanatory and response variables. 

a. Explanatory variable: \ifprintanswers \textbf{Triclosan concentration.} \fi
	
	\vspace*{\baselineskip}
	
b. Response variable: \ifprintanswers \textbf{Size/diameter of the zone of inhibition.} \fi
 

\vspace*{0.5\baselineskip}

\subsubsection*{Materials}

Everyone at your table will work as one group. Each table has the following items. Check to be sure each item is present before you begin. If you are missing an item, notify your instructor immediately.

\begin{itemize}

	\item Five petri plates with LB agar.
	
	\item One test tube culture of \textit{Bacillus cereus.}
	
	\item Four small jars of Triclosan at concentrations of 0.1\%, 0.3\%, 0.5\% and 0.7\%, dissolved in 50\% ethanol.%, and one small jar with 50\% ethanol for the control.
	
	\item One small jar with 50\% ethanol. This is a control.

	\item Glass petri dish of paper disks.
	
	\item Forceps.
	
	\item One biohazard bag.
	
	\item Sterile swabs.
	
	\item A marker for labeling.
	
	\item Non-latex gloves (on the back counter).
	
\end{itemize}

\subsubsection*{Procedure}

	\begin{enumerate}
	
		\item Put on a pair of gloves. Decide on a catchy but easy to remember name for your group.  
			
		\item Label the \emph{bottoms} of five petri plates with your catchy group name and your lab section number. \emph{Write the information near one edge so it will not obstruct your ability to take measurements next week.} Your instructor will tell you your section number.
		
		\item Label the \emph{top} of each petri plate with either C, 0.1\%, 0.3\%, 0.5\%, or 0.7\%, for the control and each of the four Triclosan concentrations. 
		
		\item \label{swab_first} Remove one sterile swab from the wrapper. 
		
		\item Remove the cap from the test tube culture of \textit{B. cereus.} Dip the cotton tip into liquid and remove. Place the cap back on the tube. Do \emph{not} dip the swab back into the culture.
		
		\item Carefully lift the lid off the petri plate enough to reach the entire surface of the agar with the swab.
		
		\item Swab the surface of the entire surface of the agar. Use gentle sweeping strokes to cover the entire surface. Rotate the plate 90° and swab again to cover the entire surface with bacteria. Do \emph{not} dip the swab back into the culture.
		
		{\centering\includegraphics[height=1.5in]{petri_plate_swab}\par
		}
		
		\item \label{swab_last} Close the lid and place the swab into the provided biohazard bag.
		
		\item Repeat steps~\ref{swab_first}–\ref{swab_last} for the other four petri plates. \emph{Use one new swab per plate.}
		
		\item \label{disk_first} Use sterile forceps to carefully remove one sterile disk from the glass petri dish. Dip the disk into the the microcentrifuge tube labeled C for one second and remove. This is control solution.
		
		\item \label{disk_last} Carefully lift the lid from the petri dish labeled C. Carefully lay the wet disk in the center of the petri plate. Place the lid back on the plate.

%		{\centering\includegraphics[height=1in]{petri_plate_disks}\par
%		}

		\item Repeat steps~\ref{disk_first}–\ref{disk_last} for each of the four concentrations of Triclosan.
		
		\item Place your petri plates in a stack \emph{upside down} on the counter at the back of the room.
		
	\end{enumerate}

The bacteria will be allowed to grow at room temperature for 24–48 hours. Your instructor will move the plates to a refrigerator to inhibit continued bacterial growth until you collect your data in lab next week.

\subsubsection*{Data collection}

\begin{enumerate}

	\item Bring your petri plates from the back counter to your table. Place the plates on your table with the top side down to take measurements. \emph{Do not open the plates to take measurements.} 
	
	\item Each plate should have%, starting from the center, 
	a clear zone of inhibition around the disk where no bacterial colonies are growing and a solid
	bacterial “lawn,” as shown in the figure below. A “halo” may be present with
	scattered bacterial colonies growing between the zone of inhibition and the lawn.
	
	%, a “halo” with scattered bacterial colonies growing, and then a solid bacterial “lawn,” as shown in the figure below.
	
		{\centering\includegraphics[height=2.5in]{01_petri_plate_measure}\par
		}

	\item Measure the height and width (in centimeter) of the zone 
	of inhibition (arrows in the figure above), and then calculate 
	the average of the two values for the diameter. Record the average 
	diameter in the table below. Do not include or measure the halo.
	%Measure the diameter in centimeters of the zone of inhibition with the ruler (horizontal line in the figure below). You will have to estimate the boundary between the zone of inhibition and the halo, if present. Record the measurement in the table below. Do this for all of your plates. %Measure the diameter of the halo (vertical line). Record the measurement in the table. You do have to hold the ruler at the same angles as indicated in the figure. Do this for all five plates.

%		{\centering\includegraphics[height=1.125in]{01_petri_plate_measure}\par
%		}

	\item After you have your measurements, return your plates in a stack to the back counter where you found them.
	
\end{enumerate}

\begin{longtable}[c]{lc}
	\toprule
		Concentation	&	Inhibition Diameter	\tabularnewline
	\midrule
		& \tabularnewline[0.5em]
		0.0\% (Control)	& \rule{0.75in}{0.4pt} \tabularnewline[1.5em]
		0.1\%	& \rule{0.75in}{0.4pt} \tabularnewline[1.5em]
		0.3\%	& \rule{0.75in}{0.4pt} \tabularnewline[1.5em]
		0.5\%	& \rule{0.75in}{0.4pt} \tabularnewline[1.5em]
		0.7\%	& \rule{0.75in}{0.4pt} \tabularnewline
	\bottomrule
\end{longtable}

%\begin{longtable}[c]{lcc}
%	\toprule
%	Concentation	&	Inhibition Diameter	&	Halo Diameter \tabularnewline
%	\midrule
%	& & \tabularnewline[0.5em]
%	0.0\% (Control)	& \rule{0.75in}{0.4pt} & \rule{0.75in}{0.4pt} \tabularnewline[1.5em]
%	0.1\%	& \rule{0.75in}{0.4pt} & \rule{0.75in}{0.4pt} \tabularnewline[1.5em]
%	0.3\%	& \rule{0.75in}{0.4pt} & \rule{0.75in}{0.4pt} \tabularnewline[1.5em]
%	0.5\%	& \rule{0.75in}{0.4pt} & \rule{0.75in}{0.4pt} \tabularnewline[1.5em]
%	0.7\%	& \rule{0.75in}{0.4pt} & \rule{0.75in}{0.4pt} \tabularnewline
%	\bottomrule
%\end{longtable}

\question \label{ques:zoi}
Based on the data from your entire lab section, what is the average diameter for the zone of inhibition for each concentration?

\vspace*{0.5\baselineskip}

0.1\%: \hfill 0.3\%: \hfill 0.5\%: \hfill 0.7\%: \hfill \phantom{|}

\vspace*{0.5\baselineskip}

%\question \label{ques:halo}
%Based on the data from your entire lab section, what is the average diameter for the halo zone for each concentration?
%
%\vspace*{0.5\baselineskip}
%
%0.1\%: \hfill 0.3\%: \hfill 0.5\%: \hfill 0.7\%: \hfill \phantom{|}
%
%\vspace*{0.5\baselineskip}

\question
What do you conclude?  Did the results agree with the prediction made by your research hypothesis or by your null hypothesis? Does this support or falsify your research hypothesis?

\vspace*{3\baselineskip}


%\subsubsection*{Graph your results}
%
%Sketch a line graph of the averages on the grid below to illustrate your results. Use a solid line to show the trend for the diameter of the zone of inhibition. Use a dashed line to show the diameter of the halo. You will determine the scale to use for the diameter and label the \textsc{y}-axis accordingly. You will make a better version of this graph in Microsoft Excel or other suitable software.
%
%\begin{center}
%
%	\begin{tikzpicture}
%	
%		\newcommand*{\xMin}{0}%
%		\newcommand*{\xMax}{10}%
%		
%		\draw (0,0) grid (10,10);
%		\draw node [below] at (5,-0.5) {Concentration (\%)};
%		\draw node [left] at (-1,5) {Diameter (cm)};
%		\foreach \i in {\xMin,...,\xMax}{
%			\draw node [below] at (\i,0) {$\i$};
%		}
%
%	\end{tikzpicture}
%
%\end{center}
%
\end{questions}

\subsubsection*{Literature Cited}

\hangpara{\litindent}{1}
Livermore, D.\,M. 2000. Antibiotic resistance to staphylococci. International Journal of Antimicrobial Agents 16 (Suppl. 1): S3–10.

\hangpara{\litindent}{1}
Serafini, A. and D.\,M.\,Matthews. 2009. Microbial resistance to Triclosan: a case study in natural selection. American Biology Teacher 71: 536–540.\footnote{This exercise is based on Serafini and Matthews 2009.}

%\newpage

\subsection*{Formal lab report: natural selection in bacteria (30 points)}

You must write an independent lab report based on
the experiment you ran in your group, and on the data collected as a
class. Your lab instructor will tell you whether to turn in your report online or in lab, and when your report is due. All reports must be turned in by the start of lab on the assigned due date. If you turn in your report in lab, then it must be typed (double-spaced).  

A lab report is written in narrative form and in past tense—you are
explaining to the reader your hypothesis, how you tested it, and the
results from the experiment. Each lab report section should be written
in paragraph form, \emph{no bulleted lists or orphan sentences!} Keep in
mind that a paragraph has at least three sentences, each paragraph should
discuss one major point or idea, and all sentences in the paragraph
should provide support to the first sentence (your topic sentence).

Your report must include the information described below, divided into the
sections as listed. Include the section headings in your lab report in the order listed. You must 
include all information listed in each section to earn full credit. 
Use this handout to guide you as you write.

\subsection*{Lab report sections}

\subsubsection*{Introduction (1 paragraph; 5 points)} 

This section should begin with general observations made regarding
bacteria and anti-microbial resistance that motivated your experiment.
Next, state your hypothesis (you developed this as a group in question~\ref{ques:hypothesis}).
From your hypothesis, make clear predictions in if/then form. Review your handouts from
the first week on the scientific method.

\subsubsection*{Methods (2 paragraphs, 5 points)} 

This section should be written in narrative form and in past tense. Do
not make a list of materials. You will mention the materials you used
throughout this section as relevant. \emph{This is not like writing a
recipe; you are not providing instructions for your reader}. Rather, you
are explaining to the reader how you set up the experiment and collected
your data.

As you write, consider which aspects are important to report to allow
the reader to understand how you set up your experiment. For example, it
is not necessary to include step A from the procedure (creating a catchy
name for your group), because this step is not necessary to repeat the
experiment. Likewise, ``remove the cap,'' ``put on a pair of
gloves,'' ``remove swab from wrapper,'' and ``lift the lid off of the petri
dish'' are also unnecessary to include. In the methods, you would simply
write something like, ``We dipped a sterile swab into the test culture
tube of \emph{B. cereus} and swabbed the entire surface of the agar.''
Notice how this simple statement combined steps \ref{swab_first}–\ref{swab_last} of your procedure.

The second paragraph should explain how you
collected your data. This section must be written in \emph{past tense} because you have
already completed the experiment and data collection. No credit will be
awarded for methods sections written in list or bullet form.

\subsubsection*{Results (1 paragraph, 1 figure, 10 points)}

%The results section will have \emph{two parts}. First, you will complete
%Table~1 and Figure~1 from this handout \emph{using class data.} You must
%write a suitable legend (caption) for the table \emph{and} the figure on
%that sheet and staple it to your lab report. Table and figure legends
%should provide all the necessary information for a reader to understand
%the data without searching through the rest of your lab report.

The results section describes your results in narrative
form. Describe the overall trend in the class data for the size of the zone of inhibition and halo 
relative to the different treatment groups that you recorded above for questions~\ref{ques:zoi} and~\ref{ques:halo}. \emph{Use metric
units to report your results.} Refer to Figure~1 (see below) in the text when you are describing 
the overall trend. For example, ``The zone of inhibition decreased in
size between 0.1\% and 0.5\% Triclosan (Figure 1).'' \emph{Avoid
providing any explanation or interpretation} of your results; this
section is only used to present the data collected in a brief and
descriptive format. You explain and interpret your results in the discussion section. 

%\emph{Do not} list mean values in this section; rather, 

Figure~1 will be a column chart that you will in Microsoft Excel during Lab~2. The chart will
be based on the results for your lab section (question~\ref{ques:zoi} above).% and~\ref{ques:halo} above). 
Include your figure with a proper caption in the results section. Figure captions
must provide all the information necessary for a reader to understand
the data without searching through the rest of your lab report. Lab~2 will teach 
you how to make proper scientific figures in Excel.

%\emph{Remove the completed final page from this handout and attach it to your lab report.}

\subsubsection*{Discussion (2 paragraphs, 10 points)} 

In this section, you summarize and interpret your results. Use what
was discussed in class as well as this handout to guide
your Discussion. But, \emph{do not} directly copy from the material provided.
You must address the following questions in narrative form, \emph{in your own words,} with two
paragraphs.

Was your hypothesis supported by your results? Explain why or why not. 
The results should agree with your prediction for your hypothesis to be supported. 

If a halo was present (not measured), why was it present? Was the halo visible on all of
the plates with Triclosan? If so, did the relative size of the halo vary
with concentration? Why? How might this relate to bacterial resistance? How could this
relate to natural selection? 

Address these questions as necessary: Did anything go wrong during your
experiment? Did you get any unexpected results? Can you hypothesize
why that may have happened? What might you do differently next time?

Finally, place your experiment in the broader context of natural
selection as related to antimicrobial resistance in one or two
sentences. In other words, why might your results be important to the
scientific community?

%\newpage

%\thispagestyle{empty}

%\vfill

%	\hfil This page wastefully left blank. Hug a tree today!\hfill
	
%\vfill


%\newpage
%\thispagestyle{empty}
%
%Table 1. Use this table to record your data. Use these data to make your column chart in Excel.
%
%\vspace{5\baselineskip}
%
%\begin{longtable}[c]{lcc}
%	\toprule
%		Concentation	&	Inhibition Diameter	&	Halo Diameter \tabularnewline
%	\midrule
%		& & \tabularnewline[0.75em]
%		0.0\% (Control)	& \rule{0.75in}{0.4pt} & \rule{0.75in}{0.4pt} \tabularnewline[2em]
%		0.1\%	& \rule{0.75in}{0.4pt} & \rule{0.75in}{0.4pt} \tabularnewline[2em]
%		0.3\%	& \rule{0.75in}{0.4pt} & \rule{0.75in}{0.4pt} \tabularnewline[2em]
%		0.5\%	& \rule{0.75in}{0.4pt} & \rule{0.75in}{0.4pt} \tabularnewline[2em]
%		0.7\%	& \rule{0.75in}{0.4pt} & \rule{0.75in}{0.4pt} \tabularnewline
%	\bottomrule
%\end{longtable}
%
%\vspace{\stretch{1}}

%\vskip0pt plus 1filll

%\emph{Remove this page and attach it to your lab report.  Complete both sides.}

%\newpage
%\thispagestyle{empty}
%
%Figure 1.\vspace{5\baselineskip}
%
%\begin{center}
%
%	\begin{tikzpicture}
%	
%		\newcommand*{\xMin}{0}%
%		\newcommand*{\xMax}{10}%
%		
%		\draw (0,0) grid (10,10);
%		\draw node [below] at (5,-0.5) {Concentration (\%)};
%		\draw node [left] at (-1,5) {Diameter (cm)};
%		\foreach \i in {\xMin,...,\xMax}{
%			\draw node [below] at (\i,0) {$\i$};
%		}
%
%	\end{tikzpicture}
%
%\end{center}


\end{document}  