%!TEX TS-program = lualatex
%!TEX encoding = UTF-8 Unicode

\documentclass[12pt, hidelinks]{exam}

%\printanswers


\usepackage{graphicx}
	\graphicspath{{/Users/goby/Pictures/teach/163/lab/}
	{img/}} % set of paths to search for images

\usepackage{geometry}
\geometry{left=1.5in, bottom=1in}                   
%\geometry{landscape}                % Activate for for rotated page geometry
\usepackage[parfill]{parskip}    % Activate to begin paragraphs with an empty line rather than an indent
%\usepackage{amssymb, amsmath}
%\usepackage{mathtools}
%	\everymath{\displaystyle}

\usepackage{pdflscape}

\usepackage{fontspec}
\setmainfont[Ligatures={TeX}, BoldFont={* Bold}, ItalicFont={* Italic}, BoldItalicFont={* BoldItalic}, Numbers={OldStyle}]{Linux Libertine O}
\setsansfont[Scale=MatchLowercase,Ligatures=TeX]{Linux Biolinum O}
\setmonofont[Scale=MatchLowercase]{Linux Libertine Mono O}
\usepackage{microtype}

%\usepackage{unicode-math}
%\setmathfont[Scale=MatchLowercase]{Asana Math}
%\setmathfont[Scale=MatchLowercase]{XITS Math}

% To define fonts for particular uses within a document. For example, 
% This sets the Libertine font to use tabular number format for tables.
\newfontfamily{\dnatable}[Numbers={Monospaced}]{Linux Libertine Mono O}
\newfontfamily{\regfont}[ItalicFont={* Italic}]{Linux Libertine O}

\usepackage{booktabs}
\usepackage{multicol}

\usepackage[justification=raggedright, labelsep=period]{caption}
\captionsetup{singlelinecheck=off}
\captionsetup{skip=0.2em}
\captionsetup{hypcap=false}

%\usepackage{tabularx}
\usepackage{longtable}
%\usepackage{siunitx}
\usepackage{array}
\newcolumntype{L}[1]{>{\raggedright\let\newline\\\arraybackslash\hspace{0pt}}p{#1}}
\newcolumntype{C}[1]{>{\centering\let\newline\\\arraybackslash\hspace{0pt}}p{#1}}
\newcolumntype{R}[1]{>{\raggedleft\let\newline\\\arraybackslash\hspace{0pt}}p{#1}}

\usepackage{enumitem}
\usepackage{enumitem}
\setlist{leftmargin=*}
\setlist[1]{labelindent=\parindent}
\setlist[enumerate]{label=\textsc{\alph*}.}

\usepackage{hyperref}
%\usepackage{placeins} %PRovides \FloatBarrier to flush all floats before a certain point.
%\usepackage{hanging}

\usepackage{tikz}
%\tikzstyle{every picture}+=[remember picture,overlay]
\usetikzlibrary{arrows, arrows.meta}
\usetikzlibrary{positioning, backgrounds}

\usepackage{forest}
\forestset{
	every leaf node/.style={
		if n children=0{#1}{}
	},
	every tree node/.style={
		if n children=0{}{#1}
	},
	mytree/.style={
		for tree={
			edge path={
				\noexpand\path [draw, very thick, \forestoption{edge}] (!u.parent anchor) |- (.child anchor)\forestoption{edge label};
			},
			every tree node={draw=none,inner sep=0, outer sep=0, minimum size=0},
			every leaf node/.style={align=left},
			grow=east,
			parent anchor=east, 
			child anchor=west,
			anchor=west,
			l sep=3mm,
			s sep=3mm,
			draw=none,
			if n children=0{tier=word}{}
		}
	}
}

\usepackage[sc]{titlesec}

\renewcommand{\solutiontitle}{\noindent}
\unframedsolutions
\SolutionEmphasis{\bfseries}

\renewcommand{\questionshook}{%
	\setlength{\leftmargin}{-\leftskip}%
}

%Change \half command from 1/2 to .5
\renewcommand*\half{.5}


%% Allows fullwidth command to break across pages.
%% See 
%\makeatletter
%\def\SetTotalwidth{\advance\linewidth by \@totalleftmargin
%\@totalleftmargin=0pt}
%\makeatother


\pagestyle{headandfoot}
\firstpageheader{\textsc{bi} 063: Evolution and Ecology}{}{\ifprintanswers\textbf{KEY}\else Name: \enspace \makebox[2.5in]{\hrulefill}\fi}
\runningheader{}{}{\footnotesize{pg. \thepage}}
\footer{}{}{}
\runningheadrule

\newcommand*\AnswerBox[2]{%
    \parbox[t][#1]{0.92\textwidth}{%
    \begin{solution}#2\end{solution}}
    \vspace{\stretch{1}}
}

\newenvironment{AnswerPage}[1]
    {\begin{minipage}[t][#1]{0.92\textwidth}%
    \begin{solution}}
    {\end{solution}\end{minipage}
    \vspace{\stretch{1}}}

\newlength{\basespace}
\setlength{\basespace}{5\baselineskip}

\begin{document}

\subsection*{Evolution of lizards on the Canary Islands\footnote{Based on a laboratory exercise developed \href{http://www.ucmp.berkeley.edu/fosrec/Filson.html}{R.P. Filson at the University of California Berkeley.}}}

Volcanic islands that form in the oceans, like the Hawaiian Islands, create natural experiments in evolution. The islands cannot be inhabited by organisms until after they have formed. Islands become inhabited through founder events. Older islands should be founded or colonized first and thus have older species. Younger islands would be founded later and thus have younger species. These predictions can be tested by combing the geology of islands with biological evidence such as morphology and genetics.

In this exercise, you will study three lizard species, \textit{Gallotia atlantica, G. stehlini,} and \textit{G. galloti} (Figure~\ref{fig:gallotia_lizard}). Further, \textit{G. galloti} has four morphologically distinct island populations. These lizards inhabit the Canary Islands, just west of the continent of Africa (Figure~\ref{fig:canary_islands}).

\begin{multicols}{2}

	\begin{center}
		\includegraphics[width=\columnwidth]{06_lizard_biogeo_gallotia}
		
		\captionof{figure}{\textit{Gallotia galloti} from Tenerife. {\small Photo by Jörg Hempel, \textsc{cc-by-sa 3.0}.}\label{fig:gallotia_lizard}}
	\end{center}
\columnbreak
	
	\begin{center}
	
		\includegraphics[width=\columnwidth]{06_lizard_biogeo_canary_islands}
	
		\captionof{figure}{The Canary Islands. {\small Based on Anguita et al. 1986.}\label{fig:canary_islands}}
	\end{center}
\end{multicols}

The Canary Island chain starts
about 85 km (50 miles) west of the continent, following a geological fault line from
the Atlas Mountains in northern Africa. As Africa drifts eastward by plate tectonics, 
new sections of the fault line move over the hot spot, creating new volcanic islands
 (Similar hot spots cause the formation of the Hawaiian Islands, and the hot springs and eruptions of Old Faithful in Yellowstone National Park.) Thus
the most eastern island, Lanzarote, is oldest at about 24 million years, while the smaller western
island, Hierro, is the youngest, about 0.8 million years old and closest to the hot spot location.

\emph{Gallotia} lizards probably arrived by way of rafting. Rafts of natural vegetation are often
washed out to sea when high river levels cause river banks to collapse,
carrying away both plants and clinging animals. The distance and size of the islands affect the chance of colonization of the islands by the lizards, as well as subsequent gene flow and chances for speciation.

The interaction of island distance and size and their effects on evolutionary processes can be summarized by five general principles.

%\newpage

\begin{enumerate}\label{general_principles}

	\item The closer the island to another land mass, the higher the
	probability of colonization.

	\item The older the island, the more likely it will be colonized.
	
	\item The larger the island, the more species are likely to become
	established on the island.

	\item Geographic isolation (larger distances) among islands reduces 
	gene flow among populations of the lizards.
	
	\item Over time, founder populations become genetically divergent from
	their parent population due to combinations of natural selection, random 
	mutation, and genetic drift.

\end{enumerate} 

\subsubsection*{Evolutionary questions}

Evolutionary biologists are interested
in the following questions. What is the phylogenetic history of the three
species and seven populations of \emph{Gallotia} lizards on the Canary
Islands? Does the presence of four morphologically different populations
of \emph{G. galloti} on the four westernmost islands
(Figure~\ref{fig:distribution}) imply continuing evolution? In this
exercise, you will use data from geography, geological history,
morphology (body size), and molecular genetics to answer
these questions.

\begin{center}

	\includegraphics[width=\textwidth]{06_lizard_biogeo_distribution}
	
	\captionof{figure}{Distribution of \textit{Gallotia atlantica, G. stehlini,} and \textit{G. galloti.} \textit{Gallotia galloti} has colonized the four westernmost islands and each population is morphologically distinct from the others. {\small Redrawn from Thorpe et al., 1993.}\label{fig:distribution}}

\end{center}

\subsubsection*{Part 1: Lizard phylogeny based on geographic distance}

\textit{Review the general principles on page~\pageref{general_principles} to successfully answer the questions below.}

\begin{questions}
	
\question
Using Figure~\ref{fig:canary_islands}, measure the distance in kilometers of each island to the mainland (Africa). List these distances on a separate page. Include the following islands: Lanzarote, Fuerteventura, Gran Canaria, Tenerife, Gomera, Palma, and Hierro.

\question\label{ques:colonization_order_prediction}
Which island is most likely to have been colonized first and which last? Tell why you think so.

\AnswerBox{3\baselineskip}{%
Fuerteventura (or/and Lanzarote) should be colonized first. It is the largest and closest to Africa. Hierro is smallest and farthest so it was probably colonized last (most recently).}

\question\label{geography_phylogeny}
Using Figure~\ref{fig:distribution} and your reasoning from Question~\ref{ques:colonization_order_prediction}, draw a phylogenetic tree of the species and populations of \emph{Gallotia.} Draw your tree on a separate sheet of paper. Label your end branches with the following species and island names: 

\begin{tabular}{@{}L{0.78in}L{0.78in}L{0.78in}L{0.78in}L{0.78in}L{0.99in}@{}}
	\textit{atlantica} Gran Canaria		&
	\textit{galloti} Gomera		&
	\textit{galloti} Hierro		&
	\textit{galloti} Palma 		&
	\textit{galloti} Tenerife	&
	\textit{stehlini}\newline Fuerteventura/\newline Lanzarote			\tabularnewline
\end{tabular}

Ask your instructor to check your tree before proceeding to the next part. Be prepared to explain your reasoning.\ifprintanswers \textbf{See last page of key for tree.}\fi


\subsubsection*{Part 2: Phylogeny based on geological history.}

Check your hypothetical phylogenetic tree against the geological 
data in Table~\ref{tab:geological_history}. The maximum age of 
each island was estimated by sampling volcanic rocks found on all 
islands. The ratio of radioactive potassium to its breakdown product, 
argon, was used to estimate the age of the rocks.


\begin{longtable}[l]{@{}C{1.1in}C{0.8in}C{0.75in}C{0.75in}C{0.75in}C{0.74in}@{}}
\caption{Maximum age of the Canary Islands in millions of years. {\small Anguita et al. 1986.
\label{tab:geological_history}}}\tabularnewline
	\toprule
	Lanzarote \& 
	Fuerteventura	& 
	Gran Canaria 	& 
	Tenerife		&
	Gomera			&
	Palma 			&
	Hierro			\tabularnewline
	\midrule
	24.0			&
	17.1			&
	15.1			&
	5.3				&
	2.0				&
	0.8					\tabularnewline
	\bottomrule
\end{longtable}

%\vspace{\baselineskip}

\question
Explain how the data in Table~\ref{tab:geological_history} support your phylogenetic tree. Or what changes should you make and why?

\AnswerBox{2\baselineskip}{Depends on their phylogeny but overall, the age of the islands should match. 
The oldest islands are closest to land. This will generally correspond to the distance of the islands from Africa.}

\subsubsection*{Part 3: Phylogeny based on molecular genetics}

Studies by Roger Thorpe and colleagues (1993, 1994) have tested various phylogenetic hypotheses by comparing genetic differences among the populations of the \textit{Gallotia} lizards on the Canary Islands. The gene for cytochrome b, which is coded by \textsc{dna} found in every cell’s mitochondria, was used in his studies. Cytochrome b is an important substance for cell metabolism and has probably been around since the first bacteria. Changes in its nucleotide base sequence (\textsc{A, T, C,} and \textsc{G}) that do not disrupt the gene’s function provides a kind of evolutionary clock. The rate of these changes due to mutations is relatively constant. 

When two populations are isolated and gene flow between them is restricted, the number of nucleotide differences accumulate over time. This means that the more time that has passed since two species last shared a common ancestor, the more differences there should be in their \textsc{dna.}  Thorpe and his colleagues compared the \textsc{dna} for the three species of lizards, and included the four populations of \textit{G. galloti.} Their results appear on page~\ref{tab:dna_results} of this exercise. 

\question
Fill in Table~\ref{tab:dna_similarity} by comparing the \textsc{dna} sequences between each pair of lizard species or populations (Table~\ref{tab:dna_results} on page~\pageref{tab:dna_results}). \emph{You can tear off the last page if you want.}

\vspace{\baselineskip}

{\captionof{table}{D\textsc{na} sequence differences among Canary Island lizards.\label{tab:dna_similarity}}
	\begin{tikzpicture}
		\node {\includegraphics[width=0.95\textwidth]{06_lizard_biogeo_pairwise_dna}};
		
		\ifprintanswers
		
		\node at (-3.5,0.8) {{\Large 39}}; % stehlini v galloti Palma
		\node at (-1.7,0.8) {{\Large 24}}; % atlantica v galloti Palma
		
		\node at (-3.5,-0.8) {{\Large 38}}; % stehlini v galloti Tenerife
		\node at (-1.7,-0.8) {{\Large 22}}; % atlantica v galloti Tenerife
		\node at (0.1,-0.8) {{\Large 8}}; % galloti Palma v galloti Tenerife
		
		\node at (-3.5,-2.4) {{\Large 42}}; % stehlini v galloti Gomera
		\node at (-1.7,-2.4) {{\Large 23}}; % atlantica v galloti Gomera
		\node at (0.1,-2.4) {{\Large 18}}; % galloti Palma v galloti Gomera
		\node at (1.9,-2.4) {{\Large 18}}; % galloti Tenerife v galloti Gomera

		\node at (-3.5,-4) {{\Large 46}}; % stehlini v galloti Hierro
		\node at (-1.7,-4) {{\Large 28}}; % atlantica v galloti Hierro
		\node at (0.1,-4) {{\Large 18}}; % galloti Palma v galloti Hierro
		\node at (1.9,-4) {{\Large 20}}; % galloti Tenerife v galloti Hierro
		\node at (3.7,-4) {{\Large 4}}; % galloti Gomera v galloti Hierro
		\fi
	\end{tikzpicture}
}

\newpage

\question
Which two species or populations are most closely related? Justify your answer. 

\AnswerBox{1\baselineskip}{\textit{G. galloti} from Hierro and Gomera should have the fewest differences (about 4).}

\question
Which species or population is most distantly related to the rest? Why do you say so? 

\AnswerBox{1\baselineskip}{\textit{G. stehlini} has the greatest genetic distance to other species so it is the basal species.}

\question
You will now make a phylogenetic tree from \textsc{dna} differences but the results are too complex to make a phylogeny by hand, so simplify the data. For differences 

\begin{itemize}
	\begin{multicols}{2}
	\item between 30--49, use 35.
	\item between 21--29, use 25.
	\item between 11--20, use 15.
	\item of 10 or less, use the actual value.
\end{multicols}
\end{itemize}

\question \label{genetic_phylogeny}
Use the simplified differences to draw a phylogenetic tree on a separate sheet of paper. \emph{Read the important information below.}

\textsc{Important:} Low numbers express greater genetic similarity between two species, which means they share a more recent common ancestor. For example, \textit{G. galloti} from Gomera and Hierro will show the fewest number of differences, meaning they share a more recent common ancestor compared to the other pairs. Pairs with high numbers have lower genetic similarity, so their common ancestry would be farther back in time.

%\textsc{Hint:} Interpret these number broadly. Consider species with 30–45 differences to have the \emph{same number} of differences. Do the same for 15–29 differences. Treat numbers less than 10 as is. That should make tree construction easier. 

Refer to your phylogenetic tree based on genetic similarities found in Table~\ref{tab:dna_similarity}. Compare it to the phylogeny chart you drew based on the geographic distances and geologic age of the islands. 

\question Describe the difference(s) between the phylogeny you made based on island distance and age to the phylogeny you made based on \textsc{dna} differences.

\AnswerBox{2\baselineskip}{Depends on their phylogenies. Roll with it.}

%\question
%Use your two phylogenies to tell which species, \textit{G. stehlini} or \textit{G. atlantica,} is the ancestor of the other? Explain your reasoning. 
%
%\AnswerBox{3\baselineskip}{Probably \textit{G. stehlini} because of molecular data. If needed, tell student chance and variable ocean currents could affect when organisms arrive on an island.}
%

\question
Tenerife is the largest of the four islands that have \textit{G. galloti} (Figure~\ref{fig:distribution}). Tenerife is moist and lush in the north while arid and barren in the south. The \textit{G. galloti} sequence you used earlier was from a northern lizard. 

Would you predict any \textsc{dna} differences between lizards that live on the northern part of Tenerife compared to lizards that live on the southern part? Explain.

\AnswerBox{2\baselineskip}{Student's choice.}

\question
Would you expect the number of \textsc{dna} differences between northern and southern \textit{G. galloti} to be less or more than the number of differences than between the pairs of \textit{G. galloti} shown in Table~\ref{tab:dna_similarity}? Why?

\AnswerBox{2\baselineskip}{Student's choice. Probably most will say few differences within island.}

\question
Table~\ref{tab:galloti} shows the sequences for northern and southern \textit{G. galloti.} As you did before, count the number of differences between this pair.

{\dnatable
\begin{longtable}[l]{ll}
\caption{D\textsc{na} for \textit{G. galloti} from northern and southern Tenerife. A period \\indicates the same nucleotide as northern \textit{G. galloti.}\label{tab:galloti}}\tabularnewline
\toprule
{\regfont North} & TTCAATACACCACCGTCATTGAAATATTGG \tabularnewline
{\regfont South} & .C....G.....T...T..C....C..C.. \tabularnewline
\bottomrule
\end{longtable}
}

Number of differences: \ifprintanswers \textbf{7} \fi

\vspace{\baselineskip}

\question
Did the results agree with your prediction? 

\AnswerBox{2\baselineskip}{Depends on prediction. Most will probably be surprised becase the within island difference is greater than the difference between Gomera and Hierro populations.}

\question[Checkout]
Provide an explanation based on habitat, natural selection, and gene flow that might explain the genetic differences between the northern and southern lizards on Tenerife.

\AnswerBox{3\baselineskip}{Wide latitude but allow for adaptation to environmental conditions may limit gene flow. Some may even go hybrid zone route.}

\question[Checkout]
Predict what is likely to happen to the four populations of \textit{G. galloti} on the four westernmost islands. State what evolutionary conditions will support this prediction, and why.

\AnswerBox{3\baselineskip}{If gene flow is disrupted could become different species. They might include natural selection, drift, and mutation but \emph{should} mention gene flow for speciation. If they say gene flow is high so no speciation, ask them if there would be genetic differences among the island populations if gene flow is high.}



\end{questions}


\newpage

\begin{landscape}

{\fontsize{11pt}{13pt}\selectfont
\dnatable
\begin{longtable}[l]{lccccccccccc}
\caption{D\textsc{na} for \textit{Gallotia} lizards from the Canary Islands. \textit{Gallotia stehlini} is the reference sequence. A period \\indicates the same nucleotide as \textit{G. stehlini.}\label{tab:dna_results}}\tabularnewline
\toprule
{\regfont\textit{G. stehlini}}	& GACTC & AATCA & TTCAA & CACAG & GCCTC & TTCCT & AGCCA & TGCAC & ACATT & TGCCC \tabularnewline
{\regfont\textit{G. atlantica}}	& ....T & ...T. & ..... & T.... & ....A & ...T. & ...A. & ..... & ..... & ..... \tabularnewline
{\regfont\textit{G. galloti} (P)}	& ....T & ...T. & .C... & T.... & ....A & ...T. & G..A. & .A... & ....C & C.... \tabularnewline
{\regfont\textit{G. galloti} (T)}	& ....T & ...T. & ..... & T.... & ....A & ...T. & G..A. & .A... & ....C & C.... \tabularnewline
{\regfont\textit{G. galloti} (G)}	& ....T & ...T. & .C... & T.... & ....A & ...T. & G..A. & ..... & ....C & C.... \tabularnewline
{\regfont\textit{G. galloti} (H)}	& ....T & ...T. & .C... & T.... & ....A & ...T. & G..A. & ..... & ....C & C.... \tabularnewline%[1ex]
&&&&&&&&&&\tabularnewline
{\regfont\textit{G. stehlini}}	& ACATC & CACCG & TGATG & TCCAA & CACGG & ATGAC & TCATT & CGCAA & TGTCC & AACGG \tabularnewline
{\regfont\textit{G. atlantica}}	& ..... & ..... & ..... & .T... & ..T.. & T.... & .T..C & ..A.. & .A... & ..... \tabularnewline
{\regfont\textit{G. galloti} (P)}	& ....T & ..T.. & ...C. & ..... & ..... & T.... & .A..C & ..A.. & .A... & ..... \tabularnewline
{\regfont\textit{G. galloti} (T)}	& ....T & ..T.. & ...C. & ..... & ..... & T.... & .T... & ..A.. & .A... & ..T.. \tabularnewline
{\regfont\textit{G. galloti} (G)}	& ..... & ..... & ...C. & ..... & ..... & T..G. & .T..C & ..A.. & CA... & ..T.. \tabularnewline
{\regfont\textit{G. galloti} (H)}	& ....T & ..... & ...C. & ..... & ..... & T..G. & .T..C & ..A.. & CA... & ..T.. \tabularnewline
&&&&&&&&&&\tabularnewline
{\regfont\textit{G. stehlini}}	& CGCTT & TTCTT & CATCT & TACGC & GCATA & TCGGA & CGTGG & CCTGT & ATTAC & ATACC \tabularnewline
{\regfont\textit{G. atlantica}}	& A..C. & ..T.. & T.... & ...AT & ...C. & .T... & ..... & ..... & .C... & ....T \tabularnewline
{\regfont\textit{G. galloti} (P)}	& A..C. & ..T.. & T.... & ...AT & A..C. & .T... & ..G.. & ..... & ..... & ....T \tabularnewline
{\regfont\textit{G. galloti} (T)}	& A..C. & ..T.. & T.... & ...AT & A..C. & .T... & ..... & ...A. & .C... & ....T \tabularnewline
{\regfont\textit{G. galloti} (G)}	& A..C. & ..T.. & T.... & ...AT & A..C. & .T... & ..G.. & .T.A. & ..... & ...T. \tabularnewline
{\regfont\textit{G. galloti} (H)}	& A..C. & ..T.. & T.... & ...AT & A..C. & .T... & ..G.. & TT.A. & ..... & ...T. \tabularnewline
&&&&&&&&&&\tabularnewline
{\regfont\textit{G. stehlini}}	& TATTT & ACTGA & AACCT & GAAAC & ATTGG & AGTCC & TCCTC & CTTCT & GCTAG & TTATA \tabularnewline
{\regfont\textit{G. atlantica}}	& ..... & GT... & ...T. & ..... & ..C.. & ...A. & .T..A & ..A.. & T.... & .C... \tabularnewline
{\regfont\textit{G. galloti} (P)}	& .G... & ..... & ..... & ....T & ..... & ...A. & .T..T & ..C.. & AT... & .C... \tabularnewline
{\regfont\textit{G. galloti} (T)}	& ..... & ..... & ..... & ....T & ..... & ...A. & .T..T & ..C.. & AT... & .C... \tabularnewline
{\regfont\textit{G. galloti} (G)}	& ..... & .T... & ...A. & ....T & ..C.. & ...A. & .T... & ..C.. & A...G & .C... \tabularnewline
{\regfont\textit{G. galloti} (H)}	& .G... & .T... & ...A. & ....T & ..C.. & G..A. & .T... & ..C.. & A...G & .C... \tabularnewline
\bottomrule
\end{longtable}
}

\end{landscape}

\newpage
\newgeometry{top=1in, bottom=1in}
\thispagestyle{empty}
\ifprintanswers

\textsc{Tree for question~\ref{geography_phylogeny}.} They might switch Gomera and Palma, which is OK.

\begin{forest} mytree
[
  [
    [\textit{G. stehlini}]
    [
      [\textit{G. atlantica}]
      [
        [\textit{G. galloti} Tenerife]
        [
          [\textit{G. galloti} Gomera] 
          [
            [\textit{G.galloti} Palma]
            [\textit{G. galloti} Hierro]
          ]
        ]
      ]
    ]
  ]
]
\end{forest}

\bigskip

\textsc{Tree based on Table~\ref{tab:geological_history}}, for comparison to above tree.

\begin{forest} mytree
	[
	[
	[Lanzarote \& Fuerteventura: 24.0 \textsc{mya}]
	[
	[Gran Canaria: 17.1 \textsc{mya}]
	[
	[Tenerife 15.1 \textsc{mya}]
	[
	[Gomera 5.3 \textsc{mya}]
	[
	[Palma 2.0 \textsc{mya}]
	[Hierro 0.8 \textsc{mya}]
	]
	]
	]
	]
	]
	]
\end{forest}

\bigskip

\textsc{Tree for question~\ref{genetic_phylogeny}.}

%\begin{forest} mytree
%[
% [
%  [\textit{G. stehlini}]
%  [
%   [\textit{G. atlantica}]
%   [
%	[\textit{G. galloti} Tenerife]
%	[
%	 [\textit{G. galloti} Palma] 
%	 [
%	  [\textit{G.galloti} Gomera]
%	  [\textit{G. galloti} Hierro]
%	 ]
%	]
%   ]
%  ]
% ]
%]
%\end{forest}

\begin{forest} mytree
[
 [
  [\textit{G. stehlini}]
  [
   [\textit{G. atlantica}]
   [[
    [\textit{G. galloti} Tenerife]
	[\textit{G. galloti} Palma] 
   ]
   [
	[\textit{G.galloti} Gomera]
	[\textit{G. galloti} Hierro]
   ]
  ]]
 ]
]
%	]
\end{forest}

\fi

\end{document}  