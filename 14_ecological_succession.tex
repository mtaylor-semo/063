%!TEX TS-program = lualatex
%!TEX encoding = UTF-8 Unicode

\documentclass[12pt, hidelinks]{exam}

\printanswers

\usepackage{graphicx}
	\graphicspath{{/Users/goby/Pictures/teach/163/lab/}
	{img/}} % set of paths to search for images

\usepackage{geometry}
\geometry{letterpaper, left=1.5in, bottom=1in}                   
%\geometry{landscape}                % Activate for for rotated page geometry
\usepackage[parfill]{parskip}    % Activate to begin paragraphs with an empty line rather than an indent
\usepackage{amssymb, amsmath}
\usepackage{mathtools}
	\everymath{\displaystyle}

\usepackage{fontspec}
\setmainfont[Ligatures={TeX}, BoldFont={* Bold}, ItalicFont={* Italic}, BoldItalicFont={* BoldItalic}, Numbers={Proportional, OldStyle}]{Linux Libertine O}
\setsansfont[Scale=MatchLowercase,Ligatures=TeX, Numbers={Proportional,OldStyle}]{Linux Biolinum O}
\setmonofont[Scale=MatchLowercase]{Linux Libertine Mono O}
\newfontfamily{\liningnum}[Numbers=Lining]{Linux Libertine O}
\usepackage{microtype}

\usepackage[table]{xcolor}

\usepackage{unicode-math}
\setmathfont[Scale=MatchLowercase]{Tex Gyre Pagella Math}


% To define fonts for particular uses within a document. For example, 
% This sets the Libertine font to use tabular number format for tables.
 %\newfontfamily{\tablenumbers}[Numbers={Monospaced}]{Linux Libertine O}
% \newfontfamily{\libertinedisplay}{Linux Libertine Display O}

\usepackage{booktabs}
\usepackage{multicol}

\usepackage{caption}
\captionsetup{format=plain, justification=raggedright, singlelinecheck=off,labelsep=period,skip=3pt} % Removes colon following figure / table number.

%\usepackage{caption}
%\captionsetup{font=small} 
%\captionsetup{singlelinecheck=false}
%\captionsetup[figure]{labelsep=period, format=plain}

\usepackage{longtable}
\usepackage{caption}
\captionsetup{format=plain, justification=raggedright, singlelinecheck=off,labelsep=period,skip=3pt} 

\usepackage{array}
\newcolumntype{L}[1]{>{\raggedright\let\newline\\\arraybackslash\hspace{0pt}}p{#1}}
\newcolumntype{C}[1]{>{\centering\let\newline\\\arraybackslash\hspace{0pt}}p{#1}}
\newcolumntype{R}[1]{>{\raggedleft\let\newline\\\arraybackslash\hspace{0pt}}p{#1}}

\usepackage{enumitem}
\setlist{leftmargin=*}
\setlist[1]{labelindent=\parindent}
\setlist[enumerate]{label=\textsc{\alph*}.}
\setlist[itemize]{label=\color{gray}\textbullet}

\usepackage{hyperref}
%\usepackage{placeins} %PRovides \FloatBarrier to flush all floats before a certain point.
\usepackage{hanging}

\usepackage[sc]{titlesec}

%% Commands for Exam class
\renewcommand{\solutiontitle}{\noindent}
\unframedsolutions
\SolutionEmphasis{\bfseries}

\renewcommand{\questionshook}{%
	\setlength{\leftmargin}{-\leftskip}%
}

%Change \half command from 1/2 to .5
\renewcommand*\half{.5}

\pagestyle{headandfoot}
\firstpageheader{\textsc{bi}\,063 Evolution and Ecology}{}{\ifprintanswers\textbf{KEY}\else Name: \enspace \makebox[2.5in]{\hrulefill}\fi}
\runningheader{}{}{\footnotesize{pg. \thepage}}
\footer{}{}{}
\runningheadrule

\newcommand*\AnswerBox[2]{%
    \parbox[t][#1]{0.92\textwidth}{%
    \begin{solution}#2\end{solution}}
    \vspace{\stretch{1}}
}

\newenvironment{AnswerPage}[1]
    {\begin{minipage}[t][#1]{0.92\textwidth}%
    \begin{solution}}
    {\end{solution}\end{minipage}
    \vspace{\stretch{1}}}

\newlength{\basespace}
\setlength{\basespace}{5\baselineskip}


\newcommand\chisq{$\chi^2$}
\newcommand*\meanY{\overline{Y}\kern0.67pt}

\newcommand*\AnswerBlank[1]{%
	\ifprintanswers%
		\textbf{#1}
	\else%
		\rule{0.75in}{0.4pt}\kern0.67pt.\fi%
	}

%\newcommand*\AnswerBlank{\rule{0.75in}{0.4pt}\kern0.67pt.}
\newcommand*\xcell[1]{cell~\liningnum{#1}}

%
%\makeatletter
%\def\SetTotalwidth{\advance\linewidth by \@totalleftmargin
%\@totalleftmargin=0pt}
%\makeatother



\begin{document}

\subsection*{Floristic relay: a game of ecological succession\footnote{Based on Ortiz-Barney et al. 2005. Teaching Issues and Experiments in Ecology 3.}}

Ecological succession is a response to disturbance of the environment in which an area is colonized by a variety of species that are gradually replaced by other species over time. A disturbance is an event such as a fire, storm, drought, or human activity that changes a community by removing organisms or altering resource availability. The types of disturbance, their frequencies, and their intensities determine the composition of a community.  Therefore, the species that form the community at any point in time coexist as they share similar abiotic and biotic requirements.  Early successional species, often referred to as pioneer species, thrive in disturbed areas, are fast growing, and typically prefer an open, sunny environment with bare soil for germination. Late successional communities are typically dominated by multi-aged and sized shrub and tree species.  These communities have a well-developed understory  dominated by shade-tolerant plants.  Late successional communities are often referred to as climax communities. In this activity, you will examine the response of early versus late successional species to various disturbance events over time.  Though many species can be considered mid-successional, we will only examine early and late successional species in this game.

\subsubsection*{Early and late succession plants}

The game at your table includes Event cards (fire, grazing, landslide, and no disturbance), Interaction cards (competition, facilitation, and tolerance), and Plant Species cards. Look at the species cards to see which plants are early successional and which are late successional. Look 
at how each plant responds to disturbance events. A plant moves forward if it gains an advantage from a disturbance. The more spaces it moves forward, the greater the advantage. If a plant moves backwards, then the disturbance is disadvantageous. The more spaces it moves backwards, the greater the disadvantage. Use this information to answer the following questions.

\begin{questions}

\question
If fire is the most frequent disturbance event, which of the six species would you predict to be most common at the end of the game? Why?

\AnswerBox{0.35\basespace}{Momerath Herb responds best to fires.}

\question
If grazing is the most frequent disturbance event, which of the six species would you predict to be most common at the end of the game?  Why?

\AnswerBox{0.35\basespace}{Grickle Grass responds best to grazing.}

\question \label{ques:hypothesis}
If few disturbances occur during succession then which character type of plants (early or late successional) should dominate the final community?

\AnswerBox{0.35\basespace}{Late successional.}

\question
Based on your answer to question~\ref{ques:hypothesis}, which three species should be most numerous in the final community? 

\AnswerBox{0.35\basespace}{Borogrove Grass, Mimsy Bush, Truffula Tree.}

\begin{samepage}

\question
Will the three species be equally numerous at the end of the game? Explain why you think so.
\textsc{Hint:} Compare the disturbance response of each species. 

\AnswerBox{0.35\basespace}{Probably not. The types and frequency of disturbance might affect the
	final abundance of each plant.}

	
\subsubsection*{Instructions}

Play the game with 5–6 students, preferably 6. The first player 
to reach “Finish” is the winner. Big surprise there, hunh?

\end{samepage}

\begin{enumerate}
	
	\item Choose a dealer. 
	
	\item Each player, including the dealer, chooses a plant species and then
	places the associated game piece in the “Start” square.
	
	\item Dealer shuffles the Event cards and places them face down in the 
	Future Events spot on the playing board. Shuffle and place the 
	Interaction cards face down in their spot.
	
	\item \label{q:event_card} The dealer draws the first Event Card and places it face 
	up in the Current Event spot.
	
	\item Each player then plays according to the Event and Character 
	Card directions, starting with the dealer and going clockwise.
	
	\item \label{q:interactions} After all players have their turn, check the board for 
	players who landed on the same square. These players are interacting.
	
	\begin{enumerate}[label=(\alph*)]
		\item Interactions are played in the same order as Events (clockwise starting at the dealer).

		\item Two at a time, the interacting players draw one Interaction Card.
		
		\item Play according to the interaction card.
	\end{enumerate}

	\filbreak

	\item Repeat Steps~\ref{q:event_card}–\ref{q:interactions} until a player wins. 
	Record the number of each type of event that occurred during the game and the finishing order of the players in Table~\ref{tab:round1_results} on page~\pageref{tab:round1_results}.
	
	
\end{enumerate}

\subsubsection*{What does your community look like?}

Follow these steps to make a diagram of the final community.

\begin{enumerate}[resume]
	\item Record the order of the players’ plants in the table.
	
	\item Next to the name of the plant is a number of plants. 
	Draw this number of plants in the space below.
	
	\item Use the key for the appropriate symbol to represent each plant.
	
\end{enumerate}

Use this example to guide you. This table shows the plant species
listed in order of finishing and the number of individuals for each
plant species to be drawn.

{\centering
	\includegraphics[width=0.58\textwidth]{14_example_table}\par
}

Here is the key showing how to draw each plant species. 

{\centering
	\includegraphics[width=0.58\textwidth]{14_key_to_plant_diagrams}\par
}

Here is the final diagram showing the proper number of each species
from the table.

{\centering
	\includegraphics[width=0.58\textwidth]{14_example_final_diagram}\par
}


\question
Fill in Table~\ref{tab:round1_results} with the results of your game.

{\setlength{\LTcapwidth}{5.8in}
\begin{longtable}{@{}|L{1.2in}|L{0.6in}|L{0.6in}|L{2in}|L{0.6in}|@{}}
\caption{Record the number of each type of event and the number of individuals for each plant species where indicated.}\label{tab:round1_results}\tabularnewline
	\hline
	Event & Number of cards & Order of players	& Name of plant species	& Number of plants \tabularnewline
	\hline
	Fire & & 1st & & 6 plants \tabularnewline[0.25cm]
	\hline
	Grazing & & 2nd & & 5 plants \tabularnewline[0.25cm]
	\hline
	Landslide & & 3rd & & 4 plants\tabularnewline[0.25cm]
	\hline
	No Disturbance & & 4th & & 3 plants \tabularnewline[0.25cm]
	\hline
	& & 5th & & 2 plants \tabularnewline[0.25cm]
	\hline
	& & 6th & & 1 plant \tabularnewline[0.25cm]
	\hline
\end{longtable}}


\question
Draw your final community here. Select one member of your group to draw this community on the whiteboard at the front of the room. Draw the early successional species with one color and the late successional species with a different color. Your instructor will tell you which colors to use. %Draw the early successional species with one color and the late successional species with a different color.

\vspace*{3\basespace}

\question
Would you describe the diagram produced as more like a forest, a grassland or a shrubland? Why?

\AnswerBox{2\baselineskip}{Answers will vary depending on their results.}


\newpage


\question
Will the winning plant always be the same? Why or why not?

\AnswerBox{0.2\basespace}{No. The winning species is determined
	by the frequency and types of disturbance, interactions with other species, and random events.}

%\question[Checkout]
%Describe how would you change the Events deck to ensure that your species wins?

%\AnswerBox{0.2\basespace}{Stack the deck with more event cards that
%	favor your species, e.g., more grazing or more no disturbance cards.
%	Minimize the number of disturbance cards that most sets back the species.}

\subsubsection*{Play round two}

Your instructor will give you a different deck of event cards. Play a second round of Floristic Relay, using the new deck. Fill in the table and draw the diagram as you did for the first round. 

\question
Fill in Table~\ref{tab:round2_results} with the results of your game.

{\setlength{\LTcapwidth}{5.8in}
	\begin{longtable}{@{}|L{1.2in}|L{0.6in}|L{0.6in}|L{2in}|L{0.6in}|@{}}
		\caption{Second round. Record the number of each type of event and the number of individuals for each plant species where indicated.}\label{tab:round2_results}\tabularnewline
		\hline
		Event & Number of cards & Order of players	& Name of plant species	& Number of plants \tabularnewline
		\hline
		Fire & & 1st & & 6 plants \tabularnewline[0.25cm]
		\hline
		Grazing & & 2nd & & 5 plants \tabularnewline[0.25cm]
		\hline
		Landslide & & 3rd & & 4 plants\tabularnewline[0.25cm]
		\hline
		No Disturbance & & 4th & & 3 plants \tabularnewline[0.25cm]
		\hline
		& & 5th & & 2 plants \tabularnewline[0.25cm]
		\hline
		& & 6th & & 1 plant \tabularnewline[0.25cm]
		\hline
\end{longtable}}


\question
Draw your final community from round 2.  Select one member of your group to draw this community on the whiteboard at the front of the room. Draw the early successional species with one color and the late successional species with a different color. Your instructor will tell you which colors to use.

\vskip0pt plus 1fill

Answer the questions on the next page.

\newpage

\question
Study the communities drawn on the white board. Which community or communities (by number) showed the least disturbance? Which community or communities showed the most disturbance? Can you identify the type of disturance(s) for the disturbed communities? Explain below.\medskip

%\begin{parts}
%	\part Least disturbance? \AnswerBlank{Number depends on order drawn on board.} \medskip
%	
%	\part Fire? \AnswerBlank{Number depends on order drawn on board.} \medskip
%	
%	\part Grazing? \AnswerBlank{Number depends on order drawn on board.} \medskip
%	
%	\part Landslide? \AnswerBlank{Number depends on order drawn on board.}
%	
%\end{parts}

\AnswerBox{1\basespace}{Each community should be distinctive enough to reflect the most likely disturbance. Random events may affect the final outcome.}

\question[Checkout]
The Mooselet is an animal species that responds well to abundant grickle grass. You are a manager charged with increasing the population size of Mooselets. What management (disturbance) strategy would you use to increase the Mooselet population? Tell why.

\AnswerBox{2\basespace}{Grickle Grass responds favorably to grazing. Therefore, to increase
	mooselets, your management strategy should increase the frequency of grazing. More grazing
	should increase the abundance of grickle grass and therefore the abundance of mooselets.}


\vspace{\stretch{1}}

\end{questions}
	


\end{document}  